%!TEX root = ../dissertation.tex
% the abstract

The Higgs looks more Standard Model by the day.  The bulk of this thesis is vomiting up what amount to book reports of the main analysis documents, a technical paper, and a R1 quality theory steak hach\'e.

For a ``unique intellectual contribution,'' I made three different BDT's and went shake and bake to what I'm sure will be a set of inconclusive results of dubious actual scientific value.

If you insist, though:

This thesis describes variations on the two lepton channel of the Run-2 search for the SM Higgs boson produced in association with a vector boson using different variable sets for MVA training.  
The three variable sets in question are the set of variables from the fiducial analysis, a set based on the Lorentz Invariants (LI) concept, and a set based on a combination of masses and decay angles derived using the RestFrames (RF) package.
%The motivation for both the LI and RF variable schemes is a more complete description of a collision event in a more orthogonal basis of variablwes to both increase sensitivity and to reduce the overall impact of systematic uncertainties on fit quantities.
Aside from the variable sets used for MVA training and discirminant distributions, the analysis is otherwise identical to the fiducial anlaysis. %, employing the same simulation samples and collision data, object and event selection, signal and background modeling, sysematic uncertainties, and statistical treatment to extract the signal strength and other quantities of interest.
Both the LI and RF sets perform competitively on the basis of significances, with the RF set showing a $\sim3.5$\% improvement in expected fits to Asimov and data, though neither set boosts observed significance.
Both sets also reduce the observed error on $\hat{\mu}$, with the LI set reducing the error due to systematics by 7.5\% and the RF set doing so by 16\%.

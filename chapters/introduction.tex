%!TEX root = ../dissertation.tex
\begin{savequote}[75mm]
Your life has a limit but knowledge has none\ldots if you understand this and still strive for knowledge, you will be in danger for certain!
\qauthor{Zhuangzi}
\end{savequote}

\chapter{Introduction}

%JEH:  Could say a few more words to preface the H->bbbar – e.g. largest branching fraction, most likely place to observe quark coupling.
\newthought{Since the discovery} of a Standard Model (SM) like Higgs boson at the LHC in 2012\cite{:2012gk}\cite{:2012gu}, one of the main outstanding physics goals of the LHC has been to observe the primary SM Higgs decay mode, $H\to b\bar{b}$, with efforts primarily targeted at searching for Higgs bosons produced in association with a leptonically decaying vector ($W$ or $Z$, denoted generically as $V$) boson.  As the integrated luminosity of data collected at the LHC increases, $H\to b\bar{b}$ searches will increasingly become limited by the ability to constrain systematic uncertainties, with the latest result from ATLAS at $\sqrt{s}=13$ \TeV\, using \LUMI of $pp$ collision data already approaching this regime, having a $VH\left(b\bar{b}\right)$ signal strength of $1.20^{+0.24}_{-0.23}(\textrm{stat.})^{+0.34}_{-0.28}(\textrm{syst.})$ at $m_{H}=125$ \GeV\,\cite{paper}.

While this effort will likely require a combination of several different methods at various different stages in the analysis chain, one possible avenue forward is to revise the multivariate anlaysis (MVA) discriminant input variables used, as various schemes offer the promise of reducing systematic uncertainties through more efficient use of both actual and simulated collision data.  This thesis discusses two such alternate MVA schemes, the RestFrames (RF) and Lorentz Invarants (LI) variables, in the context of the 2-lepton channel of the Run 2 analysis in \cite{paper} and \cite{supportnote}, henceforth referred to as the ``fiducial analysis,'' before a brief discussion of combinations across channels and datasets.



%While this effort will likely require a combination of several different methods at various different stages in the analysis chain, one possible avenue forward is to revise the multivariate anlaysis (MVA) discriminant input variables used.  Novel variable sets often promise to increase performance in two ways.  The first is by having higher descriptive power, often through some sophistocated treatment of the missing transverse energy in an event, $E_T^{miss}$.  The second is through using a more orthogonal basis of description, which allows one to more efficiently use data and simulation samples.  \footnote{Heuristically, the more orthogonal one's basis, the less overlapping information the variables contain, and the more efficiently something like a numerical minimization can proceed.  Hence, even if the physical likelihood that something like an MVA is approximating is has the same discriminating power for two variable sets in a stats only context, a more orthogonal basis can allow for a more efficent exploration of the extra dimensionality added through systematic uncertainty terms in a typical analysis, mitigating the usual broadening and smearing of the likelihood from systematics and reducing errors on fit quantities of interest.}  This set of studies will seek to address the latter issue.

%In order to largely factor out the first issue, gains from better treatments of $E_T^{miss}$, a closed final state, the 2-lepton \ZH\,channel, will be studied here in an analysis that very closely mirrors the approach in \cite{supportnote} (henceforth refered to as the ``fiducial analysis'').  In addition to the standard variable set considered there, two additional variable sets, the ``Lorentz Invariant'' (LI) \cite{litalk} and ``RestFrames inspired'' (RF) variable \cite{rjr} are also studied.

Data and simulation samples used are described in Chapter \ref{ch:samples}.  Signal and background modeling with accompanying systematis are defined in Chapter \ref{ch:modeling}.  Object and event reconstruction definitions and event selection requirements are outlined in Chapter \ref{ch:object}.  The multivariate analysis, including a description of the LI and RF variable sets and a summary of performance  in the absence of systematic uncertainties, is described in Section \ref{ch:mva}.  The statistical fit model and systematic uncertainties are described in Section \ref{ch:fit}, and the fit results may be found in Chapter \ref{ch:results}.  Combining channels and datasets at different $\sqrt{s}$ values is discussed in the context of the Run 1 + Run 2 SM \vhbb\, combination in Chapter \ref{ch:comb}.  Finally, conclusions and closing thoughts are presented in Chapter \ref{ch:conclusion}.


Editorial notes: 
\begin{enumerate}
\item pdf will be \emph{probability} distrubtion function
\item PDF will be \emph{parton} distribution function
\item Unless otherwise stated, ATLAS and LHC/CERN images are from public available material from experiment webpages.  Copyright terms may be found here \url{https://atlas.cern/copyright}.
\end{enumerate}

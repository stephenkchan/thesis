%-------------------------------------------------------------------------------
% This file provides a skeleton ATLAS document.
%-------------------------------------------------------------------------------
% Specify where ATLAS LaTeX style files can be found.
\newcommand*{\ATLASLATEXPATH}{latex/}
% Use this variant if the files are in a central location, e.g. $HOME/texmf.
% \newcommand*{\ATLASLATEXPATH}{}
%-------------------------------------------------------------------------------
\documentclass[UKenglish,texlive=2013]{\ATLASLATEXPATH atlasdoc}
% The following command is needed by arXiv to ensure use of pdflatex.
% It should be included in the first 5 lines of the preamble.
% \pdfoutput=1
% The language of the document must be set: usually UKenglish or USenglish.
% british and american also work!
% Commonly used options:
%  texlive=YYYY          Specify TeX Live version (2013 is default).
%  atlasstyle=true|false Use ATLAS style for document (default).
%  coverpage             Create ATLAS draft cover page for collaboration circulation.
%                        See atlas-draft-cover.tex for a list of variables that should be defined.
%  cernpreprint          Create front page for a CERN preprint.
%                        See atlas-preprint-cover.tex for a list of variables that should be defined.
%  PAPER                 The document is an ATLAS paper (draft).
%  CONF                  The document is a CONF note (draft).
%  PUB                   The document is a PUB note (draft).
%  txfonts=true|false    Use txfonts rather than the default newtx - needed for arXiv submission.
%  paper=a4|letter       Set paper size to A4 (default) or letter.

%-------------------------------------------------------------------------------
% Extra packages:
\usepackage[subfigure=true]{\ATLASLATEXPATH atlaspackage}
\usepackage{\ATLASLATEXPATH atlaspackage}
\usepackage{multirow}
\usepackage{verbatim} % for begin/end comment commands
\usepackage{amsfonts}
\usepackage{lipsum}
\usepackage{lscape}
% Commonly used options:
%  biblatex=true|false   Use biblatex (default) or bibtex for the bibliography.
%  backend=biber         Use the biber backend rather than bibtex.
%  subfigure|subfig|subcaption  to use one of these packages for figures in figures.
%  minimal               Minimal set of packages.
%  default               Standard set of packages.
%  full                  Full set of packages.
%-------------------------------------------------------------------------------
% Style file with biblatex options for ATLAS documents.
\usepackage{\ATLASLATEXPATH atlasbiblatex}

% Package for creating list of authors and contributors to the analysis.
%\usepackage[xtab]{\ATLASLATEXPATH atlascontribute}

% Useful macros
\usepackage{\ATLASLATEXPATH atlasphysics}
% See doc/atlas-physics.pdf for a list of the defined symbols.
% Default options are:
%   true:  journal, misc, particle, unit, xref
%   false: BSM, hion, math, process, other, texmf
% See the package for details on the options.

% Files with references for use with biblatex.
% Note that biber gives an error if it finds empty bib files.
\addbibresource{RFLI_note.bib}
\addbibresource{bibtex/bib/ATLAS.bib}

% Paths for figures - do not forget the / at the end of the directory name.
\graphicspath{{logos/}{figures/}}

% Add you own definitions here (file RFLI_note-defs.sty).
\usepackage{RFLI_note-defs}

%-------------------------------------------------------------------------------
% Generic document information
%-------------------------------------------------------------------------------

% Title, abstract and document 
\AtlasTitle{Variations on MVA Variables in the SM $ZH\to \ell\ell b\bar{b}$ Search}

%\author{Hbb group}
%{\scriptsize
\usepackage{latex/atlascontribute}
\usepackage{authblk}
\renewcommand\Authands{, }
\renewcommand\Affilfont{\itshape\footnotesize}
\AtlasAuthorContributor{Chan, Stephen}{a}{Statistical analysis}
\AtlasAuthorContributor{Huth, John}{a}{Supervision}
\affil[a]{Harvard University}

%}
% Authors and list of contributors to the analysis
% \AtlasAuthorContributor also adds the name to the author list
% Include package latex/atlascontribute to use this
% Use authblk package if there are multiple authors, which is included by latex/atlascontribute
%\usepackage{authblk}
%\renewcommand\Authands{, } % avoid ``. and'' for last author
%\renewcommand\Affilfont{\itshape\small} % affiliation formatting
% \AtlasAuthorContributor{First AtlasAuthorContributor}{a}{Author's contribution.}
% \AtlasAuthorContributor{Second AtlasAuthorContributor}{b}{Author's contribution.}
% \AtlasAuthorContributor{Third AtlasAuthorContributor}{a}{Author's contribution.}
% \AtlasContributor{Fourth AtlasContributor}{Contribution to the analysis.}
%\author[a]{Stephen Chan}
%\author[a]{John Huth}
% \author[b]{Third Author}
%\affil[a]{Harvard University}
% \affil[b]{Another Institution}
%\AtlasAuthorContributor{Stephen Chan}{a}{Analysis}
%\AtlasAuthorContributor{John Huth}{a}{Supervision}
% \AtlasAuthorContributor{Second AtlasAuthorContributor}{b}{Author's contribution.}
%\affil[a]{Harvard University}

% If a special author list should be indicated via a link use the following code:
% Include the two lines below if you do not use atlasstyle:
% \usepackage[marginal,hang]{footmisc}
% \setlength{\footnotemargin}{0.5em}
% Use the following lines in all cases:
% \usepackage{authblk}
% \author{The ATLAS Collaboration%
% \thanks{The full author list can be found at:\newline
%   \url{https://atlas.web.cern.ch/Atlas/PUBNOTES/ATL-PHYS-PUB-2014-007/authorlist.pdf}}
% }

% Date: if not given, uses current date
%\date{\today}

% Draft version:
% Should be 1.0 for the first circulation, and 2.0 for the second circulation.
% If given, adds draft version on front page, a 'DRAFT' box on top of each other page, 
% and line numbers.
% Comment or remove in final version.
\AtlasVersion{1.5}

% ATLAS reference code, to help ATLAS members to locate the paper
%\AtlasRefCode{ATL-COM-PHYS-2016-XX}

% ATLAS note number. Can be an COM, INT, PUB or CONF note
% \AtlasNote{ATLAS-CONF-2014-XXX}
% \AtlasNote{ATL-PHYS-PUB-2014-XXX}
%\AtlasNote{ATL-COM-PHYS-2016-1724}

% CERN preprint number
% \PreprintIdNumber{CERN-PH-2014-XX}

% ATLAS date - arXiv submission; to be filled in by the Physics Office
% \AtlasDate{\today}

% arXiv identifier
% \arXivId{14XX.YYYY}

% HepData record
% \HepDataRecord{ZZZZZZZZ}

% Submission journal and final reference
% \AtlasJournal{Phys.\ Lett.\ B.}
% \AtlasJournalRef{\PLB 789 (2014) 123}
% \AtlasDOI{}

% Abstract - % directly after { is important for correct indentation
% same $pp$ collision data collected in 2015 and 2016 LHC Run-2 at $\sqrt{s}=13$~TeV, corresponding to an integrated luminosity of $\approx$ \LUMI. 
\AtlasAbstract{ %
This note describes variations on the two lepton channel of the Run-2 search for the SM Higgs boson produced in association with a vector boson using different variable sets for MVA training.  
The three variable sets in question are the set of variables from the fiducial analysis (utilizing the results of the kinematic fitter for the invariant masses of the $b$-jet pair and the lepton pair), a set based on the Lorentz Invariants (LI) concept, and a set based on a combination of masses and decay angles derived using the RestFrames (RF) package.
%The motivation for both the LI and RF variable schemes is a more complete description of a collision event in a more orthogonal basis of variablwes to both increase sensitivity and to reduce the overall impact of systematic uncertainties on fit quantities.
Aside from the variable sets used for MVA training and discirminant distributions, the analysis is otherwise identical to the fiducial anlaysis. %, employing the same simulation samples and collision data, object and event selection, signal and background modeling, sysematic uncertainties, and statistical treatment to extract the signal strength and other quantities of interest.
Both the LI and RF variable sets provide improvements with respect to the standard variable set, with the LI set showing an 11\% (17\%) improvement in expected (observed) significance on Run-2 data, and the RF set showing a 19\% (8\%) improvement in expected (observed) significance.
The error on the observed $\hat{\mu}$ signal strength from systematic uncertainties is similarly reduced in the LI and RF schemes, with the LI and RF sets showing a 11\% and 21\% reduction in systematic uncertainty on $\hat{\mu}$, respectively.
}

%-------------------------------------------------------------------------------
% The following information is needed for the cover page. The commands are only defined
% if you use the coverpage option in atlasdoc or use the atlascover package
%-------------------------------------------------------------------------------

% List of supporting notes  (leave as null \AtlasCoverSupportingNote{} if you want to skip this option)
% \AtlasCoverSupportingNote{Short title note 1}{https://cds.cern.ch/record/XXXXXXX}
% \AtlasCoverSupportingNote{Short title note 2}{https://cds.cern.ch/record/YYYYYYY}
%
% OR (the 2nd option is deprecated, especially for CONF and PUB notes)
%
% Supporting material TWiki page  (leave as null \AtlasCoverTwikiURL{} if you want to skip this option)
% \AtlasCoverTwikiURL{https://twiki.cern.ch/twiki/bin/view/Atlas/WebHome}

% Comment deadline
% \AtlasCoverCommentsDeadline{DD Month 2014}

% Analysis team members - contact editors should no longer be specified
% as there is a generic email list name for the editors
% \AtlasCoverAnalysisTeam{Peter Analyser, Susan Editor1, Jenny Editor2, Alphonse Physicien}

% Editorial Board Members - indicate the Chair by a (chair) after his/her name
% Give either all members at once (then they appear on one line), or separately
% \AtlasCoverEdBoardMember{EdBoard~Chair~(chair), EB~Member~1, EB~Member~2, EB~Member~3}
% \AtlasCoverEdBoardMember{EdBoard~Chair~(chair)}
% \AtlasCoverEdBoardMember{EB~Member~1}
% \AtlasCoverEdBoardMember{EB~Member~2}
% \AtlasCoverEdBoardMember{EB~Member~3}

% A PUB note has readers and not an EdBoard -- give their names here (one line or several entries)
% \AtlasCoverReaderMember{Reader~1, Reader~2}
% \AtlasCoverReaderMember{Reader~1}
% \AtlasCoverEdBoardMember{Reader~2}

% Editors egroup
% \AtlasCoverEgroupEditors{atlas-GROUP-2014-XX-editors@cern.ch}

% EdBoard egroup
% \AtlasCoverEgroupEdBoard{atlas-GROUP-2014-XX-editorial-board@cern.ch}


% Author and title for the PDF file
%\hypersetup{pdftitle={Search for a Standard Model Higgs boson produced in association with a vector boson and decaying to a pair of $b$-quarks},pdfauthor={add the list of names here}}

%-------------------------------------------------------------------------------
% Content
%-------------------------------------------------------------------------------
\begin{document}
\maketitle

%\tableofcontents
%\listoffigures
%\listoftables

% List of contributors - print here or after the Bibliography.
%\PrintAtlasContribute{0.30}

%-------------------------------------------------------------------------------
%\section{Updates expected during circulation}
%\label{sec:missingitems}
%\input{MissingItems.tex}
%-------------------------------------------------------------------------------
%-------------------------------------------------------------------------------
\section{Introduction}
\label{sec:intro}
\input{Introduction.tex}
%-------------------------------------------------------------------------------


%-------------------------------------------------------------------------------
%\section{ATLAS detector}
%\label{sec:detector}
%-------------------------------------------------------------------------------
%The ATLAS detector~\cite{PERF-2007-01} ...
% \input{atlas-detector}

%-------------------------------------------------------------------------------
\section{Data and Simulated Samples}
\label{sec:samples}
%!TEX root = ../dissertation.tex
\begin{savequote}[75mm]
If it's stupid but it works, it isn't stupid.
\qauthor{Conventional Wisdom}
\end{savequote}

\chapter{Data and Simulated Samples}

\newthought{Much has been said} 

The data and Monte Carlo simulation (MC) samples are the same as in the fiducial analysis.  The data corresponds to \LUMI of $pp$ collision data collected in 2015+16 at the ATLAS detector at $\sqrt{s}=13$ \TeV.  Only events recorded with all systems in ATLAS in good working order and passing certain quality requirements, according to a Good Run List (GRL), are analyzed.

Details about MC samples may be found in \cite{modelingnote}.  The \ZH\, process is considered for both multivariate analysis (MVA) optimization and the final statistical analysis, while $WH\to\ell\nu b\bar{b}$ and $ZH\to\nu\nu b\bar{b}$ production are included in the final statistical analysis only.  Signal MC samples were generated separately for $qq$ and $gg$ initiated $VH$ processes.  $qqVH$ samples were generated with \textsc{Powheg MiNLO + Pythia8} with the AZNLO tune set and NNPDF3.0 PDF.  %, with alternate samples generated using \textsc{MadGraph5\_aMC@NLO} for the hard scatter generation and \textsc{Pythia8} for the hardronization, parton shower (PS), underlying event (UE), and multiple parton interactions (MPI).
  Nominal $ggZH$ samples were generated using \textsc{Powheg} for the matrix element (ME) and \textsc{Pythia8} for the parton shower (PS), underlying event (UE), and multiple parton interactions (MPI), again applying the AZNLO tune and NNPDF3.0 PDF set. \cite{support17}

The background processes considered in these studies are $Z+$jets, \tt, and diboson production for both MVA optimization and the final statistical anlaysis and single top production and $W+$jets only considered in the final statistical analysis.  $V+$jets samples are generated using \textsc{Sherpa 2.2.1} \cite{support18} for both the ME and PS.  These samples are generated in different groups, according to the idenity of the $V$, the $\max\left(H_T,p_T^V\right)$ of events, and also further subdivided according to the flavor of the two leading jets in an event, $b$, $c$, or $l$, for a total of six categories..  \tt\, samples are generated using \textsc{Powheg} with the NNPDF3.0 PDF set interfaced with \textsc{Pythia8} using the NNPDF2.3 PDF's and the A14 tune \cite{support19}.  Single top samples use \textsc{Powheg} with the CT10 PDF's interfaced with \textsc{Pythia6} using the CTEQ6L1 PDF's \cite{support20,support21}.  Diboson samples are generated with \textsc{Sherpa 2.2.1} interfaced with the NNPDF3.0 NNLO PDF set normalized to NLO cross sections \cite{support22}.


%-------------------------------------------------------------------------------

%-------------------------------------------------------------------------------
\section{Event Reconstruction and Selection}
\label{sec:evtsel}
%!TEX root = ../dissertation.tex
\begin{savequote}[75mm]
If it's stupid but it works, it isn't stupid.
\qauthor{Conventional Wisdom}
\end{savequote}

\chapter{Object Definitions and Event Selection}

\newthought{Much has been said} 

\section{Event Selection and Analysis Regions}
This analysis focuses specifically on the 2-lepton channel of the fiducial analysis, with the event selection and analysis region definitions being identical.  Common to all lepton channels in the fiducial analysis is the set of requirements on the jets in a given event.  There must be at least two central jets and exactly two signal jets that have been ``$b$-tagged'' according to the MV2c10 algorithm \cite{btag}, with at least one of these $b$-jets having $p_T>45$ \GeV.  For MVA training and certain background samples, a process known as ``truth-tagging'' is applied instead of the standard $b$-tagging to boost sample statistics and stabilize training/fits (cf. \cite{supportnote} Section 4.2 for details).  After event selection, the \emph{muon-in-jet} and \emph{PtReco} corrections, described in \cite{objectnote} 6.3.3-4, are applied to the $b$-jets.

In addition to the common selections, there are 2-lepton specific selections.  All events are required to pass an un-prescaled single lepton trigger, a full list of which may be found in Tables 5 and 6 of \cite{objectnote} with the requirement that one of the two selected leptons in the event must have fired the trigger.  There must be 2 VH-loose leptons, and at least one of these must be a ZH-signal lepton (cf. Tables \ref{tbl:elecsel} and \ref{tbl:muonsel} for definitions).  This lepton pair must have an invariant mass between 81 and 101 \GeV.  In addition to the jet corrections described above, a kinematic fitter is applied to the leptons and two leading corrected jets in an event with three or fewer jets\footnote{The gain from using the kinematic fitter is found to be smeared out in events with higher jet multiplicities.} to take advantage of the fact that the 2-lepton final state is closed (cf. \cite{run1note,epsJetRes}); these objects are only used for MVA training/fit inputs.  

In order to increase analysis sensitivity, the analysis is split into orthogonal regions based on the number of jets and the transverse momentum of the $Z$ candidate (the vectoral sum of the lepton pair; this $p_T$ is denoted \ptv): 2 and $\ge3$ jets; $p_T^V$ in $\left[75,150\right),\left[150,\infty\right)$ \GeV.  In addition to the signal regions where the leptons are required to be the same flavor ($e$ or $\mu$), there are top $e-\mu$ control regions used to constrain the top backgrounds.

All of these requirements are summarized in \ref{tab:evsel}.

\begin{table}[!htbp]
  \begin{center}\begin{tabular}{cc}
      \hline\hline
      Category & Requirement\\
      \hline
      Trigger & un-prescaled, single lepton\\
      Jets & $\ge2$ central jets; 2 $b$-tagged signal jets, harder jet with $p_T>45$ \GeV\\
      Leptons & 2 VH-loose leptons ($\ge1$ ZH-signal lepton); same (opp) flavor for SR (CR)\\
      \mll & $\mll\in\left(81,101\right)$ GeV\\
      \ptv regions (\GeV) & $\left[75,150\right),\left[150,\infty\right)$\\
      \hline\hline
    \end{tabular}
    \caption{Event selectrion requirements}
  \end{center}
  \label{tab:evsel}
\end{table}


%-------------------------------------------------------------------------------
% should contain at least
% - object definition
% - MVA preselection
% - cut-based analysis additional selection

%-------------------------------------------------------------------------------
\section{Multi-Variate Analysis}
\label{sec:mva}
\input{MVA.tex}

%-------------------------------------------------------------------------------

%-------------------------------------------------------------------------------
\section{Systematic Uncertainties and Statistical Treatment}
\label{sec:sys}
\input{Fit.tex}
%-------------------------------------------------------------------------------

\clearpage
\newpage
%-------------------------------------------------------------------------------
\section{Fit Results}
\label{sec:result}
%!TEX root = ../dissertation.tex
\begin{savequote}[75mm]
Kein Operationsplan reicht mit einiger Sicherheit \"uber das erste Zusammentreffen mit der feindlichen Hauptmacht hinaus.
\qauthor{Helmuth von Moltke}
\end{savequote}

\chapter{Fit Results}

\newthought{The results in this chapter} were first reported in \cite{rflinote} and describe how the three different fit models detailed and validated Chapter \ref{ch:fit}, corresponding to the standard, RF, and LI variable sets described in Chapter \ref{ch:mva} perform on actual $VH$ fits.  In particular sensitivities, nuisance parameter impacts, and signal strengths on expected fits to Asimov datasets and both expected and observed fits on the actual \lumi dataset are compared.

  Expected and observed sensitivities for the different variable sets may be found in Table \ref{tab:Sensitivities}.  The RF fits feature the highest expected sensitivities, outperforming the standard set by 3.5\% and 3.4\% for fits to Asimov and observed datasets, respectively.  The LI variable has a lower significance than both for expected fits to both Asimov and data with a 6.7\% (1.7\%) significance than the standard set for the Asimov (observed) dataset.  While the fit using standard variables does have a higher observed significance than both the LI and RF fits, by 2.8\% and 8.6\%, respectively, these numbers should be viewed in the context of the best fit $\hat{\mu}$ values, discussed below.  That is, the standard set may yield the highest sensitivity for this particular dataset, but this is not necessarily (and likely is not) the case for any given dataset.

\begin{table}[!htbp]\captionsetup{justification=centering}
\begin{center}
\begin{tabular}{lccc}
\hline\hline
 & Standard &LI &RF\\
\hline
Expected (Asimov) & 2.06 & 1.92 & 2.13\\
\hline
Expected (data) & 1.76 & 1.73 & 1.80\\
\hline
Observed (data) & 2.87 & 2.79 & 2.62\\
\hline
\hline
\end{tabular}
\end{center}
\caption{Expected (for both data and Asimov) and observed significances for the standard, LI, and RF variable sets.}
\label{tab:Sensitivities}
\end{table}

A summary of fitted signal strengths and errors for both the Asimov (a) and observed (b) datasets are shown in Figure \ref{fig:MuhatSummary}.\footnote{For reference, the standalone 2-lepton fit from the fiducial analysis is $2.11^{+0.50}_{-0.48}(\textrm{stat.})^{+0.64}_{-0.47}(\textrm{syst.})$}  A summary of error breakdowns is given in Tables \ref{tab:bd-asi-summary} (Asimov) and \ref{tab:bd-obs-summary} (observed) for total error, data statistics contributions, total systematic error contributions, and categories for which the total impact is $\ge0.1$ for the standard fit.  As is to be expected for both the Asimov and observed dataset fits, the contribution to the total error on $\mu$ arising from data statistics is nearly identical, since each set of fits uses the same selections and data.\footnote{Though not exactly identical.  Since the BDT's are different for the different variable sets, the binning (as determined by transformation D) and bin contents in each set are generally different, leading to slightly different data statistics errors.}

\begin{table}[!htbp]\captionsetup{justification=centering}
\begin{center}
\begin{tabular}{lccc}
\hline\hline
 &Std-KF &LI+MET &RF\\
\hline
Total &  +0.608 / -0.511  &  +0.632 / -0.539  &  +0.600 / -0.494 \\
\hline
DataStat &  +0.420 / -0.401  &  +0.453 / -0.434  &  +0.424 / -0.404 \\
\hline
FullSyst &  +0.440 / -0.318  &  +0.441 / -0.319  &  +0.425 / -0.284 \\
\hline
Signal Systematics &  +0.262 / -0.087  &  +0.272 / -0.082  &  +0.290 / -0.088 \\
\hline
MET &  +0.173 / -0.091  &  +0.140 / -0.074  &  +0.117 / -0.063 \\
\hline
Flavor Tagging &  +0.138 / -0.136  &  +0.069 / -0.071  &  +0.076 / -0.078 \\
\hline
Model Zjets &  +0.119 / -0.117  &  +0.124 / -0.127  &  +0.095 / -0.086 \\
\hline
\end{tabular}
\caption{Summary of error impacts on total $\mu$ error for principal categories in the Asimov standard, LI, and RF fits.}
\label{tab:bd-asi-summary}
\end{center}
\end{table}

\begin{table}[!htbp]\captionsetup{justification=centering}
\begin{center}
\begin{tabular}{lccc}
\hline\hline
 &Std-KF &LI+MET &RF\\
\hline
Total &  +0.811 / -0.662  &  +0.778 / -0.641  &  +0.731 / -0.612 \\
\hline
DataStat &  +0.502 / -0.484  &  +0.507 / -0.489  &  +0.500 / -0.481 \\
\hline
FullSyst &  +0.637 / -0.451  &  +0.591 / -0.415  &  +0.533 / -0.378 \\
\hline
Signal Systematics &  +0.434 / -0.183  &  +0.418 / -0.190  &  +0.364 / -0.152 \\
\hline
MET &  +0.209 / -0.130  &  +0.190 / -0.102  &  +0.152 / -0.077 \\
\hline
Flavor Tagging &  +0.162 / -0.166  &  +0.093 / -0.070  &  +0.115 / -0.099 \\
\hline
Model Zjets &  +0.164 / -0.152  &  +0.141 / -0.143  &  +0.101 / -0.105 \\
\hline
\end{tabular}
\caption{Summary of error impacts on total $\hat{\mu}$ error for principal categories in the observed standard, LI, and RF fits.}
\label{tab:bd-obs-summary}
\end{center}
\end{table}


The contribution from systematic uncertainties, however, does vary considerably across the variable sets.  The Asimov fits are a best case scenario in the sense that, by construction, all NP's are equal to their predicted values (and so no ``penalty'' is paid for pulls on Gaussian NP's).  The systematics error from the LI fit is slightly higher (subprecent) than that from the standard fit, and 4.6\% higher error overall due to differences in data stats.  The RF Asimov fit, however, has a 6.5\% lower total error from systematics than the standard Asimov fit (and a 2.2\% lower error overall).  Moreover, for both the LI and RF sets, errors are markedly smaller for the MET and Flavor Tagging categories, with the RF fit also featuring a smaller errors on $Z+$jets modeling; the only notable exception to this trend in Asimov fits are the signal systematics.

These trends are more pronounced in the observed fits.  As can be seen in Table \ref{tab:bd-obs-summary}, both the LI and RF fits have smaller errors from systematic uncertainties, both overall and in all principal categories, with the LI and RF fits having 7.5\% (3.7\%) and 16\% (8.8\%) lower systematics (total) error on $\hat{\mu}$, respectively.

\begin{figure}[!htbp]\captionsetup{justification=centering}
  \centering
  \begin{subfigure}[t]{0.49\textwidth}\centering\includegraphics[width=\textwidth]{figures/Plot_mu_rfli-prod-exp}\caption{Asimov}\end{subfigure}
  \begin{subfigure}[t]{0.49\textwidth}\centering\includegraphics[width=\textwidth]{figures/Plot_mu_rfli-prod-obs}\caption{Observed}\end{subfigure}
  \caption{$\mu$ summary plots for the standard, LI, and RF variable sets.  The Asimov case (with $\mu=1$ by construction) is in (a), and $\hat{\mu}$ best fit values and error summary are in (b).}
  \label{fig:MuhatSummary}
\end{figure}

Studying the performance of the Lorentz Invariant and RestFrames variable sets at both a data statistics only context and with the full fit model in the \ZH\, channel of the $VH\left(b\bar{b}\right)$ analysis suggests that these variables may offer a potential method for better constraining systematic uncertainties in $VH\left(b\bar{b}\right)$ searches as more orthogonal bases in describing the information in collision events.  

The marginally worse performance of the LI and RF variables (7.9\% and 6.9\%, respectively) with respect to the standard variable at a stats only level illustrates that neither variable set has greater intrinsic descriptive power in the absence of systematics in this closed final state.  Hence, any gains from either of these variable sets in a full fit come from improved treatment of systematic uncertainties.

With full systematics, the LI variable set narrows the sensitivity gap somewhat, with lower significances by 6.7\% (1.7\%) on expected fits to Asimov (data) and by 2.8\% on observed significances.  The RF variable set outperforms the standard set in expected fits with 3.5\% (3.4\%) higher significance on Asimov (data), but has an 8.6\% lower observed significance, though the observed significances should be viewed in the context of observed $\hat{\mu}$ values.

Moreover, the LI and RF variable sets generally perform better in the context of the error on $\mu$.  The LI fit is comparable to the standard set on Asimov data and has a 7.5\% lower total systematics error on $\hat{\mu}$ on observed data, while the RF fit is lower in both cases, with systematics error being 6.5\% (16\%) lower on Asimov (observed) data.  


\begin{comment}
A summary of performance metrics in this document may be found in Table \ref{tab:kahuna}.  


\begin{table}[!htbp]\captionsetup{justification=centering}
\begin{center}
\begin{tabular}{lccc}
\hline\hline
 & Standard &LI &RF\\
\hline
$\hat{\mu}$ & $1.75^{+0.24}_{-0.23}(\textrm{stat.})^{+0.34}_{-0.28}(\textrm{syst.})$ & $1.65^{+0.24}_{-0.23}(\textrm{stat.})^{+0.34}_{-0.28}(\textrm{syst.})$ & $1.50^{+0.24}_{-0.23}(\textrm{stat.})^{+0.34}_{-0.28}(\textrm{syst.})$\\
Asi. $\Delta err\left(\mu\right)$ &  --- & $<1$\%, +4.6\% & -6.5\%, -2.2\%\\
Obs. $\Delta err\left(\hat{\mu}\right)$ &  --- & -7.5\%, -3.7\% & -16\%, -8.8\%\\
\hline
Stat only sig. & 4.78 & 4.39 (-7.9\%) & 4.44 (-6.9\%)\\
Exp. (Asi.) sig. & 2.06 & 1.92 (-6.7\%) & 2.13 (+3.5\%)\\
Exp. (data) sig. & 1.76 & 1.73 (-1.7\%) & 1.80 (+3.4\%)\\
Obs. (data) sig. & 2.87 & 2.79 (-2.8\%) & 2.62 (-8.6\%)\\
\hline\hline
\end{tabular}
\end{center}
\caption{Summary of performance figures for the standard, LI, and RF variable sets.  In the case of the latter two, \% differences are given where relevant.  Differences in errors on $\mu$ are on full systematics and total error, respectively.}
\label{tab:kahuna}
\end{table}
\end{comment}

These figures of merit suggest that both the LI and RF variables are more orthogonal than the standard variable set used in the fiducial analysis.  Moreover, the RF variable set does seem to consistently perform better than the LI set.  Furthermore, both variable sets have straightforward extensions to the other lepton channels in the \vhbb\, analysis.  The magnitude of any gain from the more sophistocated treatment of $E_T^{miss}$ in these extensions is beyond the scope of these studies, but the performance in this closed final state do suggest that there is some value to be had in these non-standard descriptions independent of these considerations.

%A search for the decay of a Standard Model Higgs boson into $b\bar{b}$ pair when produced in association with a $W$ and $Z$ boson has been performed with Run1 and Run2 full data. In Run2 dataset, the measured signal strength with respect to the SM expectation is found to be $1.20^{+0.24}_{-0.23}(\textrm{stat.})^{+0.34}_{-0.28}(\textrm{syst.})$ at $m_{H}=125$ GeV. The observed (expected) significance is 3.5(3.0) standard deviations. The analysis has been validated by measuring the signal strength of $V(W/Z)Z(\rightarrow b\bar{b})$. The measured signal strength is $1.11^{+0.12}_{-0.11}(\textrm{stat.})^{+0.22}_{-0.19}(\textrm{syst.})$, corresponding to observed(expected) significance of 5.8(5.3) standard deviations.  

%This result based on Run2 data has been also combined with previous results on the Run1 dataset. The observed(expected) significance combined Run1 and Run2 is 3.6(4.0) standard deviation. The measured signal strength is $0.90^{+0.18}_{-0.18}(\textrm{stat.})^{+0.21}_{-0.19}(\textrm{syst.})$. 


%-------------------------------------------------------------------------------


% should contain, if available
% - nominal results
% - cross-checks: diboson fit, cut-based analysis results

%-------------------------------------------------------------------------------

\section{Conclusions}
\label{sec:conclusion}
\input{Conclusion.tex}
%-------------------------------------------------------------------------------


%%-------------------------------------------------------------------------------
%\section*{Acknowledgements}
%%-------------------------------------------------------------------------------
%
%%\input{acknowledgements/Acknowledgements}
%
%The \texttt{atlaslatex} package contains the acknowledgements that were valid 
%at the time of the release you are using.
%These can be found in the \texttt{acknowledgements} subdirectory.
%When your ATLAS paper or PUB/CONF note is ready to be published,
%download the latest set of acknowledgements from:\\
%\url{https://twiki.cern.ch/twiki/bin/view/AtlasProtected/PubComAcknowledgements}
%
%The supporting notes for the analysis should also contain a list of contributors.
%This information should usually be included in \texttt{mydocument-metadata.tex}.
%The list should be printed either here or before the table of contents.


%-------------------------------------------------------------------------------
\clearpage
\newpage
\appendix
\part*{Appendices}
\addcontentsline{toc}{part}{Appendices}
%-------------------------------------------------------------------------------

%A number of cross-checks and additional plots are provided in the Appendices. Moreover, additional cross-checks, not fully documented here, can also be found in \url{https://twiki.cern.ch/twiki/bin/view/AtlasProtected/HSG5VHbbICHEPCheckList}.
\section{Full MVA Plots}
\label{app:mvaplots}
Correlation, ranking, and input variable plots for the standard, Lorentz Invariant, and RestFrames variable sets.

\begin{figure}[!htbp]
  \centering
    \subfigure[2 jet, low pTV (S)]{\includegraphics[width=0.220000\linewidth]{figures/diag/std-kf-2tag2jet-75_150ptv_0of2/CorrelationMatrixS.pdf}}
    \subfigure[2 jet, low pTV (B)]{\includegraphics[width=0.220000\linewidth]{figures/diag//std-kf-2tag2jet-75_150ptv_0of2/CorrelationMatrixB.pdf}}
    \subfigure[3+ jet, low pTV (S)]{\includegraphics[width=0.220000\linewidth]{figures/diag//std-kf-2tag3jet-75_150ptv_0of2/CorrelationMatrixS.pdf}}
    \subfigure[3+ jet, low pTV (B)]{\includegraphics[width=0.220000\linewidth]{figures/diag//std-kf-2tag3jet-75_150ptv_0of2/CorrelationMatrixB.pdf}}
    \subfigure[2 jet, high pTV (S)]{\includegraphics[width=0.220000\linewidth]{figures/diag//std-kf-2tag2jet-150ptv_0of2/CorrelationMatrixS.pdf}}
    \subfigure[2 jet, high pTV (B)]{\includegraphics[width=0.220000\linewidth]{figures/diag//std-kf-2tag2jet-150ptv_0of2/CorrelationMatrixB.pdf}}
    \subfigure[3+ jet, high pTV (S)]{\includegraphics[width=0.220000\linewidth]{figures/diag//std-kf-2tag3jet-150ptv_0of2/CorrelationMatrixS.pdf}}
    \subfigure[3+ jet, high pTV (B)]{\includegraphics[width=0.220000\linewidth]{figures/diag//std-kf-2tag3jet-150ptv_0of2/CorrelationMatrixB.pdf}}
  \caption{Signal and background variable correlations for the standard variable set.}
  \label{fig:std-kf-Correlations}
\end{figure}

\begin{figure}[!htbp]
  \centering
  \subfigure[2 jet, low pTV]{\includegraphics[width=0.49\linewidth]{figures/diag//training-std-kf-2tag2jet-75_150ptv_tD.pdf}}
    \subfigure[3+ jet, low pTV]{\includegraphics[width=0.49\linewidth]{figures/diag//training-std-kf-2tag3jet-75_150ptv_tD.pdf}}
    \subfigure[2 jet, high pTV]{\includegraphics[width=0.49\linewidth]{figures/diag//training-std-kf-2tag2jet-150ptv_tD.pdf}}
    \subfigure[3+ jet, high pTV]{\includegraphics[width=0.49\linewidth]{figures/diag//training-std-kf-2tag3jet-150ptv_tD.pdf}}
  \caption{Signal and background variable correlations for the standard variable set.}
  \label{fig:std-kf-Ranking}
\end{figure}

\begin{figure}[!htbp]
  \centering
  \subfigure[2 jet, low pTV]{\includegraphics[width=0.49\linewidth]{figures/diag/bdt_dist-test-std-kf-2tag2jet-75_150ptv_tD-all.pdf}}
    \subfigure[3+ jet, low pTV]{\includegraphics[width=0.49\linewidth]{figures/diag/bdt_dist-test-std-kf-2tag3jet-75_150ptv_tD-all.pdf}}
    \subfigure[2 jet, high pTV]{\includegraphics[width=0.49\linewidth]{figures/diag/bdt_dist-test-std-kf-2tag2jet-150ptv_tD-all.pdf}}
    \subfigure[3+ jet, high pTV]{\includegraphics[width=0.49\linewidth]{figures/diag/bdt_dist-test-std-kf-2tag3jet-150ptv_tD-all.pdf}}
  \caption{Training (points) and testing (block histogram) MVA distributions used for stat only testing for the standard variable set.}
  \label{fig:std-kf-testing}
\end{figure}

\begin{figure}[!htbp]
  \centering
    \subfigure[2 jet, low pTV (1/2)]{\includegraphics[width=0.49\linewidth]{figures/diag//std-kf-2tag2jet-75_150ptv_0of2/variables_id_c1.pdf}}
    \subfigure[2 jet, low pTV (2/2)]{\includegraphics[width=0.49\linewidth]{figures/diag//std-kf-2tag2jet-75_150ptv_0of2/variables_id_c2.pdf}}
    \subfigure[3+ jet, low pTV (1/2)]{\includegraphics[width=0.49\linewidth]{figures/diag//std-kf-2tag3jet-75_150ptv_0of2/variables_id_c1.pdf}}
    \subfigure[3+ jet, low pTV (2/2)]{\includegraphics[width=0.49\linewidth]{figures/diag//std-kf-2tag3jet-75_150ptv_0of2/variables_id_c2.pdf}}
    \subfigure[2 jet, high pTV (1/2)]{\includegraphics[width=0.49\linewidth]{figures/diag//std-kf-2tag2jet-150ptv_0of2/variables_id_c1.pdf}}
    \subfigure[2 jet, high pTV (2/2)]{\includegraphics[width=0.49\linewidth]{figures/diag//std-kf-2tag2jet-150ptv_0of2/variables_id_c2.pdf}}
    \subfigure[3+ jet, high pTV (1/2)]{\includegraphics[width=0.49\linewidth]{figures/diag//std-kf-2tag3jet-150ptv_0of2/variables_id_c1.pdf}}
    \subfigure[3+ jet, high pTV (2/2)]{\includegraphics[width=0.49\linewidth]{figures/diag//std-kf-2tag3jet-150ptv_0of2/variables_id_c2.pdf}}
  \caption{Input variables for the standard variable set.}
  \label{fig:std-kf-inputs}
\end{figure}


\begin{figure}[!htbp]
  \centering
    \subfigure[2 jet, low pTV (S)]{\includegraphics[width=0.220000\linewidth]{figures/diag/li-met-2tag2jet-75_150ptv_0of2/CorrelationMatrixS.pdf}}
    \subfigure[2 jet, low pTV (B)]{\includegraphics[width=0.220000\linewidth]{figures/diag//li-met-2tag2jet-75_150ptv_0of2/CorrelationMatrixB.pdf}}
    \subfigure[3+ jet, low pTV (S)]{\includegraphics[width=0.220000\linewidth]{figures/diag//li-met-2tag3jet-75_150ptv_0of2/CorrelationMatrixS.pdf}}
    \subfigure[3+ jet, low pTV (B)]{\includegraphics[width=0.220000\linewidth]{figures/diag//li-met-2tag3jet-75_150ptv_0of2/CorrelationMatrixB.pdf}}
    \subfigure[2 jet, high pTV (S)]{\includegraphics[width=0.220000\linewidth]{figures/diag//li-met-2tag2jet-150ptv_0of2/CorrelationMatrixS.pdf}}
    \subfigure[2 jet, high pTV (B)]{\includegraphics[width=0.220000\linewidth]{figures/diag//li-met-2tag2jet-150ptv_0of2/CorrelationMatrixB.pdf}}
    \subfigure[3+ jet, high pTV (S)]{\includegraphics[width=0.220000\linewidth]{figures/diag//li-met-2tag3jet-150ptv_0of2/CorrelationMatrixS.pdf}}
    \subfigure[3+ jet, high pTV (B)]{\includegraphics[width=0.220000\linewidth]{figures/diag//li-met-2tag3jet-150ptv_0of2/CorrelationMatrixB.pdf}}
  \caption{Signal and background variable correlations for the LI variable set.}
  \label{fig:li-met-Correlations}
\end{figure}

\begin{figure}[!htbp]
  \centering
  \subfigure[2 jet, low pTV]{\includegraphics[width=0.49\linewidth]{figures/diag//training-li-met-2tag2jet-75_150ptv_tD.pdf}}
    \subfigure[3+ jet, low pTV]{\includegraphics[width=0.49\linewidth]{figures/diag//training-li-met-2tag3jet-75_150ptv_tD.pdf}}
    \subfigure[2 jet, high pTV]{\includegraphics[width=0.49\linewidth]{figures/diag//training-li-met-2tag2jet-150ptv_tD.pdf}}
    \subfigure[3+ jet, high pTV]{\includegraphics[width=0.49\linewidth]{figures/diag//training-li-met-2tag3jet-150ptv_tD.pdf}}
  \caption{Signal and background variable correlations for the LI variable set.}
  \label{fig:li-met-Ranking}
\end{figure}

\begin{figure}[!htbp]
  \centering
  \subfigure[2 jet, low pTV]{\includegraphics[width=0.49\linewidth]{figures/diag/bdt_dist-test-li-met-2tag2jet-75_150ptv_tD-all.pdf}}
    \subfigure[3+ jet, low pTV]{\includegraphics[width=0.49\linewidth]{figures/diag/bdt_dist-test-li-met-2tag3jet-75_150ptv_tD-all.pdf}}
    \subfigure[2 jet, high pTV]{\includegraphics[width=0.49\linewidth]{figures/diag/bdt_dist-test-li-met-2tag2jet-150ptv_tD-all.pdf}}
    \subfigure[3+ jet, high pTV]{\includegraphics[width=0.49\linewidth]{figures/diag/bdt_dist-test-li-met-2tag3jet-150ptv_tD-all.pdf}}
  \caption{Training (points) and testing (block histogram) MVA distributions used for stat only testing for the LI variable set.}
  \label{fig:li-met-testing}
\end{figure}

\begin{figure}[!htbp]
  \centering
    \subfigure[2 jet, low pTV (1/2)]{\includegraphics[width=0.49\linewidth]{figures/diag//li-met-2tag2jet-75_150ptv_0of2/variables_id_c1.pdf}}
    \subfigure[2 jet, low pTV (2/2)]{\includegraphics[width=0.49\linewidth]{figures/diag//li-met-2tag2jet-75_150ptv_0of2/variables_id_c2.pdf}}
    \subfigure[3+ jet, low pTV (1/2)]{\includegraphics[width=0.49\linewidth]{figures/diag//li-met-2tag3jet-75_150ptv_0of2/variables_id_c1.pdf}}
    \subfigure[3+ jet, low pTV (2/2)]{\includegraphics[width=0.49\linewidth]{figures/diag//li-met-2tag3jet-75_150ptv_0of2/variables_id_c2.pdf}}
    \subfigure[2 jet, high pTV (1/2)]{\includegraphics[width=0.49\linewidth]{figures/diag//li-met-2tag2jet-150ptv_0of2/variables_id_c1.pdf}}
    \subfigure[2 jet, high pTV (2/2)]{\includegraphics[width=0.49\linewidth]{figures/diag//li-met-2tag2jet-150ptv_0of2/variables_id_c2.pdf}}
    \subfigure[3+ jet, high pTV (1/2)]{\includegraphics[width=0.49\linewidth]{figures/diag//li-met-2tag3jet-150ptv_0of2/variables_id_c1.pdf}}
    \subfigure[3+ jet, high pTV (2/2)]{\includegraphics[width=0.49\linewidth]{figures/diag//li-met-2tag3jet-150ptv_0of2/variables_id_c2.pdf}}
  \caption{Input variables for the LI variable set.}
  \label{fig:li-met-inputs}
\end{figure}


\begin{figure}[!htbp]
  \centering
    \subfigure[2 jet, low pTV (S)]{\includegraphics[width=0.220000\linewidth]{figures/diag/rf-sel-2tag2jet-75_150ptv_0of2/CorrelationMatrixS.pdf}}
    \subfigure[2 jet, low pTV (B)]{\includegraphics[width=0.220000\linewidth]{figures/diag//rf-sel-2tag2jet-75_150ptv_0of2/CorrelationMatrixB.pdf}}
    \subfigure[3+ jet, low pTV (S)]{\includegraphics[width=0.220000\linewidth]{figures/diag//rf-sel-2tag3jet-75_150ptv_0of2/CorrelationMatrixS.pdf}}
    \subfigure[3+ jet, low pTV (B)]{\includegraphics[width=0.220000\linewidth]{figures/diag//rf-sel-2tag3jet-75_150ptv_0of2/CorrelationMatrixB.pdf}}
    \subfigure[2 jet, high pTV (S)]{\includegraphics[width=0.220000\linewidth]{figures/diag//rf-sel-2tag2jet-150ptv_0of2/CorrelationMatrixS.pdf}}
    \subfigure[2 jet, high pTV (B)]{\includegraphics[width=0.220000\linewidth]{figures/diag//rf-sel-2tag2jet-150ptv_0of2/CorrelationMatrixB.pdf}}
    \subfigure[3+ jet, high pTV (S)]{\includegraphics[width=0.220000\linewidth]{figures/diag//rf-sel-2tag3jet-150ptv_0of2/CorrelationMatrixS.pdf}}
    \subfigure[3+ jet, high pTV (B)]{\includegraphics[width=0.220000\linewidth]{figures/diag//rf-sel-2tag3jet-150ptv_0of2/CorrelationMatrixB.pdf}}
  \caption{Signal and background variable correlations for the RF variable set.}
  \label{fig:rf-sel-Correlations}
\end{figure}

\begin{figure}[!htbp]
  \centering
  \subfigure[2 jet, low pTV]{\includegraphics[width=0.49\linewidth]{figures/diag//training-rf-sel-2tag2jet-75_150ptv_tD.pdf}}
    \subfigure[3+ jet, low pTV]{\includegraphics[width=0.49\linewidth]{figures/diag//training-rf-sel-2tag3jet-75_150ptv_tD.pdf}}
    \subfigure[2 jet, high pTV]{\includegraphics[width=0.49\linewidth]{figures/diag//training-rf-sel-2tag2jet-150ptv_tD.pdf}}
    \subfigure[3+ jet, high pTV]{\includegraphics[width=0.49\linewidth]{figures/diag//training-rf-sel-2tag3jet-150ptv_tD.pdf}}
  \caption{Signal and background variable correlations for the RF variable set.}
  \label{fig:rf-sel-Ranking}
\end{figure}

\begin{figure}[!htbp]
  \centering
  \subfigure[2 jet, low pTV]{\includegraphics[width=0.49\linewidth]{figures/diag/bdt_dist-test-rf-sel-2tag2jet-75_150ptv_tD-all.pdf}}
    \subfigure[3+ jet, low pTV]{\includegraphics[width=0.49\linewidth]{figures/diag/bdt_dist-test-rf-sel-2tag3jet-75_150ptv_tD-all.pdf}}
    \subfigure[2 jet, high pTV]{\includegraphics[width=0.49\linewidth]{figures/diag/bdt_dist-test-rf-sel-2tag2jet-150ptv_tD-all.pdf}}
    \subfigure[3+ jet, high pTV]{\includegraphics[width=0.49\linewidth]{figures/diag/bdt_dist-test-rf-sel-2tag3jet-150ptv_tD-all.pdf}}
  \caption{Training (points) and testing (block histogram) MVA distributions used for stat only testing for the RF variable set.}
  \label{fig:rf-sel-testing}
\end{figure}

\begin{figure}[!htbp]
  \centering
    \subfigure[2 jet, low pTV (1/2)]{\includegraphics[width=0.49\linewidth]{figures/diag//rf-sel-2tag2jet-75_150ptv_0of2/variables_id_c1.pdf}}
    \subfigure[2 jet, low pTV (2/2)]{\includegraphics[width=0.49\linewidth]{figures/diag//rf-sel-2tag2jet-75_150ptv_0of2/variables_id_c2.pdf}}
    \subfigure[3+ jet, low pTV (1/2)]{\includegraphics[width=0.49\linewidth]{figures/diag//rf-sel-2tag3jet-75_150ptv_0of2/variables_id_c1.pdf}}
    \subfigure[3+ jet, low pTV (2/2)]{\includegraphics[width=0.49\linewidth]{figures/diag//rf-sel-2tag3jet-75_150ptv_0of2/variables_id_c2.pdf}}
    \subfigure[2 jet, high pTV (1/2)]{\includegraphics[width=0.49\linewidth]{figures/diag//rf-sel-2tag2jet-150ptv_0of2/variables_id_c1.pdf}}
    \subfigure[2 jet, high pTV (2/2)]{\includegraphics[width=0.49\linewidth]{figures/diag//rf-sel-2tag2jet-150ptv_0of2/variables_id_c2.pdf}}
    \subfigure[3+ jet, high pTV (1/2)]{\includegraphics[width=0.49\linewidth]{figures/diag//rf-sel-2tag3jet-150ptv_0of2/variables_id_c1.pdf}}
    \subfigure[3+ jet, high pTV (2/2)]{\includegraphics[width=0.49\linewidth]{figures/diag//rf-sel-2tag3jet-150ptv_0of2/variables_id_c2.pdf}}
  \caption{Input variables for the RF variable set.}
  \label{fig:rf-sel-inputs}
\end{figure}


\clearpage
\section{Fit Validation and  Postfit Plots}
\label{app:postfit}
\subsection{Full Breakdown of Errors}

A postfit ranking of nuisance parameters according to their impact on $\hat{\mu}$ for the different variable sets may be found in Figure \ref{fig:npranking}.

\begin{figure}[!htbp]
    \subfigure[Standard]{\includegraphics[width=0.300000\linewidth]{figures/RFLI_std-kf_v04-v04_rank_RFLI_std-kf_v04-v04_pulls_125.pdf}}
    \subfigure[LI]{\includegraphics[width=0.300000\linewidth]{figures/RFLI_li-met_v04-v04_rank_RFLI_li-met_v04-v04_pulls_125.pdf}}
    \subfigure[RF]{\includegraphics[width=0.300000\linewidth]{figures/RFLI_rf-sel_v04-v04_rank_RFLI_rf-sel_v04-v04_pulls_125.pdf}}
  \caption{Plots for the top 25 nuisance parameters according to their postfit impact on $\hat{\mu}$ for the standard (a), LI (b), and RF (c) variable sets.}
  \label{fig:npranking}
\end{figure}


\begin{table}[!htbp]
\begin{center}
\begin{tabular}{lccc}
\hline\hline
 &Std-KF &LI+MET &RF\\
\hline
Total &  +0.608 / -0.511  &  +0.632 / -0.539  &  +0.600 / -0.494 \\
DataStat &  +0.420 / -0.401  &  +0.453 / -0.434  &  +0.424 / -0.404 \\
FullSyst &  +0.440 / -0.318  &  +0.441 / -0.319  &  +0.425 / -0.284 \\
Floating normalizations &  +0.122 / -0.125  &  +0.110 / -0.111  &  +0.093 / -0.089 \\
All normalizations &  +0.128 / -0.129  &  +0.112 / -0.112  &  +0.099 / -0.092 \\
All but normalizations &  +0.403 / -0.274  &  +0.387 / -0.250  &  +0.382 / -0.227 \\
\hline
Jets, MET &  +0.180 / -0.097  &  +0.146 / -0.079  &  +0.122 / -0.083 \\
Jets &  +0.051 / -0.030  &  +0.044 / -0.035  &  +0.025 / -0.042 \\
MET &  +0.173 / -0.091  &  +0.140 / -0.074  &  +0.117 / -0.063 \\
BTag &  +0.138 / -0.136  &  +0.069 / -0.071  &  +0.076 / -0.078 \\
BTag b &  +0.125 / -0.125  &  +0.067 / -0.070  &  +0.073 / -0.075 \\
BTag c &  +0.018 / -0.016  &  +0.004 / -0.004  &  +0.005 / -0.005 \\
BTag light &  +0.057 / -0.051  &  +0.020 / -0.014  &  +0.009 / -0.018 \\
Leptons &  +0.013 / -0.012  &  +0.029 / -0.026  &  +0.012 / -0.023 \\
Luminosity &  +0.052 / -0.020  &  +0.050 / -0.016  &  +0.050 / -0.019 \\
Diboson &  +0.043 / -0.039  &  +0.035 / -0.031  &  +0.038 / -0.029 \\
Model Zjets &  +0.119 / -0.117  &  +0.124 / -0.127  &  +0.095 / -0.086 \\
Zjets flt. norm. &  +0.080 / -0.106  &  +0.052 / -0.092  &  +0.026 / -0.072 \\
Model Wjets &  +0.001 / -0.001  &  +0.001 / -0.001  &  +0.000 / -0.001 \\
Wjets flt. norm. &  +0.000 / -0.000  &  +0.000 / -0.000  &  +0.000 / -0.000 \\
Model ttbar &  +0.076 / -0.080  &  +0.025 / -0.035  &  +0.025 / -0.040 \\
Model Single Top &  +0.015 / -0.015  &  +0.002 / -0.004  &  +0.021 / -0.007 \\
Model Multi Jet &  +0.000 / -0.000  &  +0.000 / -0.000  &  +0.000 / -0.000 \\
Signal Systematics &  +0.262 / -0.087  &  +0.272 / -0.082  &  +0.290 / -0.088 \\
MC stat &  +0.149 / -0.136  &  +0.168 / -0.154  &  +0.153 / -0.136 \\
\hline\hline
\end{tabular}
\end{center}
\caption{Expected error breakdowns for the standard, LI, and RF variable sets}
\label{tab:breakdownexp}
\end{table}

\begin{table}[!htbp]
\begin{center}
\begin{tabular}{lccc}
\hline\hline
 &Std-KF &LI+MET &RF\\
\hline
$\hat{\mu}$ & 1.7458 & 1.6467 & 1.5019\\
\hline
Total &  +0.811 / -0.662  &  +0.778 / -0.641  &  +0.731 / -0.612 \\
\hline
DataStat &  +0.502 / -0.484  &  +0.507 / -0.489  &  +0.500 / -0.481 \\
FullSyst &  +0.637 / -0.451  &  +0.591 / -0.415  &  +0.533 / -0.378 \\
Floating normalizations &  +0.153 / -0.143  &  +0.128 / -0.118  &  +0.110 / -0.109 \\
All normalizations &  +0.158 / -0.147  &  +0.130 / -0.119  &  +0.112 / -0.110 \\
All but normalizations &  +0.599 / -0.402  &  +0.544 / -0.354  &  +0.486 / -0.318 \\
\hline
Jets, MET &  +0.218 / -0.145  &  +0.198 / -0.113  &  +0.167 / -0.106 \\
Jets &  +0.071 / -0.059  &  +0.065 / -0.047  &  +0.036 / -0.051 \\
MET &  +0.209 / -0.130  &  +0.190 / -0.102  &  +0.152 / -0.077 \\
BTag &  +0.162 / -0.166  &  +0.093 / -0.070  &  +0.115 / -0.099 \\
BTag b &  +0.142 / -0.147  &  +0.090 / -0.066  &  +0.110 / -0.094 \\
BTag c &  +0.022 / -0.021  &  +0.006 / -0.006  &  +0.007 / -0.007 \\
BTag light &  +0.074 / -0.072  &  +0.025 / -0.022  &  +0.031 / -0.029 \\
Leptons &  +0.039 / -0.029  &  +0.035 / -0.031  &  +0.034 / -0.030 \\
Luminosity &  +0.079 / -0.039  &  +0.073 / -0.034  &  +0.069 / -0.032 \\
Diboson &  +0.047 / -0.043  &  +0.031 / -0.028  &  +0.029 / -0.028 \\
Model Zjets &  +0.164 / -0.152  &  +0.141 / -0.143  &  +0.101 / -0.105 \\
Zjets flt. norm. &  +0.070 / -0.109  &  +0.041 / -0.086  &  +0.033 / -0.083 \\
Model Wjets &  +0.001 / -0.001  &  +0.001 / -0.000  &  +0.001 / -0.001 \\
Wjets flt. norm. &  +0.000 / -0.000  &  +0.000 / -0.000  &  +0.000 / -0.000 \\
Model ttbar &  +0.067 / -0.102  &  +0.029 / -0.040  &  +0.040 / -0.048 \\
Model Single Top &  +0.015 / -0.020  &  +0.001 / -0.005  &  +0.004 / -0.006 \\
Model Multi Jet &  +0.000 / -0.000  &  +0.000 / -0.000  &  +0.000 / -0.000 \\
Signal Systematics &  +0.434 / -0.183  &  +0.418 / -0.190  &  +0.364 / -0.152 \\
MC stat &  +0.226 / -0.201  &  +0.221 / -0.200  &  +0.212 / -0.189 \\
\hline\hline
\end{tabular}
\end{center}
\caption{Observed signal strengths, and error breakdowns for the standard, LI, and RF variable sets}
\label{tab:breakdownobs}
\end{table}
\clearpage
\subsection{S/B Plot}
Plots for the binned S/B in signal region distributions may be found in Figure \ref{fig:SoB}.

\begin{figure}[!htbp]
  \centering
    \subfigure[Standard]{\includegraphics[width=0.300000\linewidth]{figures/RFLI_std-kf_v04.v04_fullRes_RFLI_std-kf_v04.v04/Global_SoverB_2015_details_pulls.pdf}}
    \subfigure[LI]{\includegraphics[width=0.300000\linewidth]{figures/RFLI_li-met_v04.v04_fullRes_RFLI_li-met_v04.v04/Global_SoverB_2015_details_pulls.pdf}}
    \subfigure[RF]{\includegraphics[width=0.300000\linewidth]{figures/RFLI_rf-sel_v04.v04_fullRes_RFLI_rf-sel_v04.v04/Global_SoverB_2015_details_pulls.pdf}}
    \caption{Binned S/B plots for the standard (a), LI (b), and RF (c) variable sets.}
    \label{fig:SoB}
\end{figure}


\subsection{Postfit Distributions}
Postfit distributions for the MVA discriminant ($m_{bb}$) distribution in the signal (top $e-\mu$ control) region for the standard, Lorentz Invariant, and RestFrames variable sets.

\begin{figure}[!htbp]
    \centering
    \subfigure[2 jet, low pTV]{\includegraphics[width=0.45000\linewidth]{figures/RFLI_std-kf_v04.v04_fullRes_RFLI_std-kf_v04.v04/Region_BMax150_BMin75_J2_T2_L2_Y2015_distmva_DSR_GlobalFit_unconditionnal_mu1.pdf}}
    \subfigure[3+ jet, low pTV]{\includegraphics[width=0.45000\linewidth]{figures/RFLI_std-kf_v04.v04_fullRes_RFLI_std-kf_v04.v04/Region_BMax150_BMin75_incJet1_J3_T2_L2_Y2015_distmva_DSR_GlobalFit_unconditionnal_mu1.pdf}}
    \subfigure[2 jet, high pTV]{\includegraphics[width=0.45000\linewidth]{figures/RFLI_std-kf_v04.v04_fullRes_RFLI_std-kf_v04.v04/Region_BMin150_J2_T2_L2_Y2015_distmva_DSR_GlobalFit_unconditionnal_mu1.pdf}}
    \subfigure[3+ jet, high pTV]{\includegraphics[width=0.45000\linewidth]{figures/RFLI_std-kf_v04.v04_fullRes_RFLI_std-kf_v04.v04/Region_BMin150_incJet1_J3_T2_L2_Y2015_distmva_DSR_GlobalFit_unconditionnal_mu1.pdf}}
  \caption{Postfit $BDT_{VH}$ plots in the signal region for the standard variable set.}
  \label{fig:stdPostfitmva}
\end{figure}

\begin{figure}[!htbp]
    \centering
    \subfigure[3+ jet, low pTV]{\includegraphics[width=0.45000\linewidth]{figures/RFLI_std-kf_v04.v04_fullRes_RFLI_std-kf_v04.v04/Region_BMax150_BMin75_incJet1_J3_T2_L2_Y2015_distmBBMVA_Dtopemucr_GlobalFit_unconditionnal_mu1.pdf}}
    \subfigure[2 jet, low pTV]{\includegraphics[width=0.45000\linewidth]{figures/RFLI_std-kf_v04.v04_fullRes_RFLI_std-kf_v04.v04/Region_BMax150_BMin75_J2_T2_L2_Y2015_distmBBMVA_Dtopemucr_GlobalFit_unconditionnal_mu1.pdf}}
    \subfigure[2 jet, high pTV]{\includegraphics[width=0.45000\linewidth]{figures/RFLI_std-kf_v04.v04_fullRes_RFLI_std-kf_v04.v04/Region_BMin150_J2_T2_L2_Y2015_distmBBMVA_Dtopemucr_GlobalFit_unconditionnal_mu1.pdf}}
    \subfigure[3+ jet, high pTV]{\includegraphics[width=0.45000\linewidth]{figures/RFLI_std-kf_v04.v04_fullRes_RFLI_std-kf_v04.v04/Region_BMin150_incJet1_J3_T2_L2_Y2015_distmBBMVA_Dtopemucr_GlobalFit_unconditionnal_mu1.pdf}}
  \caption{Postfit $m_{bb}$ plots in the top $e-\mu$ CR for the standard variable set.}
  \label{fig:stdPostfittopemu}
\end{figure}

\begin{figure}[!htbp]
  \centering
    \subfigure[2 jet, low pTV]{\includegraphics[width=0.45000\linewidth]{figures/RFLI_li-met_v04.v04_fullRes_RFLI_li-met_v04.v04/Region_BMax150_BMin75_J2_T2_L2_Y2015_distmva_DSR_GlobalFit_unconditionnal_mu1.pdf}}
    \subfigure[3+ jet, low pTV]{\includegraphics[width=0.45000\linewidth]{figures/RFLI_li-met_v04.v04_fullRes_RFLI_li-met_v04.v04/Region_BMax150_BMin75_incJet1_J3_T2_L2_Y2015_distmva_DSR_GlobalFit_unconditionnal_mu1.pdf}}
    \subfigure[2 jet, high pTV]{\includegraphics[width=0.45000\linewidth]{figures/RFLI_li-met_v04.v04_fullRes_RFLI_li-met_v04.v04/Region_BMin150_J2_T2_L2_Y2015_distmva_DSR_GlobalFit_unconditionnal_mu1.pdf}}
    \subfigure[3+ jet, high pTV]{\includegraphics[width=0.45000\linewidth]{figures/RFLI_li-met_v04.v04_fullRes_RFLI_li-met_v04.v04/Region_BMin150_incJet1_J3_T2_L2_Y2015_distmva_DSR_GlobalFit_unconditionnal_mu1.pdf}}
  \caption{Postfit $BDT_{VH}$ plots in the signal region for the LI variable set.}
  \label{fig:LIPostfitmva}
\end{figure}

\begin{figure}[!htbp]
    \centering
    \subfigure[3+ jet, low pTV]{\includegraphics[width=0.45000\linewidth]{figures/RFLI_li-met_v04.v04_fullRes_RFLI_li-met_v04.v04/Region_BMax150_BMin75_incJet1_J3_T2_L2_Y2015_distmBBMVA_Dtopemucr_GlobalFit_unconditionnal_mu1.pdf}}
    \subfigure[2 jet, low pTV]{\includegraphics[width=0.45000\linewidth]{figures/RFLI_li-met_v04.v04_fullRes_RFLI_li-met_v04.v04/Region_BMax150_BMin75_J2_T2_L2_Y2015_distmBBMVA_Dtopemucr_GlobalFit_unconditionnal_mu1.pdf}}
    \subfigure[2 jet, high pTV]{\includegraphics[width=0.45000\linewidth]{figures/RFLI_li-met_v04.v04_fullRes_RFLI_li-met_v04.v04/Region_BMin150_J2_T2_L2_Y2015_distmBBMVA_Dtopemucr_GlobalFit_unconditionnal_mu1.pdf}}
    \subfigure[3+ jet, high pTV]{\includegraphics[width=0.45000\linewidth]{figures/RFLI_li-met_v04.v04_fullRes_RFLI_li-met_v04.v04/Region_BMin150_incJet1_J3_T2_L2_Y2015_distmBBMVA_Dtopemucr_GlobalFit_unconditionnal_mu1.pdf}}
  \caption{Postfit $m_{bb}$ plots in the top $e-\mu$ CR for the LI variable set.}
  \label{fig:LIPostfittopemu}
\end{figure}

\begin{figure}[!htbp]
    \centering
    \subfigure[2 jet, low pTV]{\includegraphics[width=0.45000\linewidth]{figures/RFLI_rf-sel_v04.v04_fullRes_RFLI_rf-sel_v04.v04/Region_BMax150_BMin75_J2_T2_L2_Y2015_distmva_DSR_GlobalFit_unconditionnal_mu1.pdf}}
    \subfigure[3+ jet, low pTV]{\includegraphics[width=0.45000\linewidth]{figures/RFLI_rf-sel_v04.v04_fullRes_RFLI_rf-sel_v04.v04/Region_BMax150_BMin75_incJet1_J3_T2_L2_Y2015_distmva_DSR_GlobalFit_unconditionnal_mu1.pdf}}
    \subfigure[2 jet, high pTV]{\includegraphics[width=0.45000\linewidth]{figures/RFLI_rf-sel_v04.v04_fullRes_RFLI_rf-sel_v04.v04/Region_BMin150_J2_T2_L2_Y2015_distmva_DSR_GlobalFit_unconditionnal_mu1.pdf}}
    \subfigure[3+ jet, high pTV]{\includegraphics[width=0.45000\linewidth]{figures/RFLI_rf-sel_v04.v04_fullRes_RFLI_rf-sel_v04.v04/Region_BMin150_incJet1_J3_T2_L2_Y2015_distmva_DSR_GlobalFit_unconditionnal_mu1.pdf}}
  \caption{Postfit $BDT_{VH}$ plots in the signal region for the RF variable set.}
  \label{fig:RFPostfitmva}
\end{figure}

\begin{figure}[!htbp]
    \centering
    \subfigure[3+ jet, low pTV]{\includegraphics[width=0.45000\linewidth]{figures/RFLI_rf-sel_v04.v04_fullRes_RFLI_rf-sel_v04.v04/Region_BMax150_BMin75_incJet1_J3_T2_L2_Y2015_distmBBMVA_Dtopemucr_GlobalFit_unconditionnal_mu1.pdf}}
    \subfigure[2 jet, low pTV]{\includegraphics[width=0.45000\linewidth]{figures/RFLI_rf-sel_v04.v04_fullRes_RFLI_rf-sel_v04.v04/Region_BMax150_BMin75_J2_T2_L2_Y2015_distmBBMVA_Dtopemucr_GlobalFit_unconditionnal_mu1.pdf}}
    \subfigure[2 jet, high pTV]{\includegraphics[width=0.45000\linewidth]{figures/RFLI_rf-sel_v04.v04_fullRes_RFLI_rf-sel_v04.v04/Region_BMin150_J2_T2_L2_Y2015_distmBBMVA_Dtopemucr_GlobalFit_unconditionnal_mu1.pdf}}
    \subfigure[3+ jet, high pTV]{\includegraphics[width=0.45000\linewidth]{figures/RFLI_rf-sel_v04.v04_fullRes_RFLI_rf-sel_v04.v04/Region_BMin150_incJet1_J3_T2_L2_Y2015_distmBBMVA_Dtopemucr_GlobalFit_unconditionnal_mu1.pdf}}
  \caption{Postfit $m_{bb}$ plots in the top $e-\mu$ CR for the RF variable set.}
  \label{fig:RFPostfittopemu}
\end{figure}

\subsection{Nuisance Parameter Pulls}
As can be seen in Figures \ref{fig:PullComparisons-allExceptGammas}--\ref{fig:PullComparisons-Zjets}, the fits for the three different variable sets are fairly similar from a NP pull perspective.  Black is the standard variable set, red is the LI set, and blue is the RF set.

\begin{figure}
    \subfigure[Asimov]{\includegraphics[width=0.490000\linewidth]{./figures/pullcomp-asi-prod/NP_allExceptGammas.pdf}}
    \subfigure[Observed]{\includegraphics[width=0.490000\linewidth]{./figures/pullcomp-obs-prod/NP_allExceptGammas.pdf}}
  \caption{Pull comparison for all NP's but MC stats.}
  \label{fig:PullComparisons-allExceptGammas}
\end{figure}

\begin{figure}
    \subfigure[Asimov]{\includegraphics[width=0.490000\linewidth]{./figures/pullcomp-asi-prod/NP_Jet.pdf}}
    \subfigure[Observed]{\includegraphics[width=0.490000\linewidth]{./figures/pullcomp-obs-prod/NP_Jet.pdf}}
  \caption{Pull comparison for jet NP's.}
  \label{fig:PullComparisons-Jet}
\end{figure}

\begin{figure}
    \subfigure[Asimov]{\includegraphics[width=0.490000\linewidth]{./figures/pullcomp-asi-prod/NP_MET.pdf}}
    \subfigure[Observed]{\includegraphics[width=0.490000\linewidth]{./figures/pullcomp-obs-prod/NP_MET.pdf}}
  \caption{Pull comparison for MET NP's.}
  \label{fig:PullComparisons-MET}
\end{figure}

\begin{figure}
    \subfigure[Asimov]{\includegraphics[width=0.490000\linewidth]{./figures/pullcomp-asi-prod/NP_BTag.pdf}}
    \subfigure[Observed]{\includegraphics[width=0.490000\linewidth]{./figures/pullcomp-obs-prod/NP_BTag.pdf}}
  \caption{Pull comparison for Flavour Tagging NP's.}
  \label{fig:PullComparisons-BTag}
\end{figure}

\begin{figure}
    \subfigure[Asimov]{\includegraphics[width=0.490000\linewidth]{./figures/pullcomp-asi-prod/NP_Zjets.pdf}}
    \subfigure[Observed]{\includegraphics[width=0.490000\linewidth]{./figures/pullcomp-obs-prod/NP_Zjets.pdf}}
  \caption{Pull comparison for $Z+$jets NP's.}
  \label{fig:PullComparisons-Zjets}
\end{figure}

\clearpage
\subsection{Nuisance Parameter Correlations}
Nuisance parameter correlation matrices (for correlations with magnitude at least 0.25) for all three variable set fits can be found in Figure \ref{fig:AsimovHighCorrelationsprod} for Asimov fits and Figure \ref{fig:ObservedHighCorrelationsprod} for observed fits.

\begin{figure}[!htbp]
    \subfigure[Asimov]{\includegraphics[width=0.49000\linewidth]{./figures/RFLI_std-kf_v04.v04_fullRes_RFLI_std-kf_v04.v04/Asimov/corr_HighCorr.pdf}}
    \subfigure[Observed]{\includegraphics[width=0.49000\linewidth]{./figures/RFLI_std-kf_v04.v04_fullRes_RFLI_std-kf_v04.v04/Global/corr_HighCorr.pdf}}
  \caption{NP correlations for standard variable fits.}
  \label{fig:corrstd-kf}
\end{figure}


\begin{figure}[!htbp]
    \subfigure[Asimov]{\includegraphics[width=0.49000\linewidth]{./figures/RFLI_li-met_v04.v04_fullRes_RFLI_li-met_v04.v04/Asimov/corr_HighCorr.pdf}}
    \subfigure[Observed]{\includegraphics[width=0.49000\linewidth]{./figures/RFLI_li-met_v04.v04_fullRes_RFLI_li-met_v04.v04/Global/corr_HighCorr.pdf}}
  \caption{NP correlations for standard variable fits.}
  \label{fig:corrli-met}
\end{figure}


\begin{figure}[!htbp]
    \subfigure[Asimov]{\includegraphics[width=0.49000\linewidth]{./figures/RFLI_rf-sel_v04.v04_fullRes_RFLI_rf-sel_v04.v04/Asimov/corr_HighCorr.pdf}}
    \subfigure[Observed]{\includegraphics[width=0.49000\linewidth]{./figures/RFLI_rf-sel_v04.v04_fullRes_RFLI_rf-sel_v04.v04/Global/corr_HighCorr.pdf}}
  \caption{NP correlations for standard variable fits.}
  \label{fig:corrrf-sel}
\end{figure}






%\section{data-MC comparisons including background scale factors}
%\label{app:prefitplots_scaled}
%\input{AppPrefitPlots_scaled.tex}

\clearpage

%-------------------------------------------------------------------------------
% If you use biblatex and either biber or bibtex to process the bibliography
% just say \printbibliography here
\printbibliography
% If you want to use the traditional BibTeX you need to use the syntax below.
%\bibliographystyle{bibtex/bst/atlasBibStyleWoTitle}
%\bibliography{SMVh_supporting,bibtex/bib/ATLAS}
%-------------------------------------------------------------------------------

%-------------------------------------------------------------------------------
% Print the list of contributors to the analysis
% The argument gives the fraction of the text width used for the names
%-------------------------------------------------------------------------------
%%-------------------------------------------------------------------------------
%\clearpage
%\appendix
%\part*{Auxiliary material}
%\addcontentsline{toc}{part}{Auxiliary material}
%%-------------------------------------------------------------------------------
%
%In an ATLAS paper, auxiliary plots and tables that are supposed to be made public 
%should be collected in an appendix that has the title \enquote{Auxiliary material}.
%This appendix should be printed after the Bibliography.
%At the end of the paper approval procedure, this information can be split into a separate document
%-- see \texttt{atlas-auxmat.tex}.
%
%In an ATLAS note, use the appendices to include all the technical details of your work
%that are relevant for the ATLAS Collaboration only (e.g.\ dataset details, software release used).
%This information should be printed after the Bibliography.

\end{document}

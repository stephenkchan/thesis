%!TEX root = ../dissertation.tex
\begin{savequote}[75mm]
If I have seen further, it is by standing on ye shoulders of giants.
\qauthor{Isaac Newton}
\end{savequote}

\chapter{Measurement Combinations}
\label{ch:comb}
\newthought{While the discussion} thus far has focused on improvements looking towards future in just the \ZH\, channel, any actual result for SM \vhbb\, combines all channels and all available datasets.  Using additional channels at a given center of mass energy is straightforward since the fit model is designed with this combination in mind.  Combining dataset results (known as ``workspaces'') from different center of mass energies is not so simple an exercise since both the underlying physics (and its associated modeling) and the treatment of key experimental considerations, like flavor tagging, and their associated systematics change from dataset to dataset.  A combined fit model must take these considerations into account, and the formulation of the fit model combining the Run 1 ($\sqrt{s}=$7 \TeV with 4.7 fb$^{-1}$ of data, and $\sqrt{s}=$8 \TeV with 20.3 fb$^{-1}$ of data) and Run 2 ($\sqrt{s}=$13 \TeV with \lumi) SM \vhbb\, results is the topic of Section \ref{sec:combmodel}.  Its results, as reported in \cite{paper}, are given in \ref{sec:combres}.

\section{The Combined Fit Model}
\label{sec:combmodel}
It is clear the signal strength parameter of interest should be fully correlated among the different datasets.  Some signal modeling systematics were left unchanged from Run 1 through Run 2 and/or were designed to be explicitly correlated.  Beyond these two special cases, it is not immediately clear what level of correlation should be imposed.  The general methodology for settling upon a correlation scheme is as follows:

\begin{enumerate}
\item Identify which NP categories have significant impacts on $\mu$
\item Of these NP's, identify which have one-to-one correspondences or established correlation schemes among $\sqrt{s}$ values
\item Test whether correlation has a sizeable impact on expected fit quantities
\end{enumerate}

The only two sizeable experimental NP categories are jet energy scale (JES) and flavor tagging systematics.  Correlation schemes of varying degrees of completeness exist for these categories, so explicit NP correlations can be tested for these two categories.  As these studies were conducted before unblinding, ``sizeable impact'' was judged by comparing fit results (sensitivities, pull comparisons, and breakdowns) on combined workspaces using the unblinded and public $\mu=0.51$ result for Run 1 and Asimov data for the Run 2 result.  These are treated in Sections \ref{sec:jes} and \ref{sec:ft}.  Modeling systematics require a slightly different treatment, and are explored in \ref{sec:model}.

As noted in Chapter \ref{ch:fit} when looking at pull comparison plots for combined workspaces, the error bars in these plots are calculated using a simultaneous \texttt{HESSE} matrix inversion, which can fail to give sensible values for high dimensional models (the combined workspaces have well over 500 NP's).  This is not true of the nuisance parameter ranking plots, which use a \texttt{MINOS} based approach to test the effect of each NP individually.  This is much slower but much more rigorous, which is why only ranking plots appear outside of supporting material and pull comparisons are considered ``diagnostic'' plots.

\subsection{Jet Energy Scale Systematics}
\label{sec:jes}
%twiki for recommendations can be found at https://twiki.cern.ch/twiki/bin/viewauth/AtlasProtected/JESCorrelationRecommendations2015; not sure how to cite
Fortunately for the case of jet energy scale systematics, the JetEtMiss group provides two recommended ``strong'' and ``weak'' correlation schemes between Run 1 and Run 2.  These were used as a point of departure for the JES combination correlation scheme.  However, the JES NP's in both the Run 1 and Run 2 workspaces are a reduced set of NP's, with some 56 (75) NP's reduced to 6 (8) for Run 1 (2).  In order to restore the full set of JES NP's, the effective NP's in each workspace are unfolded using maps detailing the linear combinations of unfolded NP's that form the effective NP's.  


%twiki for recommendations can be found at https://twiki.cern.ch/twiki/bin/viewauth/AtlasProtected/JESCorrelationRecommendations2015; not sure how to cite
The linear combinations used to unfold the effective JES NP's were calculated as follows:
\begin{equation}
NP_{i,eff} = \frac{\sum_j A_{ij} \left|NP_{j,unf}\right| NP_{j,unf}}{\sqrt{\sum_j A_{ij}^2 \left|NP_{j,unf}\right|^2 }}
\end{equation}
where \textit{eff} and \textit{unf} are for effective and unfolded NP's, respectively, the $A_{ij}$'s are scalar coeffecients taken from raw maps, and $\left|NP_{j,unf}\right|$ are the amplitudes of the unfolded NP's.  The raw $A_{ij}$ and scaled maps for Run 1 and Run 2 may be found in Figure~\ref{fig:comb_jes_maps}


\begin{figure}[!htbp]\captionsetup{justification=centering}
  \centering
  \begin{tabular}{cccc}
\begin{subfigure}[t]{0.24\textwidth}\centering\includegraphics[width=\textwidth]{figures/comb/run1_maps/Coeffs_2DHisto_original}\caption{}\end{subfigure}
\begin{subfigure}[t]{0.24\textwidth}\centering\includegraphics[width=\textwidth]{figures/comb/run1_maps/Coeffs_2DHisto_scaled}\caption{}\end{subfigure}
\begin{subfigure}[t]{0.24\textwidth}\centering\includegraphics[width=\textwidth]{figures/comb/run2_maps/Coeffs_2DHisto_original}\caption{}\end{subfigure}
\begin{subfigure}[t]{0.24\textwidth}\centering\includegraphics[width=\textwidth]{figures/comb/run2_maps/Coeffs_2DHisto_scaled}\caption{}\end{subfigure}
    \end{tabular}
    \caption{The raw and scaled coefficients for unfolding Run 1 (a and b) and Run 2 (c and d), respectively}
    \label{fig:comb_jes_maps}
\end{figure}


Unfolding was found to have very little effect on both expected sensitivites and errors, as can be seen in Tables~\ref{tab:unfold_jes_sensitivities}--~\ref{tab:unfold_jes_breakdownsc}.
\begin{table}[!htbp]\captionsetup{justification=centering}
\begin{center}\begin{tabular}{lcccccc}
\hline\hline
 & R1 Unfold & R1 Eff & R2 Unfold & R2 Eff & Comb Unfold & Comb Eff\\
\hline
Exp. Sig. &  2.604 & 2.606 & 3.014 & 3.014 & 4.005 & 3.998\\
Obs. Sig. & 1.369 & 1.374 & 3.53 & 3.53 & 3.581 & 3.571\\
\hline
Exp. Limit &  $0.755^{+0.296}_{-0.211}$ & $0.755^{+0.296}_{-0.211}$ & $0.732^{+0.287}_{-0.205}$ & $0.732^{+0.287}_{-0.205}$ & $0.512^{+0.201}_{-0.143}$ & $0.51^{+0.2}_{-0.143}$\\
Obs. Limit & 1.21 & 1.21 & 1.94 & 1.94 & 1.36 & 1.37\\
\hline\hline
\end{tabular}
\caption{Expected and observed sensitivities for Run 1, Run 2, and combined workspaces with effective and unfolded JES NP's.}
\label{tab:unfold_jes_sensitivities}
\end{center}
\end{table}

\begin{table}[!htbp]\captionsetup{justification=centering}
\begin{center}\begin{tabular}{lcc}
\hline\hline
 & R1 Unfold & R1 Eff \\
\hline
$\left|\Delta\hat{\mu}\right|$ &  \multicolumn{2}{c}{0.0018} \\
$\hat{\mu}$ & 0.5064 & 0.5082 \\
\hline
Total &  +0.400 / -0.373  &  +0.401 / -0.373  \\
DataStat &  +0.312 / -0.301  &  +0.312 / -0.301\\
FullSyst &  +0.250 / -0.220  &  +0.251 / -0.220\\
\hline
Jets &  +0.060 / -0.051  &  +0.060 / -0.052  \\
BTag &  +0.094 / -0.079  &  +0.095 / -0.079  \\
\hline
\hline
\end{tabular}
\caption{Error on signal strength breakdowns for  Run 1 workspaces with effective and unfolded JES NP's.}
\label{tab:unfold_jes_breakdowns1}
\end{center}
\end{table}


\begin{table}[!htbp]\captionsetup{justification=centering}
\begin{center}\begin{tabular}{lcc}
\hline\hline
 & R2 Unfold & R2 Eff\\
\hline
$\left|\Delta\hat{\mu}\right|$ &  \multicolumn{2}{c}{0.0}\\
$\hat{\mu}$ & 1.2051 & 1.2052\\
\hline
Total & +0.421 / -0.366  &  +0.421 / -0.366\\
DataStat &  +0.239 / -0.234  &  +0.239 / -0.234\\
FullSyst &  +0.346 / -0.282  &  +0.346 / -0.282\\
\hline
Jets &  +0.066 / -0.047  &  +0.066 / -0.047\\
BTag &  +0.119 / -0.106  &  +0.119 / -0.106\\
\hline
\hline
\end{tabular}
\caption{Error on signal strength breakdowns for Run 2 workspaces with effective and unfolded JES NP's.}
\label{tab:unfold_jes_breakdowns2}
\end{center}
\end{table}


\begin{table}[!htbp]\captionsetup{justification=centering}
\begin{center}\begin{tabular}{lcc}
\hline\hline
 & Comb Unfold & Comb Eff\\
\hline
$\left|\Delta\hat{\mu}\right|$ &  \multicolumn{2}{c}{0.0006}\\
$\hat{\mu}$ & 0.8992 & 0.8985\\
\hline
Total &  +0.278 / -0.261  &  +0.278 / -0.261 \\
DataStat &  +0.185 / -0.181  &  +0.185 / -0.181 \\
FullSyst &  +0.208 / -0.187  &  +0.208 / -0.188 \\
\hline
Jets &  +0.040 / -0.044  &  +0.041 / -0.036 \\
BTag &  +0.076 / -0.076  &  +0.077 / -0.076 \\
\hline
\hline
\end{tabular}
\caption{Error on signal strength breakdowns for combined workspaces with effective and unfolded JES NP's.}
\label{tab:unfold_jes_breakdownsc}
\end{center}
\end{table}

It was also found that fit sensitivities and breakdowns were similarly indifferent to the use of either the strong or weak JES correlation schemes, as shown in Tables~\ref{tab:sw_jes_sensitivities} and~\ref{tab:sw_jes_breakdowns}.

\begin{table}[!htbp]\captionsetup{justification=centering}
\begin{center}\begin{tabular}{lcccc}
\hline\hline
 & JES Weak Unfold & JES Weak Eff & JES Strong Unfold & JES Strong Eff\\
\hline
Exp. Sig. & 3.57 & 3.57 & 3.59 & 3.59\\
\hline
Exp. Limit & $0.493^{+0.193}_{-0.138}$ & $0.494^{+0.193}_{-0.138}$ & $0.493^{+0.193}_{-0.138}$ & $0.493^{+0.193}_{-0.138}$\\
\hline\hline
\end{tabular}
\caption{Expected sensitivities for both effective and unfolded combined workspaces using the strong and weak JES correlation schemes.}
\label{tab:sw_jes_sensitivities}
\end{center}
\end{table}

\begin{table}[!htbp]\captionsetup{justification=centering}
\begin{center}\begin{tabular}{lcccc}
\hline
 & Comb Unfold & Comb Eff & Strong Unfold & Strong Eff\\
\hline\hline
$\Delta\hat{\mu}$ &  \multicolumn{2}{|c|}{0.0009} &  \multicolumn{2}{|c|}{0.0025}\\
\hline
Total &  $^{+ 0.269}_{-0.254}$ &  $^{+ 0.27}_{-0.255}$ &  $^{+ 0.27}_{-0.255}$ &  $^{+ 0.27}_{-0.255}$\\
\hline
DataStat &  $^{+ 0.181}_{-0.177}$ &  $^{+ 0.181}_{-0.177}$ &  $^{+ 0.181}_{-0.177}$ &  $^{+ 0.181}_{-0.178}$\\
\hline
FullSyst &  $^{+ 0.199}_{-0.183}$ &  $^{+ 0.2}_{-0.183}$ &  $^{+ 0.2}_{-0.183}$ &  $^{+ 0.201}_{-0.183}$\\
\hline
Jets &  $^{+ 0.0387}_{-0.032}$ &  $^{+ 0.041}_{-0.0337}$ &  $^{+ 0.0425}_{-0.0329}$ &  $^{+ 0.0432}_{-0.0338}$\\
\hline
BTag &  $^{+ 0.0975}_{-0.0933}$ &  $^{+ 0.098}_{-0.0936}$ &  $^{+ 0.0979}_{-0.0935}$ &  $^{+ 0.098}_{-0.0936}$\\
\hline\hline
\end{tabular}
\caption{Error on signal strength breakdowns for both effective and unfolded combined workspaces using the strong and weak JES correlation schemes.}

\label{tab:sw_jes_breakdowns}\end{center}
\end{table}

Comparisons of top ranked nuisance parameters in Figures \ref{fig:Ranks:jesu1}--\ref{fig:Ranks:jesuc} and for the complete JES pull comparisons in Figures \ref{fig:PullComparisons:jesu---Jet}--\ref{fig:PullComparisons:jesu---JetUnfold} also show very little difference with respect to correlation scheme (except obviously for the number of JES NP's).  Constrained pulls in pull comparisons should once again be taken as a shortcoming of \texttt{HESSE} and not the fit model.
\begin{figure}[!htbp]\captionsetup{justification=centering}
  \centering
\begin{subfigure}[t]{0.45000\textwidth}\centering\includegraphics[width=\textwidth]{figures/comb/x_combData-jesu1_rank_job100of100_Run1Run2Comb_x_combData-jesu1_pulls_125}\caption{R1 Unfold}\end{subfigure}
\begin{subfigure}[t]{0.45000\textwidth}\centering\includegraphics[width=\textwidth]{figures/comb/r1_combData-613r1_rank_job100of100_Run1Run2Comb_r1_combData-613r1_pulls_125}\caption{R1 Eff}\end{subfigure}
  \caption{Ranks for the effective and unfolded JES NP Run1 combined workspaces.}
    \label{fig:Ranks:jesu1}
\end{figure}

\begin{figure}[!htbp]\captionsetup{justification=centering}
  \centering
\begin{subfigure}[t]{0.45000\textwidth}\centering\includegraphics[width=\textwidth]{figures/comb/test_obsData-jesu2_rank_Run1Run2Comb_test_obsData-jesu2_pulls_125}\caption{R1 Unfold}\end{subfigure}
\begin{subfigure}[t]{0.45000\textwidth}\centering\includegraphics[width=\textwidth]{figures/comb/r2_obsData-907r2_rank_Run1Run2Comb_r2_obsData-907r2_pulls_125}\caption{R1 Eff}\end{subfigure}
  \caption{Ranks for the effective and unfolded JES NP Run2 combined workspaces.}
    \label{fig:Ranks:jesu2}
\end{figure}

\begin{figure}[!htbp]\captionsetup{justification=centering}
  \centering
\begin{subfigure}[t]{0.45\textwidth}\centering\includegraphics[width=\textwidth]{figures/comb/r1_combData-r2_obsData-jesu_partial-eps02_rank_job100of100_Run1Run2Comb_r1_combData-r2_obsData-jesu_partial-eps02_pulls_125}\caption{Comb Unfolded}\end{subfigure}
\begin{subfigure}[t]{0.45\textwidth}\centering\includegraphics[width=\textwidth]{figures/comb/r1_combData-r2_obsData-jes_partial-eps02_rank_job100of100_Run1Run2Comb_r1_combData-r2_obsData-jes_partial-eps02_pulls_125}\caption{Comb Effective}\end{subfigure}
  \caption{Ranks for the effective and unfolded JES NP Run1+Run2 combined workspaces.}
    \label{fig:Ranks:jesuc}
\end{figure}

\begin{figure}[!htbp]\captionsetup{justification=centering}
\centering
\includegraphics[width=\textwidth]{figures/comb/pullcomp-jesu/NP_Jet}
  \caption{Pull Comparisons: jesu---Jet  \textcolor{black}{Comb Unfold}, \textcolor{red}{Comb Eff}, \textcolor{blue}{Strong Unfold}, \textcolor{magenta}{Strong Eff}}
  \label{fig:PullComparisons:jesu---Jet}  
\end{figure}

\begin{figure}[!htbp]\captionsetup{justification=centering}
\centering
\includegraphics[width=\textwidth]{figures/comb/pullcomp-jesu/NP_JetMatched}
  \caption{Pull Comparisons: jesu---JetMatched  \textcolor{black}{Comb Unfold}, \textcolor{red}{Comb Eff}, \textcolor{blue}{Strong Unfold}, \textcolor{magenta}{Strong Eff}}
  \label{fig:PullComparisons:jesu---JetMatched}
\end{figure}

\begin{figure}[!htbp]\captionsetup{justification=centering}
\centering
\includegraphics[width=\textwidth]{figures/comb/pullcomp-jesu/NP_JetEff}
  \caption{Pull Comparisons: jesu---JetEff  \textcolor{black}{Comb Unfold}, \textcolor{red}{Comb Eff}, \textcolor{blue}{Strong Unfold}, \textcolor{magenta}{Strong Eff}}
  \label{fig:PullComparisons:jesu---JetEff}
\end{figure}

\begin{figure}[!htbp]\captionsetup{justification=centering}
\centering
\includegraphics[width=\textwidth]{figures/comb/pullcomp-jesu/NP_JetUnfold}
  \caption{Pull Comparisons: jesu---JetUnfold  \textcolor{black}{Comb Unfold}, \textcolor{red}{Comb Eff}, \textcolor{blue}{Strong Unfold}, \textcolor{magenta}{Strong Eff}}
  \label{fig:PullComparisons:jesu---JetUnfold}
\end{figure}

As a result of these studies, the weak JES correlation scheme with uncorrelated effective JES NP's (i.e. just the $b$-jet energy scale NP) has been chosen as the treatment of JES in the Run 1 + Run 2 combined fit.

\clearpage
\subsection{Flavor Tagging}
\label{sec:ft}
Unfortunately, the ATLAS Flavor Tagging group did not provide any recommendations for correlating Run 1 and Run 2 NP's, though given the high ranking of these NP's in the Run 2, result, performing at least some studies was deemed crucial.  Nevertheless, great improvements and changes to the treatment of flavor tagging between Run 1 and Run 2 does weaken the argument for any strong flavor tagging correlation scheme.

Given that $c$-tagging changed significantly between Run 1 and Run 2 and that light tagging NP's are very lowly ranked, these sets of NP's are left uncorrelated.  Moreover, the change in the physical meaning of the effective $b$-tagging NP's means a full correlation of such NP's (insomuch as they exist in each result) is one of limited utility.  Hence, it was decided to leave flavor tagging NP's uncorrelated.  However, since the meaning of the leading $b$-tagging NP's is approximately constant across years and since Run 2 $b$-tagging NP's are very highly ranked in both the Run 2 only and combined fits, tests correlating these NP's were conducted, the results of which can be seen below.  It should be noted that the leading B NP at 8 \TeV, \texttt{SysBTagB0Effic\_Y2012\_8TeV}, has an opposite effect on $t\bar{t}$ normalization than the 7 and 13 TeV NP's, and so must be flipped using a similar strategy as for JES unfolding.  Initial studies of flavor tagging correlations did not flip this NP, and so results for this scheme (labeled ``B0 8TeV Not Flipped'') have also been included for comparison.

\begin{table}[htbp]
\begin{center}\begin{tabular}{lccc}
\hline\hline
 & Comb Eff & BTag B0 & B0 8TeV Not Flipped\\
\hline
Exp. Sig. & 3.998 & 4.127 & 3.921\\
Obs. Sig. &  3.571 & 3.859 & 3.418\\ %3.572 & 3.866 & 3.4\\
\hline
Exp. Limit & $0.51^{+0.2}_{-0.143}$ & $0.5^{+0.196}_{-0.14}$ & $0.517^{+0.202}_{-0.144}$\\
Obs. Limit & 1.37 & 1.41 & 1.35\\
\hline
\end{tabular}
\caption{Expected and observed sensitivities for a combination featuring the weak JES scheme, combination with the weak JES scheme + leading $b$ NP's correlated, and the $b$ correlation with the 8 TeV NP with sign unflipped.}
\label{tab:ExpectedSensitivities:btag-b}
\end{center}
\end{table}

\begin{table}[!htbp]\captionsetup{justification=centering}
\begin{center}\begin{tabular}{lccc}
\hline\hline
 &Comb Eff &BTag B0 &B0 8TeV Not Flipped\\
$\left|\Delta\hat{\mu}\right|$ & --- & 0.0446 & 0.0268\\
$\hat{\mu}$ & 0.8985 & 0.9431 & 0.8717\\
\hline
Total &  +0.278 / -0.261  &  +0.275 / -0.256  &  +0.282 / -0.263 \\
DataStat &  +0.185 / -0.181  &  +0.180 / -0.177  &  +0.189 / -0.186 \\
FullSyst &  +0.208 / -0.188  &  +0.207 / -0.186  &  +0.209 / -0.186 \\
\hline
BTag &  +0.077 / -0.076  &  +0.071 / -0.068  &  +0.079 / -0.075 \\
BTag b &  +0.062 / -0.059  &  +0.055 / -0.049  &  +0.064 / -0.060 \\
\hline
\hline
\end{tabular}
\caption{Breakdowns of the impact of different NP sets on total error on $\hat{mu}$ for a combination featuring the weak JES scheme and a combination with the weak JES scheme + leading $b$ NP's correlated.}
\label{tab:Breakdowns:btag-b}
\end{center}\end{table}


\begin{figure}[!htbp]\captionsetup{justification=centering}
\centering\includegraphics[width=\textwidth]{figures/comb/pullcomp-btag-b/NP_BTagB0}
  \caption{Pull Comparisons: btag-b---BTagB0  \textcolor{black}{Comb Eff}, \textcolor{red}{BTag B0}}
  \label{fig:PullComparisons:btag-b---BTagB0}
\end{figure}

\begin{figure}[!htbp]\captionsetup{justification=centering}
\centering\includegraphics[width=\textwidth]{figures/comb/pullcomp-btag-b/NP_BTagB}
  \caption{Pull Comparisons: btag-b---BTagB  \textcolor{black}{Comb Eff}, \textcolor{red}{BTag B0}}
  \label{fig:PullComparisons:btag-b---BTagB}
\end{figure}

\begin{figure}[!htbp]\captionsetup{justification=centering}
  \centering
\begin{subfigure}[t]{0.32\linewidth}\centering\includegraphics[width=\textwidth]{figures/comb/r1_combData-r2_obsData-jes_partial-eps02_rank_job100of100_Run1Run2Comb_r1_combData-r2_obsData-jes_partial-eps02_pulls_125}\caption{Comb Eff}\end{subfigure}
\begin{subfigure}[t]{0.32\linewidth}\centering\includegraphics[width=\textwidth]{figures/comb/r1_combData-r2_obsData-btag-b_partial-eps02_rank_job100of100_Run1Run2Comb_r1_combData-r2_obsData-btag-b_partial-eps02_pulls_125}\caption{BTag B0}\end{subfigure}
\begin{subfigure}[t]{0.32\linewidth}\centering\includegraphics[width=\textwidth]{figures/comb/r1_combData-r2_obsData-btag-nf_partial-eps02_rank_job100of100_Run1Run2Comb_r1_combData-r2_obsData-btag-nf_partial-eps02_pulls_125}\caption{B0 8 TeV Not Flipped}\end{subfigure}
  \caption{NP rankings for a combination featuring the weak JES scheme and a combination with the weak JES scheme + leading $b$ NP's correlated.}
    \label{fig:Ranks:btag-b}
\end{figure}

It is clear from these results that correlating the leading effective Eigen NP associated with $b$'s can have a noticeable effect on final fit results and that the 8 \TeV B0 NP is the most important component of a combined B0 NP.  It is also not so surprising that the 8 \tev\, result should drive the combined nuisance parameter since it is the only result to make use of both pseudocontinuous tagging-based and 1 $b$-tag regions into the final fit, implicitly yielding much more information about $b$'s.  The 13 \tev\, fit has neither of these regions.  What is less clear is whether there are sufficient grounds for implementing this correlation (i.e. does the correspondence of these NP's across years warrant a full correlation).  While there are no current plans to do so, this matter warrants careful scrutiny if Run 1 is to be combined with future results.

\subsection{Modeling Systematics}
\label{sec:model}
Another principal systematic category is modeling uncertainties.  The effect of correlating groups of systematics was estimated using the same strategy employed by the ATLAS/CMS SM \vhbb\, combination for Run 1.  This extrapolation can be used to estimate the impact of correlations on the estimated signal strength, the total error on the signal strength, and the $\chi^2$ of the result.  The impact of such correlations is no more than a few percent effect, as the following tables demonstrate, beginning with the category with the greatest shift, W+jets modeling, in Table~\ref{tab:wjets_correlation}.

\begin{table}[!htbp]\captionsetup{justification=centering}
\begin{center}\begin{tabular}{lcccc}
\hline\hline
 & $\left|\Delta\mu\right|$ & $\sigma$ & $\left|\Delta\sigma\right|$ & $\chi^2$\\
\hline
$\rho$=-1 & 0.0024 & 0.2448 & 0.011 (4.3\%) & 0.95\\
$\rho$=-0.6 & 0.0015 & 0.2493 & 0.00654 (2.55\%) & 0.9804\\
$\rho$=-0.3 & 0.0008 & 0.2526 & 0.00325 (1.27\%) & 1.0045\\
$\rho$=0 & --- & 0.2558 & --- & 1.0298\\
$\rho$=0.3 & 0.0008 & 0.259 & 0.0032 (1.25\%) & 1.0564\\
$\rho$=0.6 & 0.0017 & 0.2622 & 0.00636 (2.49\%) & 1.0844\\
$\rho$=1 & 0.0029 & 0.2664 & 0.0105 (4.11\%) & 1.1242\\
\hline\hline
\end{tabular}
\caption{Run 1 + Run 2 W+jets modeling correlation projections}
\label{tab:wjets_correlation}
\end{center}
\end{table}

\begin{comment}

\label{tab:run1+run2DibosonCorrelationProjections}
\begin{center}
\begin{tabular}{lcccc}
\hline\hline
 &$\left|\Delta\mu\right|$ &$\sigma$ &$\left|\Delta\sigma\right|$ &$\chi^2$\\
\hline
$\rho$=-1 & 0.0 & 0.2617 & 0.000141 (0.0539\%) & 1.8116\\
\hline
$\rho$=-0.6 & 0.0 & 0.2618 & 8.4e-05 (0.0321\%) & 1.8124\\
\hline
$\rho$=-0.3 & 0.0 & 0.2618 & 4.2e-05 (0.016\%) & 1.813\\
\hline
$\rho$=0 & --- & 0.2618 & --- & 1.8136\\
\hline
$\rho$=0.3 & 0.0 & 0.2619 & 4.3e-05 (0.0164\%) & 1.8141\\
\hline
$\rho$=0.6 & 0.0 & 0.2619 & 8.5e-05 (0.0325\%) & 1.8147\\
\hline
$\rho$=1 & 0.0 & 0.262 & 0.000142 (0.0542\%) & 1.8155\\
\hline
\end{tabular}
\caption{run1+run2 Diboson Correlation Projections}
\end{center}\end{table}

\begin{table}[htbp]
\begin{center}
\label{tab:run1+run2ModelZjetsCorrelationProjections}
\begin{tabular}{lcccc}
\hline\hline
 &$\left|\Delta\mu\right|$ &$\sigma$ &$\left|\Delta\sigma\right|$ &$\chi^2$\\
\hline
$\rho$=-1 & 0.0001 & 0.2567 & 0.00515 (1.97\%) & 1.7456\\
\hline
$\rho$=-0.6 & 0.0001 & 0.2588 & 0.00308 (1.18\%) & 1.7721\\
\hline
$\rho$=-0.3 & 0.0 & 0.2603 & 0.00153 (0.586\%) & 1.7926\\
\hline
$\rho$=0 & --- & 0.2618 & --- & 1.8136\\
\hline
$\rho$=0.3 & 0.0 & 0.2634 & 0.00153 (0.583\%) & 1.835\\
\hline
$\rho$=0.6 & 0.0001 & 0.2649 & 0.00304 (1.16\%) & 1.857\\
\hline
$\rho$=1 & 0.0001 & 0.2669 & 0.00505 (1.93\%) & 1.8871\\
\hline
\end{tabular}\caption{run1+run2 Model Zjets Correlation Projections}
\end{center}\end{table}

\begin{table}[htbp]
\begin{center}
\label{tab:run1+run2Zjetsflt.norm.CorrelationProjections}
\begin{tabular}{lcccc}
\hline\hline
 &$\left|\Delta\mu\right|$ &$\sigma$ &$\left|\Delta\sigma\right|$ &$\chi^2$\\
\hline
$\rho$=-1 & 0.0 & 0.2601 & 0.00173 (0.662\%) & 1.7899\\
\hline
$\rho$=-0.6 & 0.0 & 0.2608 & 0.00104 (0.396\%) & 1.7993\\
\hline
$\rho$=-0.3 & 0.0 & 0.2613 & 0.000518 (0.198\%) & 1.8064\\
\hline
$\rho$=0 & --- & 0.2618 & --- & 1.8136\\
\hline
$\rho$=0.3 & 0.0 & 0.2624 & 0.000518 (0.198\%) & 1.8208\\
\hline
$\rho$=0.6 & 0.0 & 0.2629 & 0.00104 (0.395\%) & 1.828\\
\hline
$\rho$=1 & 0.0 & 0.2636 & 0.00172 (0.658\%) & 1.8378\\
\hline
\end{tabular}\caption{run1+run2 Zjets flt. norm. Correlation Projections}
\end{center}\end{table}

\begin{table}[htbp]
\begin{center}
\label{tab:run1+run2ModelWjetsCorrelationProjections}
\begin{tabular}{lcccc}
\hline\hline
 &$\left|\Delta\mu\right|$ &$\sigma$ &$\left|\Delta\sigma\right|$ &$\chi^2$\\
\hline
$\rho$=-1 & 0.0003 & 0.2495 & 0.0124 (4.72\%) & 1.6605\\
\hline
$\rho$=-0.6 & 0.0002 & 0.2545 & 0.00734 (2.8\%) & 1.7186\\
\hline
$\rho$=-0.3 & 0.0001 & 0.2582 & 0.00364 (1.39\%) & 1.7648\\
\hline
$\rho$=0 & --- & 0.2618 & --- & 1.8136\\
\hline
$\rho$=0.3 & 0.0001 & 0.2654 & 0.0036 (1.37\%) & 1.8651\\
\hline
$\rho$=0.6 & 0.0002 & 0.269 & 0.00714 (2.73\%) & 1.9197\\
\hline
$\rho$=1 & 0.0003 & 0.2736 & 0.0118 (4.51\%) & 1.9976\\
\hline
\end{tabular}\caption{run1+run2 Model Wjets Correlation Projections}
\end{center}\end{table}

\begin{table}[htbp]
\begin{center}
\label{tab:run1+run2Wjetsflt.norm.CorrelationProjections}
\begin{tabular}{lcccc}
\hline\hline
 &$\left|\Delta\mu\right|$ &$\sigma$ &$\left|\Delta\sigma\right|$ &$\chi^2$\\
\hline
$\rho$=-1 & 0.0 & 0.2603 & 0.0015 (0.572\%) & 1.7931\\
\hline
$\rho$=-0.6 & 0.0 & 0.2609 & 0.000898 (0.343\%) & 1.8012\\
\hline
$\rho$=-0.3 & 0.0 & 0.2614 & 0.000448 (0.171\%) & 1.8074\\
\hline
$\rho$=0 & --- & 0.2618 & --- & 1.8136\\
\hline
$\rho$=0.3 & 0.0 & 0.2623 & 0.000449 (0.171\%) & 1.8198\\
\hline
$\rho$=0.6 & 0.0 & 0.2627 & 0.000896 (0.342\%) & 1.8261\\
\hline
$\rho$=1 & 0.0 & 0.2633 & 0.00149 (0.569\%) & 1.8345\\
\hline
\end{tabular}\caption{run1+run2 Wjets flt. norm. Correlation Projections}
\end{center}\end{table}

\begin{table}[htbp]
\begin{center}
\label{tab:run1+run2ModelttbarCorrelationProjections}
\begin{tabular}{lcccc}
\hline\hline
 &$\left|\Delta\mu\right|$ &$\sigma$ &$\left|\Delta\sigma\right|$ &$\chi^2$\\
\hline
$\rho$=-1 & 0.0001 & 0.2586 & 0.00323 (1.24\%) & 1.7701\\
\hline
$\rho$=-0.6 & 0.0001 & 0.2599 & 0.00193 (0.739\%) & 1.7872\\
\hline
$\rho$=-0.3 & 0.0 & 0.2609 & 0.000966 (0.369\%) & 1.8003\\
\hline
$\rho$=0 & --- & 0.2618 & --- & 1.8136\\
\hline
$\rho$=0.3 & 0.0 & 0.2628 & 0.000963 (0.368\%) & 1.827\\
\hline
$\rho$=0.6 & 0.0001 & 0.2638 & 0.00192 (0.734\%) & 1.8407\\
\hline
$\rho$=1 & 0.0001 & 0.265 & 0.0032 (1.22\%) & 1.8592\\
\hline
\end{tabular}\caption{run1+run2 Model ttbar Correlation Projections}
\end{center}\end{table}

\begin{table}[htbp]
\begin{center}
\label{tab:run1+run2ttbarflt.norm.CorrelationProjections}
\begin{tabular}{lcccc}
\hline\hline
 &$\left|\Delta\mu\right|$ &$\sigma$ &$\left|\Delta\sigma\right|$ &$\chi^2$\\
\hline
$\rho$=-1 & 0.0 & 0.2611 & 0.000694 (0.265\%) & 1.804\\
\hline
$\rho$=-0.6 & 0.0 & 0.2614 & 0.000416 (0.159\%) & 1.8078\\
\hline
$\rho$=-0.3 & 0.0 & 0.2616 & 0.000208 (0.0794\%) & 1.8107\\
\hline
$\rho$=0 & --- & 0.2618 & --- & 1.8136\\
\hline
$\rho$=0.3 & 0.0 & 0.262 & 0.000208 (0.0794\%) & 1.8164\\
\hline
$\rho$=0.6 & 0.0 & 0.2623 & 0.000416 (0.159\%) & 1.8194\\
\hline
$\rho$=1 & 0.0 & 0.2625 & 0.000693 (0.265\%) & 1.8232\\
\hline
\end{tabular}\caption{run1+run2 ttbar flt. norm. Correlation Projections}
\end{center}\end{table}

\begin{table}[htbp]
\begin{center}
\label{tab:run1+run2ModelSingleTopCorrelationProjections}
\begin{tabular}{lcccc}
\hline\hline
 &$\left|\Delta\mu\right|$ &$\sigma$ &$\left|\Delta\sigma\right|$ &$\chi^2$\\
\hline
$\rho$=-1 & 0.0001 & 0.2599 & 0.00198 (0.756\%) & 1.7866\\
\hline
$\rho$=-0.6 & 0.0 & 0.2607 & 0.00119 (0.453\%) & 1.7973\\
\hline
$\rho$=-0.3 & 0.0 & 0.2612 & 0.000592 (0.226\%) & 1.8054\\
\hline
$\rho$=0 & --- & 0.2618 & --- & 1.8136\\
\hline
$\rho$=0.3 & 0.0 & 0.2624 & 0.000592 (0.226\%) & 1.8218\\
\hline
$\rho$=0.6 & 0.0 & 0.263 & 0.00118 (0.451\%) & 1.8301\\
\hline
$\rho$=1 & 0.0001 & 0.2638 & 0.00197 (0.751\%) & 1.8413\\
\hline
\end{tabular}\caption{run1+run2 Model Single Top Correlation Projections}
\end{center}\end{table}

\begin{table}[htbp]
\begin{center}
\label{tab:run1+run2SignalSystematicsCorrelationProjections}
\begin{tabular}{lcccc}
\hline\hline
 &$\left|\Delta\mu\right|$ &$\sigma$ &$\left|\Delta\sigma\right|$ &$\chi^2$\\
\hline
$\rho$=-1 & 0.0001 & 0.2565 & 0.00531 (2.03\%) & 1.7436\\
\hline
$\rho$=-0.6 & 0.0001 & 0.2587 & 0.00317 (1.21\%) & 1.7709\\
\hline
$\rho$=-0.3 & 0.0 & 0.2603 & 0.00158 (0.604\%) & 1.792\\
\hline
$\rho$=0 & --- & 0.2618 & --- & 1.8136\\
\hline
$\rho$=0.3 & 0.0 & 0.2634 & 0.00157 (0.6\%) & 1.8357\\
\hline
$\rho$=0.6 & 0.0001 & 0.265 & 0.00313 (1.2\%) & 1.8583\\
\hline
$\rho$=1 & 0.0001 & 0.267 & 0.0052 (1.99\%) & 1.8894\\
\hline
\end{tabular}\caption{run1+run2 Signal Systematics Correlation Projections}
\end{center}\end{table}
\end{comment}


\subsection{Final Correlation Scheme}
The final Run 1 + Run 2 correlation scheme is shown in Table~\ref{tableComb1}.  As detailed above, neither JES nor modeling systematics had any demonstrable effect on combined fit results.  Hence, only signal NP's and the $b$-jet energy scale are correlated (the weak JES scheme without unfolding).  While the effect of flavor tagging correlations is less clear, the result physical arguments for correlation are less strong; the size of effect was discovered rather late in the analysis process; and  no nuisance parameter unfolding maps exist for flavor tagging as they do for JES, so it was decided to leave these uncorrelated as well.

\begin{table}[!htbp]\captionsetup{justification=centering}
\center\small
\begin{tabular}{lll} \hline\hline
 \textbf{7 TeV NP} & \textbf{8 TeV NP} &  \textbf{13 TeV NP}\\ \hline
\multicolumn{2}{c}{ATLAS\_BR\_bb} & SysTheoryBRbb\\
\multicolumn{2}{c}{SysTheoryQCDscale\_ggZH} & SysTheoryQCDscale\_ggZH\\
\multicolumn{2}{c}{SysTheoryQCDscale\_qqVH} & SysTheoryQCDscale\_qqVH\\
--- & SysTheoryPDF\_ggZH\_8TeV & SysTheoryPDF\_ggZH\\
--- & SysTheoryPDF\_qqVH\_8TeV & SysTheoryPDF\_qqVH\\
--- & SysTheoryVHPt\_8TeV & SysVHNLOEWK\\
SysJetFlavB\_ 7TeV & SysJetFlavB\_ 8TeV & SysJET\_21NP\_JET\_BJES\_Response\\
 \hline\hline
\end{tabular}
\caption{A summary of correlated nuisance parameters among the 7, 8, and 13 TeV datasets.}
\label{tableComb1}
\end{table}

\section{Combined Fit Results}
\label{sec:combres}
\subsection{Combined Fit Model Validation}
Before moving onto the final results, we present the rest of the validaitons for the Run 1 + Run 2 combined fits, beginning with impacts of ranked individual nuisance parameters in Figure \ref{fig:combranks} and for all nuisance parameter categories in Table \ref{tab:combbreakdown}.  Both of these sets of results point to the most important nuisance parameters being signal systematics, $b$-tagging, and $V$+jets modeling systematics, with Run 2 NP's generally being higher ranked.    That some NP's are strongly pulled is not unusual as the fit model has so many NP's; $V$+jets modeling in particular has been historically difficult.

\begin{figure}[!htbp]\captionsetup{justification=centering}
  \centering
  \includegraphics[width=0.55\linewidth]{figures/comb/r1_combData-r2_obsData-jes_partial-eps02_rank_job100of100_Run1Run2Comb_r1_combData-r2_obsData-jes_partial-eps02_pulls_125}
  \caption{Ranked nuisance parameters for the Run1+Run2 combination.}
  \label{fig:combranks}
\end{figure}


\begin{table}[!htbp]\captionsetup{justification=centering}
\begin{center}\begin{tabular}{|l|c|}
\hline
Total & +0.278 / -0.261 \\\hline
DataStat & +0.185 / -0.181 \\
FullSyst & +0.208 / -0.188 \\\hline
Floating normalizations & +0.055 / -0.056 \\
All normalizations & +0.068 / -0.069 \\
All but normalizations & +0.192 / -0.172 \\
Jets, MET & +0.046 / -0.040 \\
Jets & +0.041 / -0.036 \\
MET & +0.023 / -0.018 \\
BTag & +0.077 / -0.076 \\
BTag b & +0.062 / -0.059 \\
BTag c & +0.033 / -0.032 \\
BTag light & +0.028 / -0.028 \\
Leptons & +0.008 / -0.008 \\
Luminosity & +0.026 / -0.014 \\
Diboson &  +0.030 / -0.027 \\
Model Zjets & +0.049 / -0.050 \\
Zjets flt. norm. & +0.032 / -0.040 \\
Model Wjets & +0.082 / -0.083 \\
Wjets flt. norm. & +0.031 / -0.027 \\
Model ttbar & +0.047 / -0.046 \\
ttbar flt. norm. & +0.025 / -0.026 \\
Model Single Top & +0.047 / -0.045 \\
Model Multi Jet & +0.027 / -0.038 \\
Signal Systematics & +0.098 / -0.052 \\
MC stat & +0.080 / -0.084 \\
\hline
\end{tabular}
\caption{Summary of the impact of different nuisance parameter categories on the total error on $\hat{\mu}$ for the combined Run1+Run2 fit.}
\label{tab:combbreakdown}
\end{center}
\end{table}

In addition to looking at the behaviors of nuisance parameters to gauge fit model performance and stability, fits are conducted using multiple parameters of interest.  Typical divisions are Run 1 vs. Run 2, lepton channels, and $WH$ vs $ZH$.  As mentioned in Chapter \ref{ch:fit}, the profile likelihood test statistic given in Equation \ref{eqn:teststat} is, in the limit of large sample statistics, a $\chi^2$ distribution with degrees of freedom equal to the number of parameters of interest plus number of nuisance parameters.  Thus, changing the number of interest parameters and leaving the rest of the fit model unchanged means that the difference between the nominal fit and a fit with more parameters of interest ought to also be distributed as a $\chi^2$ distribution with degrees of freedom equivalent to the number of extra parameters of interest.  This difference can then be interpreted as a compatibility between the two results using the standard tables for this distribution, giving another gauge of fit performance.  These are shown in Table \ref{tab:compatability}.

\begin{table}[!htbp]\captionsetup{justification=centering}
\begin{center}\begin{tabular}{|l|c|}
\hline
Fit & Compatability \\\hline
Leptons (3 POI) & 1.49\% \\
$WH/ZH$ (2 POI) & 34.2\% \\
Run 1/Run 2 (2 POI) & 20.8\% \\
Run 1/Run 2 $\times$ Leptons (6 POI) & 7.10\% \\
Run 1/Run 2 $\times WH/ZH$  (4 POI) & 34.6\% \\
\hline
\end{tabular}
\caption{Summary of multiple POI compatabilities.  The well-known Run 1 7 \tev\, 0-lepton deficit is responsible for the low compatability with the 6 and 3 POI fits.}
\label{tab:compatability}
\end{center}
\end{table}

The low compatabilities associated with treating the lepton channels as separate parameters of interest are a sympton of the low signal strenghts associated with the Run 1 0-lepton channel, in particular the 7 \tev\,result.  Given the relatively small amount of data associated with the 7 \tev\, result, this should not be a cause for alarm.  Signal strength summary plots for the fits treating Run 1 and Run 2 separately are shown in Figures \ref{fig:combmushat4}-\ref{fig:combmushat6}, where the effect of the Run 1 parameters can be seen graphically.

\begin{figure}[!htbp]\captionsetup{justification=centering}
  \begin{center}
  \includegraphics[width=0.75\linewidth]{figures/comb/Plot_mu_eps02_r1r2whzh}
  \caption{$\hat{\mu}$ summary plot for a four parameter of interest fit.}
  \label{fig:combmushat4}
  \end{center}
\end{figure}

\begin{figure}[!htbp]\captionsetup{justification=centering}
  \begin{center}
  \includegraphics[width=0.75\linewidth]{figures/comb/Plot_mu_eps02_r1r2channels}
  \caption{$\hat{\mu}$ summary plot for a six parameter of interest fit.}
  \label{fig:combmushat5}
  \end{center}
\end{figure}

\begin{figure}[!htbp]\captionsetup{justification=centering}
  \begin{center}
  \includegraphics[width=0.75\linewidth]{figures/comb/Plot_mu_eps02_r1r2}
  \caption{$\hat{\mu}$ summary plot for a two parameter of interest (Run 1 and Run 2) values.}
  \label{fig:combmushat6}
  \end{center}
\end{figure}

\subsection{Final Results}
The combined results yields an observed (expected) significance of 3.57 (4.00) and an observed (expected) limit of 1.37 ($0.510^{+0.200}_{-0.143}$), with a signal strength of $\hat{\mu}=0.898^{+0.278}_{-0.261}$.

The two and three parameter of interest fit signal strength summary plots, as well as a summary of the historical values of the 7, 8, and 13 \TeV\,results may be found in Figures~\ref{fig:combmushat1}-\ref{fig:combmushat3}.  The main results for Run 1, Run 2, and the combination may be found in Table~\ref{tableCombresykts}.  These results were collectively noted as the first ever experiemental evidence for SM \vhbb\, in \cite{paper}.

\begin{table}[!htbp]\captionsetup{justification=centering}
\begin{center}
\begin{tabular}{lccc} \hline\hline
Dataset & $\hat{\mu}$ & Total Error in $\hat{\mu}$ & Obs. (Exp.) Significance \\
\hline
Run 1 & 0.51 & +0.40 / -0.37 & 1.4 (2.6)\\
Run 2 & 1.20 & +0.42 / -0.36 & 3.54 (3.03)\\
Combined & 0.90 & +0.28 / -0.26 & 3.57 (4.00)\\
\hline\hline
\end{tabular}
\caption{A summary of main results for the Run 1, Run 2, and combined fits.}
\label{tableCombresykts}
\end{center}
\end{table}


\begin{figure}[!htbp]\captionsetup{justification=centering}
  \begin{center}
  \includegraphics[width=0.75\linewidth]{figures/comb/Plot_mu_eps02_WHZH}
  \caption{$\hat{\mu}$ summary plot for a two parameter of interest fit.}
  \label{fig:combmushat1}
  \end{center}
\end{figure}
\begin{figure}[!htbp]\captionsetup{justification=centering}
  \begin{center}
  \includegraphics[width=0.75\linewidth]{figures/comb/Plot_mu_eps02_channels}
  \caption{$\hat{\mu}$ summary plot for a three parameter of interest fit.}
  \label{fig:combmushat2}
  \end{center}
\end{figure}
\begin{figure}[!htbp]\captionsetup{justification=centering}
  \begin{center}
  \includegraphics[width=0.75\linewidth]{figures/comb/Plot_mu_eps02_7813}
  \caption{$\hat{\mu}$ summary plot for different $\sqrt{s}$ values.}
  \label{fig:combmushat3}
  \end{center}
\end{figure}


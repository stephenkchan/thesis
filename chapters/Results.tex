%!TEX root = ../dissertation.tex
\begin{savequote}[75mm]
Kein Operationsplan reicht mit einiger Sicherheit \"uber das erste Zusammentreffen mit der feindlichen Hauptmacht hinaus.
\qauthor{Helmuth von Moltke}
\end{savequote}

\chapter{Fit Results}
\label{ch:results}
\newthought{The results in this chapter} were first reported in \cite{rflinote} and describe how the three different fit models detailed and validated Chapter \ref{ch:fit}, corresponding to the standard, RF, and LI variable sets described in Chapter \ref{ch:mva} perform on actual $VH$ fits.  In particular sensitivities, nuisance parameter impacts, and signal strengths on expected fits to Asimov datasets and both expected and observed fits on the actual \lumi dataset are compared.

  Expected and observed sensitivities for the different variable sets may be found in Table \ref{tab:Sensitivities}.  The RF fits feature the highest expected sensitivities, outperforming the standard set by 3.5\% and 3.4\% for fits to Asimov and observed datasets, respectively.  The LI variable has a lower significance than both for expected fits to both Asimov and data with a 6.7\% (1.7\%) significance than the standard set for the Asimov (observed) dataset.  While the fit using standard variables does have a higher observed significance than both the LI and RF fits, by 2.8\% and 8.6\%, respectively, these numbers should be viewed in the context of the best fit $\hat{\mu}$ values, discussed below.  That is, the standard set may yield the highest sensitivity for this particular dataset, but this is not necessarily (and likely is not) the case for any given dataset.

\begin{table}[!htbp]\captionsetup{justification=centering}
\caption{Expected (for both data and Asimov) and observed significances for the standard, LI, and RF variable sets.}
\begin{center}
\begin{tabular}{lccc}
\hline\hline
 & Standard &LI &RF\\
\hline
Expected (Asimov) & 2.06 & 1.92 & 2.13\\
\hline
Expected (data) & 1.76 & 1.73 & 1.80\\
\hline
Observed (data) & 2.87 & 2.79 & 2.62\\
\hline
\hline
\end{tabular}
\end{center}
\label{tab:Sensitivities}
\end{table}

A summary of fitted signal strengths and errors for both the Asimov (a) and observed (b) datasets are shown in Figure \ref{fig:MuhatSummary}.\footnote{For reference, the standalone 2-lepton fit from the fiducial analysis is $2.11^{+0.50}_{-0.48}(\textrm{stat.})^{+0.64}_{-0.47}(\textrm{syst.})$}  A summary of error breakdowns is given in Tables \ref{tab:bd-asi-summary} (Asimov) and \ref{tab:bd-obs-summary} (observed) for total error, data statistics contributions, total systematic error contributions, and categories for which the total impact is $\ge0.1$ for the standard fit.  As is to be expected for both the Asimov and observed dataset fits, the contribution to the total error on $\mu$ arising from data statistics is nearly identical, since each set of fits uses the same selections and data.\footnote{Though not exactly identical.  Since the BDT's are different for the different variable sets, the binning (as determined by transformation D) and bin contents in each set are generally different, leading to slightly different data statistics errors.}

\begin{table}[!htbp]\captionsetup{justification=centering}
\caption{Summary of error impacts on total $\mu$ error for principal categories in the Asimov standard, LI, and RF fits.}
\begin{center}
\begin{tabular}{lccc}
\hline\hline
 &Std-KF &LI+MET &RF\\
\hline
Total &  +0.608 / -0.511  &  +0.632 / -0.539  &  +0.600 / -0.494 \\
\hline
DataStat &  +0.420 / -0.401  &  +0.453 / -0.434  &  +0.424 / -0.404 \\
\hline
FullSyst &  +0.440 / -0.318  &  +0.441 / -0.319  &  +0.425 / -0.284 \\
\hline
Signal Systematics &  +0.262 / -0.087  &  +0.272 / -0.082  &  +0.290 / -0.088 \\
\hline
MET &  +0.173 / -0.091  &  +0.140 / -0.074  &  +0.117 / -0.063 \\
\hline
Flavor Tagging &  +0.138 / -0.136  &  +0.069 / -0.071  &  +0.076 / -0.078 \\
\hline
Model Zjets &  +0.119 / -0.117  &  +0.124 / -0.127  &  +0.095 / -0.086 \\
\hline
\end{tabular}
\label{tab:bd-asi-summary}
\end{center}
\end{table}

\begin{table}[!htbp]\captionsetup{justification=centering}
\caption{Summary of error impacts on total $\hat{\mu}$ error for principal categories in the observed standard, LI, and RF fits.}
\begin{center}
\begin{tabular}{lccc}
\hline\hline
 &Std-KF &LI+MET &RF\\
\hline
Total &  +0.811 / -0.662  &  +0.778 / -0.641  &  +0.731 / -0.612 \\
\hline
DataStat &  +0.502 / -0.484  &  +0.507 / -0.489  &  +0.500 / -0.481 \\
\hline
FullSyst &  +0.637 / -0.451  &  +0.591 / -0.415  &  +0.533 / -0.378 \\
\hline
Signal Systematics &  +0.434 / -0.183  &  +0.418 / -0.190  &  +0.364 / -0.152 \\
\hline
MET &  +0.209 / -0.130  &  +0.190 / -0.102  &  +0.152 / -0.077 \\
\hline
Flavor Tagging &  +0.162 / -0.166  &  +0.093 / -0.070  &  +0.115 / -0.099 \\
\hline
Model Zjets &  +0.164 / -0.152  &  +0.141 / -0.143  &  +0.101 / -0.105 \\
\hline
\end{tabular}
\label{tab:bd-obs-summary}
\end{center}
\end{table}


The contribution from systematic uncertainties, however, does vary considerably across the variable sets.  The Asimov fits are a best case scenario in the sense that, by construction, all NP's are equal to their predicted values (and so no ``penalty'' is paid for pulls on Gaussian NP's).  The systematics error from the LI fit is slightly higher (subpercent) than that from the standard fit, and 4.6\% higher error overall due to differences in data stats.  The RF Asimov fit, however, has a 6.5\% lower total error from systematics than the standard Asimov fit (and a 2.2\% lower error overall).  Moreover, for both the LI and RF sets, errors are markedly smaller for the MET and Flavor Tagging categories, with the RF fit also featuring a smaller errors on $Z+$jets modeling; the only notable exception to this trend in Asimov fits are the signal systematics.

These trends are more pronounced in the observed fits.  As can be seen in Table \ref{tab:bd-obs-summary}, both the LI and RF fits have smaller errors from systematic uncertainties, both overall and in all principal categories, with the LI and RF fits having 7.5\% (3.7\%) and 16\% (8.8\%) lower systematics (total) error on $\hat{\mu}$, respectively.

\begin{figure}[!htbp]\captionsetup{justification=centering}
  \centering
  \begin{subfigure}[t]{0.49\textwidth}\centering\includegraphics[width=\textwidth]{figures/Plot_mu_rfli-prod-exp}\caption{Asimov}\end{subfigure}
  \begin{subfigure}[t]{0.49\textwidth}\centering\includegraphics[width=\textwidth]{figures/Plot_mu_rfli-prod-obs}\caption{Observed}\end{subfigure}
  \caption{$\mu$ summary plots for the standard, LI, and RF variable sets.  The Asimov case (with $\mu=1$ by construction) is in (a), and $\hat{\mu}$ best fit values and error summary are in (b).}
  \label{fig:MuhatSummary}
\end{figure}

Studying the performance of the Lorentz Invariant and RestFrames variable sets at both a data statistics only context and with the full fit model in the \ZH\, channel of the $VH\left(b\bar{b}\right)$ analysis suggests that these variables may offer a potential method for better constraining systematic uncertainties in $VH\left(b\bar{b}\right)$ searches as more orthogonal bases in describing the information in collision events.  

The marginally worse performance of the LI and RF variables (7.9\% and 6.9\%, respectively) with respect to the standard variable at a stats only level illustrates that neither variable set has greater intrinsic descriptive power in the absence of systematics in this closed final state.  Hence, any gains from either of these variable sets in a full fit come from improved treatment of systematic uncertainties.

With full systematics, the LI variable set narrows the sensitivity gap somewhat, with lower significances by 6.7\% (1.7\%) on expected fits to Asimov (data) and by 2.8\% on observed significances.  The RF variable set outperforms the standard set in expected fits with 3.5\% (3.4\%) higher significance on Asimov (data), but has an 8.6\% lower observed significance, though the observed significances should be viewed in the context of observed $\hat{\mu}$ values.

Moreover, the LI and RF variable sets generally perform better in the context of the error on $\mu$.  The LI fit is comparable to the standard set on Asimov data and has a 7.5\% lower total systematics error on $\hat{\mu}$ on observed data, while the RF fit is lower in both cases, with systematics error being 6.5\% (16\%) lower on Asimov (observed) data.  


\begin{comment}
A summary of performance metrics in this document may be found in Table \ref{tab:kahuna}.  


\begin{table}[!htbp]\captionsetup{justification=centering}
\caption{Summary of performance figures for the standard, LI, and RF variable sets.  In the case of the latter two, \% differences are given where relevant.  Differences in errors on $\mu$ are on full systematics and total error, respectively.}
\begin{center}
\begin{tabular}{lccc}
\hline\hline
 & Standard &LI &RF\\
\hline
$\hat{\mu}$ & $1.75^{+0.24}_{-0.23}(\textrm{stat.})^{+0.34}_{-0.28}(\textrm{syst.})$ & $1.65^{+0.24}_{-0.23}(\textrm{stat.})^{+0.34}_{-0.28}(\textrm{syst.})$ & $1.50^{+0.24}_{-0.23}(\textrm{stat.})^{+0.34}_{-0.28}(\textrm{syst.})$\\
Asi. $\Delta err\left(\mu\right)$ &  --- & $<1$\%, +4.6\% & -6.5\%, -2.2\%\\
Obs. $\Delta err\left(\hat{\mu}\right)$ &  --- & -7.5\%, -3.7\% & -16\%, -8.8\%\\
\hline
Stat only sig. & 4.78 & 4.39 (-7.9\%) & 4.44 (-6.9\%)\\
Exp. (Asi.) sig. & 2.06 & 1.92 (-6.7\%) & 2.13 (+3.5\%)\\
Exp. (data) sig. & 1.76 & 1.73 (-1.7\%) & 1.80 (+3.4\%)\\
Obs. (data) sig. & 2.87 & 2.79 (-2.8\%) & 2.62 (-8.6\%)\\
\hline\hline
\end{tabular}
\end{center}
\label{tab:kahuna}
\end{table}
\end{comment}

These figures of merit suggest that both the LI and RF variables are more orthogonal than the standard variable set used in the fiducial analysis.  Moreover, the RF variable set does seem to consistently perform better than the LI set.  Furthermore, both variable sets have straightforward extensions to the other lepton channels in the \vhbb\, analysis.  The magnitude of any gain from the more sophisticated treatment of $E_T^{miss}$ in these extensions is beyond the scope of these studies, but the performance in this closed final state do suggest that there is some value to be had in these non-standard descriptions independent of these considerations.

%A search for the decay of a Standard Model Higgs boson into $b\bar{b}$ pair when produced in association with a $W$ and $Z$ boson has been performed with Run1 and Run2 full data. In Run2 dataset, the measured signal strength with respect to the SM expectation is found to be $1.20^{+0.24}_{-0.23}(\textrm{stat.})^{+0.34}_{-0.28}(\textrm{syst.})$ at $m_{H}=125$ GeV. The observed (expected) significance is 3.5(3.0) standard deviations. The analysis has been validated by measuring the signal strength of $V(W/Z)Z(\rightarrow b\bar{b})$. The measured signal strength is $1.11^{+0.12}_{-0.11}(\textrm{stat.})^{+0.22}_{-0.19}(\textrm{syst.})$, corresponding to observed(expected) significance of 5.8(5.3) standard deviations.  

%This result based on Run2 data has been also combined with previous results on the Run1 dataset. The observed(expected) significance combined Run1 and Run2 is 3.6(4.0) standard deviation. The measured signal strength is $0.90^{+0.18}_{-0.18}(\textrm{stat.})^{+0.21}_{-0.19}(\textrm{syst.})$. 


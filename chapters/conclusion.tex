%!TEX root = ../dissertation.tex
\begin{savequote}[75mm]
Vanitas vanitatum, omnis vanitas
\qauthor{Ecclesiastes 1:2}
\end{savequote}

\chapter{Conclusions}

\newthought{Much has been said} 
Studying the performance of the Lorentz Invariant and RestFrames variable sets at both a data statistics only context and with the full fit model in the \ZH\, channel of the $VH\left(b\bar{b}\right)$ analysis suggests that these variables may offer a potential method for better constraining systematic uncertainties in $VH\left(b\bar{b}\right)$ searches as more orthogonal bases in describing the information in collision events.  

The marginally worse performance of the LI and RF variables (7.9\% and 6.9\%, respectively) with respect to the standard variable at a stats only level illustrates that neither variable set has greater intrinsic descriptive power in the absence of systematics in this closed final state.  Hence, any gains from either of these variable sets in a full fit come from improved treatment of systematic uncertainties.

With full systematics, the LI variable set narrows the sensitivity gap somewhat, with lower significances by 6.7\% (1.7\%) on expected fits to Asimov (data) and by 2.8\% on observed significances.  The RF variable set outperforms the standard set in expected fits with 3.5\% (3.4\%) higher significance on Asimov (data), but has an 8.6\% lower observed significance, though the observed significances should be viewed in the context of observed $\hat{\mu}$ values.

Moreover, the LI and RF variable sets generally perform better in the context of the error on $\mu$.  The LI fit is comparable to the standard set on Asimov data and has a 7.5\% lower total systematics error on $\hat{\mu}$ on observed data, while the RF fit is lower in both cases, with systematics error being 6.5\% (16\%) lower on Asimov (observed) data.  A summary of performance metrics in this document may be found in Table \ref{tab:kahuna}.  


\begin{table}[!htbp]
\begin{center}
\begin{tabular}{lccc}
\hline\hline
 & Standard &LI &RF\\
\hline
$\hat{\mu}$ & $1.75^{+0.24}_{-0.23}(\textrm{stat.})^{+0.34}_{-0.28}(\textrm{syst.})$ & $1.65^{+0.24}_{-0.23}(\textrm{stat.})^{+0.34}_{-0.28}(\textrm{syst.})$ & $1.50^{+0.24}_{-0.23}(\textrm{stat.})^{+0.34}_{-0.28}(\textrm{syst.})$\\
Asi. $\Delta err\left(\mu\right)$ &  --- & $<1$\%, +4.6\% & -6.5\%, -2.2\%\\
Obs. $\Delta err\left(\hat{\mu}\right)$ &  --- & -7.5\%, -3.7\% & -16\%, -8.8\%\\
\hline
Stat only sig. & 4.78 & 4.39 (-7.9\%) & 4.44 (-6.9\%)\\
Exp. (Asi.) sig. & 2.06 & 1.92 (-6.7\%) & 2.13 (+3.5\%)\\
Exp. (data) sig. & 1.76 & 1.73 (-1.7\%) & 1.80 (+3.4\%)\\
Obs. (data) sig. & 2.87 & 2.79 (-2.8\%) & 2.62 (-8.6\%)\\
\hline\hline
\end{tabular}
\end{center}
\caption{Summary of performance figures for the standard, LI, and RF variable sets.  In the case of the latter two, \% differences are given where relevant.  Differences in errors on $\mu$ are on full systematics and total error, respectively.}
\label{tab:kahuna}
\end{table}

These figures of merit suggest that both the LI and RF variables are more orthogonal than the standard variable set used in the fiducial analysis.  Moreover, the RF variable set does seem to consistently perform better than the LI set.  Furthermore, both variable sets have straightforward extensions to the one lepton channel in the $VH\left(b\bar{b}\right)$ analysis, and the RF set has a straightforward extension to the zero lepton channel as well.  The magnitude of any gain from the more sophistocated treatment of $E_T^{miss}$ in these extensions is beyond the scope of these studies, but the performance in this closed final state do suggest that there is some value to be had in these non-standard descriptions independent of these considerations.

%A search for the decay of a Standard Model Higgs boson into $b\bar{b}$ pair when produced in association with a $W$ and $Z$ boson has been performed with Run1 and Run2 full data. In Run2 dataset, the measured signal strength with respect to the SM expectation is found to be $1.20^{+0.24}_{-0.23}(\textrm{stat.})^{+0.34}_{-0.28}(\textrm{syst.})$ at $m_{H}=125$ GeV. The observed (expected) significance is 3.5(3.0) standard deviations. The analysis has been validated by measuring the signal strength of $V(W/Z)Z(\rightarrow b\bar{b})$. The measured signal strength is $1.11^{+0.12}_{-0.11}(\textrm{stat.})^{+0.22}_{-0.19}(\textrm{syst.})$, corresponding to observed(expected) significance of 5.8(5.3) standard deviations.  

%This result based on Run2 data has been also combined with previous results on the Run1 dataset. The observed(expected) significance combined Run1 and Run2 is 3.6(4.0) standard deviation. The measured signal strength is $0.90^{+0.18}_{-0.18}(\textrm{stat.})^{+0.21}_{-0.19}(\textrm{syst.})$. 


%!TEX root = ../dissertation.tex
\begin{savequote}[75mm]
If it's stupid but it works, it isn't stupid.
\qauthor{Conventional Wisdom}
\end{savequote}

\chapter{Multivariate Analysis Configuration}
\label{ch:mva}
\newthought{In order to} fully leverage the descriptive power of the 13 \tev dataset, this analysis makes use of a multivariate (MVA) discriminant.  Where traditionally event counts or single discriminating variables per region of phase space have been fed to fits, MVA discriminants seek to integrate additional information not captured in the conventional phase space cuts plus dijet invariant mass distributions.  Formulating the MVA discriminant is an exercise in supervised learning to construct a binary classifier, where one uses labeled ``signal'' and ``background'' MC events to optimize the parameters of a statistical model---in this case a boosted decision tree (BDT) with some set of physically motivated variables (or ``factors'').  The interested reader is directed to the standard references on machine learning for further details.  Sample and variable selection, including variables derived using the the RestFrames and Lorentz Invariants concepts introduced in Sections \ref{sec:li}--\ref{sec:rfli01le}, are discussed in Section \ref{sec:samvar}; MVA training is treated in Section \ref{sec:mvatra}; and the data statistics only (no systematics) performance of the three MVA discriminants is explored in Section \ref{subsec:statonlyperf}.

\section{Training Samples and Variable Selection}
\label{sec:samvar}
A subset of samples described in Chapter \ref{ch:samples} is used for multivariate analysis training, with $qqZH\to\ell\ell b\bar{b}$ and $ggZH\to\ell\ell b\bar{b}$ used as signal samples and $Z+$jets, $t\bar{t}$, and $VV$ used as background samples.  Truth-tagging (Section \ref{sec:trf}) is used on all samples in MVA training to improve training statistics and stability.  All figures quoted in this section scale distributions to a luminosity of \LUMI.

\subsection{Standard Variables}
The standard set of variables taken as a baseline is the same as used in the fiducial analysis.  The variables fall into several main categories: energy/momentum scales of composite objects ($m_{bb}$, $m_{bbj}$, $p_T^V$, \mll), angles ($\Delta R\left(b_1,b_2\right)$, $\Delta\phi\left(V,H\right)$, $\Delta\eta\left(V,H\right)$), transverse momenta of the jets in the event ($p_T^{b_1}$, $p_T^{b_2}$, $p_T^{j_3}$), and $E_T^{miss}$.  Input distributions for these variables in all the 2 ($\ge3$ jet)  analysis signal regions may be found in Figure \ref{fig:std-kf-inputs2} (\ref{fig:std-kf-inputs3}).  The ``kf'' at the end of variable names denotes that these are derived using 4-vectors that are the result of the kinematic fitter.
\begin{figure}[!htbp]\captionsetup{justification=centering}
  \centering
\begin{subfigure}[t]{0.49\textwidth}\centering\includegraphics[width=\textwidth]{figures/diag//std-kf-2tag2jet-75_150ptv_0of2/variables_id_c1}}\caption{2 jet, low \ptv (1/2)}\end{subfigure}
\begin{subfigure}[t]{0.49\textwidth}\centering\includegraphics[width=\textwidth]{figures/diag//std-kf-2tag2jet-75_150ptv_0of2/variables_id_c2}}\caption{2 jet, low \ptv (2/2)}\end{subfigure}
\begin{subfigure}[t]{0.49\textwidth}\centering\includegraphics[width=\textwidth]{figures/diag//std-kf-2tag2jet-150ptv_0of2/variables_id_c1}}\caption{2 jet, high \ptv (1/2)}\end{subfigure}
\begin{subfigure}[t]{0.49\textwidth}\centering\includegraphics[width=\textwidth]{figures/diag//std-kf-2tag2jet-150ptv_0of2/variables_id_c2}}\caption{2 jet, high \ptv (2/2)}\end{subfigure}
  \caption{Input variables in 2 jet signal regions for the standard variable set.  Signal distributions are in red, and background distributions are in blue.}
  \label{fig:std-kf-inputs2}
\end{figure}
\begin{figure}[!htbp]\captionsetup{justification=centering}
  \centering
\begin{subfigure}[t]{0.49\textwidth}\centering\includegraphics[width=\textwidth]{figures/diag//std-kf-2tag3jet-75_150ptv_0of2/variables_id_c1}}\caption{$\ge3$ jet, low \ptv (1/2)}\end{subfigure}
\begin{subfigure}[t]{0.49\textwidth}\centering\includegraphics[width=\textwidth]{figures/diag//std-kf-2tag3jet-75_150ptv_0of2/variables_id_c2}}\caption{$\ge3$ jet, low \ptv (2/2)}\end{subfigure}
\begin{subfigure}[t]{0.49\textwidth}\centering\includegraphics[width=\textwidth]{figures/diag//std-kf-2tag3jet-150ptv_0of2/variables_id_c1}}\caption{$\ge3$ jet, high \ptv (1/2)}\end{subfigure}
\begin{subfigure}[t]{0.49\textwidth}\centering\includegraphics[width=\textwidth]{figures/diag//std-kf-2tag3jet-150ptv_0of2/variables_id_c2}}\caption{$\ge3$ jet, high \ptv (2/2)}\end{subfigure}
  \caption{Input variables in $\ge3$ jet signal regions for the standard variable set.  Signal distributions are in red, and background distributions are in blue.}
  \label{fig:std-kf-inputs3}
\end{figure}
The distributions in the figure are used as inputs for one of the two k-folded final discriminants, and the order of the distributions is the hyperparameter optimized order for feeding into the BDT; what precisely this means will be discussed in following sections.  While variables in the analysis regions are generally similar, there are some notable exceptions.  \ptv\,and the correlated \drbb\, have different shapes, by construction for the former and by correlation for the latter, at low and high \ptv.  \footnote{Recall that higher \ptv\, means, in a balanced final state like \ZH, the $b$-jet pair will have higher $p_T$ and hence be more collimated (lower \drbb); this is not necessarily the case for background events, as the distributions show.}  The $\ge3$ jet regions also have variables that are not applicable to the 2 jet regions; the inclusion of \texttt{mBBJ} (the invariant mass of the two $b$-jets and leading untagged jet) in particular is of note and suggests a potential avenue forward for refinements of the non-standard variables.

Looking at the correlation matrices for the standard variables in Figure \ref{fig:std-kf-Correlations}, it is easy to see that there are large number of non-trivial correlations
\begin{figure}[!htbp]\captionsetup{justification=centering}
  \centering
\begin{subfigure}[t]{0.220000\textwidth}\centering\includegraphics[width=\textwidth]{figures/diag/std-kf-2tag2jet-75_150ptv_0of2/CorrelationMatrixS}}\caption{2 jet, low \ptv (S)}\end{subfigure}
\begin{subfigure}[t]{0.220000\textwidth}\centering\includegraphics[width=\textwidth]{figures/diag//std-kf-2tag2jet-75_150ptv_0of2/CorrelationMatrixB}}\caption{2 jet, low \ptv (B)}\end{subfigure}
\begin{subfigure}[t]{0.220000\textwidth}\centering\includegraphics[width=\textwidth]{figures/diag//std-kf-2tag3jet-75_150ptv_0of2/CorrelationMatrixS}}\caption{$\ge3$ jet, low \ptv (S)}\end{subfigure}
\begin{subfigure}[t]{0.220000\textwidth}\centering\includegraphics[width=\textwidth]{figures/diag//std-kf-2tag3jet-75_150ptv_0of2/CorrelationMatrixB}}\caption{$\ge3$ jet, low \ptv (B)}\end{subfigure}
\begin{subfigure}[t]{0.220000\textwidth}\centering\includegraphics[width=\textwidth]{figures/diag//std-kf-2tag2jet-150ptv_0of2/CorrelationMatrixS}}\caption{2 jet, high \ptv (S)}\end{subfigure}
\begin{subfigure}[t]{0.220000\textwidth}\centering\includegraphics[width=\textwidth]{figures/diag//std-kf-2tag2jet-150ptv_0of2/CorrelationMatrixB}}\caption{2 jet, high \ptv (B)}\end{subfigure}
\begin{subfigure}[t]{0.220000\textwidth}\centering\includegraphics[width=\textwidth]{figures/diag//std-kf-2tag3jet-150ptv_0of2/CorrelationMatrixS}}\caption{$\ge3$ jet, high \ptv (S)}\end{subfigure}
\begin{subfigure}[t]{0.220000\textwidth}\centering\includegraphics[width=\textwidth]{figures/diag//std-kf-2tag3jet-150ptv_0of2/CorrelationMatrixB}}\caption{$\ge3$ jet, high \ptv (B)}\end{subfigure}
  \caption{Signal and background variable correlations for the standard variable set.}
  \label{fig:std-kf-Correlations}
\end{figure}

\clearpage
\subsection{Lorentz Invariants}
In choosing the set of variables used for a set of Lorentz Invariants based discriminants, we decided to use S. Hagebock's set from \cite{litalk} and related studies.  Distributions of these variables in the same arrangement as with the standard variables may be seen in Figures \ref{fig:li-met-inputs2} and \ref{fig:li-met-inputs3}.
\begin{figure}[!htbp]\captionsetup{justification=centering}
  \centering
\begin{subfigure}[t]{0.49\textwidth}\centering\includegraphics[width=\textwidth]{figures/diag//li-met-2tag2jet-75_150ptv_0of2/variables_id_c1}}\caption{2 jet, low \ptv (1/2)}\end{subfigure}
\begin{subfigure}[t]{0.49\textwidth}\centering\includegraphics[width=\textwidth]{figures/diag//li-met-2tag2jet-75_150ptv_0of2/variables_id_c2}}\caption{2 jet, low \ptv (2/2)}\end{subfigure}
\begin{subfigure}[t]{0.49\textwidth}\centering\includegraphics[width=\textwidth]{figures/diag//li-met-2tag2jet-150ptv_0of2/variables_id_c1}}\caption{2 jet, high \ptv (1/2)}\end{subfigure}
\begin{subfigure}[t]{0.49\textwidth}\centering\includegraphics[width=\textwidth]{figures/diag//li-met-2tag2jet-150ptv_0of2/variables_id_c2}}\caption{2 jet, high \ptv (2/2)}\end{subfigure}
  \caption{Input variables in 2 jet signal regions for the LI variable set.  Signal distributions are in red, and background distributions are in blue.}
  \label{fig:li-met-inputs2}
\end{figure}
\begin{figure}[!htbp]\captionsetup{justification=centering}
  \centering
\begin{subfigure}[t]{0.49\textwidth}\centering\includegraphics[width=\textwidth]{figures/diag//li-met-2tag3jet-75_150ptv_0of2/variables_id_c1}}\caption{$\ge3$ jet, low \ptv (1/2)}\end{subfigure}
\begin{subfigure}[t]{0.49\textwidth}\centering\includegraphics[width=\textwidth]{figures/diag//li-met-2tag3jet-75_150ptv_0of2/variables_id_c2}}\caption{$\ge3$ jet, low \ptv (2/2)}\end{subfigure}
\begin{subfigure}[t]{0.49\textwidth}\centering\includegraphics[width=\textwidth]{figures/diag//li-met-2tag3jet-150ptv_0of2/variables_id_c1}}\caption{$\ge3$ jet, high \ptv (1/2)}\end{subfigure}
\begin{subfigure}[t]{0.49\textwidth}\centering\includegraphics[width=\textwidth]{figures/diag//li-met-2tag3jet-150ptv_0of2/variables_id_c2}}\caption{$\ge3$ jet, high \ptv (2/2)}\end{subfigure}
  \caption{Input variables in $\ge3$ jet signal regions for the LI variable set.  Signal distributions are in red, and background distributions are in blue.}
  \label{fig:li-met-inputs3}
\end{figure}
One thing to note about the variable set chosen here is that \met\,has been added to the standard LI set.  Since the LI construction assumes that this quantity is zero, there is no obvious way to include it.  Nevertheless, as the correlation matrices for the LI variables show in Figure \ref{fig:li-met-Correlations}, there is actually very little correlation between \met\,and the other variables (with this being slightly less the case for the background correlations, as to be expected since \tt, a principal background, is \met-rich).  Hence, if including \met\,violates the spirit somewhat of the LI variables, it does not break terribly much with the aim of having a more orthogonal set.
\begin{figure}[!htbp]\captionsetup{justification=centering}
  \centering
\begin{subfigure}[t]{0.220000\textwidth}\centering\includegraphics[width=\textwidth]{figures/diag/li-met-2tag2jet-75_150ptv_0of2/CorrelationMatrixS}}\caption{2 jet, low \ptv (S)}\end{subfigure}
\begin{subfigure}[t]{0.220000\textwidth}\centering\includegraphics[width=\textwidth]{figures/diag//li-met-2tag2jet-75_150ptv_0of2/CorrelationMatrixB}}\caption{2 jet, low \ptv (B)}\end{subfigure}
\begin{subfigure}[t]{0.220000\textwidth}\centering\includegraphics[width=\textwidth]{figures/diag//li-met-2tag3jet-75_150ptv_0of2/CorrelationMatrixS}}\caption{$\ge3$ jet, low \ptv (S)}\end{subfigure}
\begin{subfigure}[t]{0.220000\textwidth}\centering\includegraphics[width=\textwidth]{figures/diag//li-met-2tag3jet-75_150ptv_0of2/CorrelationMatrixB}}\caption{$\ge3$ jet, low \ptv (B)}\end{subfigure}
\begin{subfigure}[t]{0.220000\textwidth}\centering\includegraphics[width=\textwidth]{figures/diag//li-met-2tag2jet-150ptv_0of2/CorrelationMatrixS}}\caption{2 jet, high \ptv (S)}\end{subfigure}
\begin{subfigure}[t]{0.220000\textwidth}\centering\includegraphics[width=\textwidth]{figures/diag//li-met-2tag2jet-150ptv_0of2/CorrelationMatrixB}}\caption{2 jet, high \ptv (B)}\end{subfigure}
\begin{subfigure}[t]{0.220000\textwidth}\centering\includegraphics[width=\textwidth]{figures/diag//li-met-2tag3jet-150ptv_0of2/CorrelationMatrixS}}\caption{$\ge3$ jet, high \ptv (S)}\end{subfigure}
\begin{subfigure}[t]{0.220000\textwidth}\centering\includegraphics[width=\textwidth]{figures/diag//li-met-2tag3jet-150ptv_0of2/CorrelationMatrixB}}\caption{$\ge3$ jet, high \ptv (B)}\end{subfigure}
  \caption{Signal and background variable correlations for the LI variable set.}
  \label{fig:li-met-Correlations}
\end{figure}
\clearpage
\subsection{RestFrames Variables}
There is no precedent for using the RestFrames variables in the \ZH\,analysis, so a subset of possible RF variables had to be selected as the basis of a discriminant.  The masses and cosines of boost angles from parent frames for the CM, $Z$, and $H$ frames gives six variables, and it was decided that it would be good to match the LI in terms of variable number and treatment (i.e. no special treatment of the third jet), which leaves four more variables.  In addition to the cosines, there are also the $\Delta\phi$ angles.  Furthermore, there are the event-by-event scaled momentum ratios, both longitudinal and transverse.  There is also both a $\Delta\phi$ and an CM-scaled ratio for the \met.  All of these variables were included in a ranking using slightly different training settings as the main hyperparameter optimization variable ranking described below.  The goal of this study was not to develop a discriminant, as the number of variables is too high, but rather to see which ones are generally useful.  
\begin{table}[!htbp]\captionsetup{justification=centering}
\begin{center}
\begin{tabular}{lp{9cm}}
\hline\hline
Region &{\footnotesize Variable Chain}\\
\hline
2jet pTVbin1 & {\tiny \textcolor{green}{Rpt (65.8\%)}, \textcolor{green}{Rpz (29.0\%)}, \textcolor{green}{cosZ (11.4\%)}, \textcolor{red}{MZ (-1.75\%)}, \textcolor{green}{dphiCMH (7.26\%)}, \textcolor{green}{cosCM (3.95\%)}, \textcolor{green}{cosH (0.142\%)}, \textcolor{green}{MCM (2.18\%)}, \textcolor{red}{dphiCMZ (-2.3\%)}, \textcolor{red}{dphiCMMet (-0.236\%)}, \textcolor{green}{dphiLABCM (0.404\%)}, \textcolor{red}{Rmet (-4.04\%)}}\\
\hline
3jet pTVbin1 & {\tiny \textcolor{green}{Rpt (50.8\%)}, \textcolor{green}{Rpz (15.6\%)}, \textcolor{green}{MZ (14.8\%)}, \textcolor{green}{cosZ (3.08\%)}, \textcolor{green}{MCM (3.79\%)}, \textcolor{green}{dphiCMH (3.24\%)}, \textcolor{green}{cosH (0.755\%)}, \textcolor{green}{dphiCMMet (1.04\%)}, \textcolor{red}{Rmet (-1.03\%)}, \textcolor{green}{cosCM (5.31\%)}, \textcolor{red}{dphiCMZ (-1.27\%)}, \textcolor{red}{dphiLABCM (-2.88\%)}, \textcolor{red}{pTJ3 (-1.27\%)}}\\
\hline
2jet pTVbin2 & {\tiny \textcolor{green}{Rpt (52.0\%)}, \textcolor{green}{Rpz (13.8\%)}, \textcolor{green}{cosZ (16.9\%)}, \textcolor{green}{cosH (6.49\%)}, \textcolor{green}{MCM (1.71\%)}, \textcolor{green}{cosCM (6.21\%)}, \textcolor{green}{Rmet (4.25\%)}, \textcolor{red}{dphiCMMet (-1.53\%)}, \textcolor{red}{dphiLABCM (-0.757\%)}, \textcolor{green}{dphiCMH (0.213\%)}, \textcolor{red}{MZ (-0.788\%)}, \textcolor{red}{dphiCMZ (-2.39\%)}}\\
\hline
3jet pTVbin2 & {\tiny \textcolor{green}{Rpt (31.3\%)}, \textcolor{green}{Rpz (21.6\%)}, \textcolor{green}{cosH (8.97\%)}, \textcolor{green}{cosZ (1.42\%)}, \textcolor{green}{cosCM (11.3\%)}, \textcolor{red}{dphiCMZ (-2.84\%)}, \textcolor{green}{MCM (8.17\%)}, \textcolor{red}{dphiCMH (-0.841\%)}, \textcolor{red}{dphiLABCM (-0.00318\%)}, \textcolor{red}{dphiCMMet (-2.6\%)}, \textcolor{red}{pTJ3 (-3.21\%)}, \textcolor{red}{MZ (-1.8\%)}, \textcolor{red}{Rmet (-6.29\%)}}\\
\hline
Aggregate & {\scriptsize Rpt (0,0,0,0), Rpz (1,1,1,1), cosZ (2,3,2,3), cosH (6,6,3,2), MCM (7,4,4,6), MZ (3,2,10,11), dphiCMH (4,5,9,7), cosCM (5,9,5,4), dphiCMMet (9,7,7,9), dphiCMZ (8,10,11,5), Rmet (11,8,6,12), dphiLABCM (10,11,8,8)}\\
\hline
\end{tabular}
\caption{Full RF variable ranking study summary.  Green (red) percentages represent gains (losses) in a validation significance at each step.}
\label{tab:RFDP-METRankingSummary}
\end{center}
\end{table}
Table \ref{tab:RFDP-METRankingSummary} shows the results of this study.  Percent gains (losses) at each step by adding the variable with biggest gain (smallest loss) are shown in green (red).  The final row shows an aggregate ranking, calculated simply by adding up a variables ranks in all bins and ordering the variables smallest to greatest.  This simple aggregation does not take into account which regions are potentially more sensitive and so where taken simply to give an idea of how variables generally performed.  With this in mind, the RF variables were chosen to be the masses \texttt{MCM}, \texttt{MH}, and \texttt{MZ}, the angles \texttt{cosCM}, \texttt{cosH}, \texttt{cosZ}, \texttt{dphiCMH}, and the ratios \texttt{Rpt}, \texttt{Rpz}, and \texttt{Rmet}.  Their distributions may be seen in Figures \ref{fig:rf-sel-inputs2} and \ref{fig:rf-sel-inputs3}.  %Interestingly, the visual difference between signal and background distributions for the angular variables appears much more noticeable for the RestFrames variables than for the LorentzInvariants 

\begin{figure}[!htbp]\captionsetup{justification=centering}
  \centering
\begin{subfigure}[t]{0.49\textwidth}\centering\includegraphics[width=\textwidth]{figures/diag//rf-sel-2tag2jet-75_150ptv_0of2/variables_id_c1}}\caption{2 jet, low \ptv (1/2)}\end{subfigure}
\begin{subfigure}[t]{0.49\textwidth}\centering\includegraphics[width=\textwidth]{figures/diag//rf-sel-2tag2jet-75_150ptv_0of2/variables_id_c2}}\caption{2 jet, low \ptv (2/2)}\end{subfigure}
\begin{subfigure}[t]{0.49\textwidth}\centering\includegraphics[width=\textwidth]{figures/diag//rf-sel-2tag2jet-150ptv_0of2/variables_id_c1}}\caption{2 jet, high \ptv (1/2)}\end{subfigure}
\begin{subfigure}[t]{0.49\textwidth}\centering\includegraphics[width=\textwidth]{figures/diag//rf-sel-2tag2jet-150ptv_0of2/variables_id_c2}}\caption{2 jet, high \ptv (2/2)}\end{subfigure}
  \caption{Input variables in 2 jet signal regions for the RF variable set.  Signal distributions are in red, and background distributions are in blue.}
  \label{fig:rf-sel-inputs2}
\end{figure}
\begin{figure}[!htbp]\captionsetup{justification=centering}
  \centering
\begin{subfigure}[t]{0.49\textwidth}\centering\includegraphics[width=\textwidth]{figures/diag//rf-sel-2tag3jet-75_150ptv_0of2/variables_id_c1}}\caption{$\ge3$ jet, low \ptv (1/2)}\end{subfigure}
\begin{subfigure}[t]{0.49\textwidth}\centering\includegraphics[width=\textwidth]{figures/diag//rf-sel-2tag3jet-75_150ptv_0of2/variables_id_c2}}\caption{$\ge3$ jet, low \ptv (2/2)}\end{subfigure}
\begin{subfigure}[t]{0.49\textwidth}\centering\includegraphics[width=\textwidth]{figures/diag//rf-sel-2tag3jet-150ptv_0of2/variables_id_c1}}\caption{$\ge3$ jet, high \ptv (1/2)}\end{subfigure}
\begin{subfigure}[t]{0.49\textwidth}\centering\includegraphics[width=\textwidth]{figures/diag//rf-sel-2tag3jet-150ptv_0of2/variables_id_c2}}\caption{$\ge3$ jet, high \ptv (2/2)}\end{subfigure}
  \caption{Input variables in $\ge3$ jet signal regions for the RF variable set.  Signal distributions are in red, and background distributions are in blue.}
  \label{fig:rf-sel-inputs3}
\end{figure}

Correlations for the chosen RF variables are shown in Figure \ref{fig:rf-sel-Correlations}.  These correlations are much lower than for the standard case but still slightly higher than for the LI case.  Notably, many strong correlations that exist for signal events do not exist in background events and vice versa, so what is lost in orthogonality may very well be recuperated in grater separation\footnote{It is very hard to say for certain whether this is the case for MVA discriminants, and such dedicated studies might make worthwhile future studies.}.  Given the generally better performance of the RF sets, as we shall see in following sections and chapters, this slight tradeoff is likely an aesthetic one, with the main benefits of a more orthogonal basis likely realized at this level of correlation.

\begin{figure}[!htbp]\captionsetup{justification=centering}
  \centering
\begin{subfigure}[t]{0.220000\textwidth}\centering\includegraphics[width=\textwidth]{figures/diag/rf-sel-2tag2jet-75_150ptv_0of2/CorrelationMatrixS}}\caption{2 jet, low \ptv (S)}\end{subfigure}
\begin{subfigure}[t]{0.220000\textwidth}\centering\includegraphics[width=\textwidth]{figures/diag//rf-sel-2tag2jet-75_150ptv_0of2/CorrelationMatrixB}}\caption{2 jet, low \ptv (B)}\end{subfigure}
\begin{subfigure}[t]{0.220000\textwidth}\centering\includegraphics[width=\textwidth]{figures/diag//rf-sel-2tag3jet-75_150ptv_0of2/CorrelationMatrixS}}\caption{$\ge3$ jet, low \ptv (S)}\end{subfigure}
\begin{subfigure}[t]{0.220000\textwidth}\centering\includegraphics[width=\textwidth]{figures/diag//rf-sel-2tag3jet-75_150ptv_0of2/CorrelationMatrixB}}\caption{$\ge3$ jet, low \ptv (B)}\end{subfigure}
\begin{subfigure}[t]{0.220000\textwidth}\centering\includegraphics[width=\textwidth]{figures/diag//rf-sel-2tag2jet-150ptv_0of2/CorrelationMatrixS}}\caption{2 jet, high \ptv (S)}\end{subfigure}
\begin{subfigure}[t]{0.220000\textwidth}\centering\includegraphics[width=\textwidth]{figures/diag//rf-sel-2tag2jet-150ptv_0of2/CorrelationMatrixB}}\caption{2 jet, high \ptv (B)}\end{subfigure}
\begin{subfigure}[t]{0.220000\textwidth}\centering\includegraphics[width=\textwidth]{figures/diag//rf-sel-2tag3jet-150ptv_0of2/CorrelationMatrixS}}\caption{$\ge3$ jet, high \ptv (S)}\end{subfigure}
\begin{subfigure}[t]{0.220000\textwidth}\centering\includegraphics[width=\textwidth]{figures/diag//rf-sel-2tag3jet-150ptv_0of2/CorrelationMatrixB}}\caption{$\ge3$ jet, high \ptv (B)}\end{subfigure}
  \caption{Signal and background variable correlations for the RF variable set.}
  \label{fig:rf-sel-Correlations}
\end{figure}

A summary of the variables used in the three cases is given in \ref{tab:variables}.

\begin{table}[!htbp]\captionsetup{justification=centering}
  \begin{center}\begin{tabular}{l|p{4.5in}}
      \hline\hline
      Variable Set & Variables\\
      \hline
      Standard & \texttt{mBB,\- mLL, (mBBJ),\- pTV,\- pTB1,\- pTB2,\- (pTJ3),\- dRBB,\- dPhiVBB,\- dEtaVBB,\- MET} 9 (11) vars\\
      Lorentz Invariants &  \texttt{j0\_j1,\- j0\_l1,\- l0\_l1,\- j1\_l1,\- j0\_l0,\- j1\_l0,\- gamma\_ZHz,\- angle\_bbz\_bbll\- angle\_bb\_z,\- MET} 10 vars\\
      RestFrames & \texttt{MH,\- MCM,\- MZ,\- cosH,\- cosCM,\- cosZ,\- Rpz,\- Rpt,\- dphiCMH,\- Rmet} 10 vars\\
      \hline\hline
    \end{tabular}
    \caption{Variables used in MVA training.  Variables in parentheses are only used in the $\ge 3$ jet regions.}
    \label{tab:variables}
  \end{center}
\end{table}

%In addition to the standard set of variables used for MVA training, two additional sets of variables, described at the end of Chapter \ref{ch:theory} were used: the Lorentz Invariants (LI) and RestFrames (RF) inspired variable sets.  These will be discussed below.


\section{MVA Training}
\label{sec:mvatra}
With variables chosen, the MVA discriminants must be trained and optimized.  MVA training and hyperparameter optimization (in this case, just the order in which variables are fed into the MVA) is conducted using the ``holdout'' method.
%{LOOK at roger's thesis for a reference}.
  In this scheme, events are divided into three equal portions (in this case using \texttt{EventNumber\%3}), with the first third (the ``training'' set) being used for the initial training, the second third (the ``validation'' set) being used for hyperparameter optimizaiton, and the final third (the ``testing'' set) used to evaluate the performance of the final discriminants in each analysis region.

The MVA discriminant used is a boosted decision tree (BDT).  Training is done in TMVA using the training settings of the fiducial analysis \cite{supportnote}\footnote{Namely, \texttt{!H:\-!V:\-BoostType=\-AdaBoost:\-AdaBoostBeta=0.15:\-SeparationType=\-GiniIndex:\-PruneMethod=\-NoPruning:\-NTrees=\-200:\-MaxDepth=4:\-nCuts=100:\-nEventsMin=5\%}}.  For the purposes of hyperparameterization and testing, transformation D with $z_s=z_b=10$ is applied to the BDT distributions, and the cumulative sum of the significance $S/\sqrt{S+B}$ in each bin is calculated for each pair of distributions.  

Transformation D is a histogram transformation, developed during the Run 1 SM \vhbb\, search, designed to reduce the number of bins in final BDT distributions and thereby mitigate the effect of statistical fluctuations in data while also maintaining sensitivity.  Such an arbitrary transformation may be expressed as:
\begin{equation}
Z\left(I\left[k,l\right]\right) = Z\left(z_s,n_s\left(I\left[k,l\right]\right),N_s,z_b,n_b\left(I\left[k,]]\right),N_b\right)
\end{equation}
where
\begin{itemize}
\item $I[k,l]$ is an interval of the histograms, containing the bins between bin $k$ and bin $l$;
\item $N_{s}$ is the total number of signal events in the histogram;
\item $N_{b}$ is the total number of background events in the histogram;
\item $n_{s}(I[k,l])$ is the total number of signal events in the interval $I[k,l]$;
\item $n_{b}(I[k,l])$ is the total number of background events in the interval $I[k,l]$;
\item $z_{s}$ and $z_{b}$ are parameters used to tune the algorithm.
\end{itemize}
Transformation D uses:
\begin{equation}
Z = z_{s}\frac{n_{s}}{N_{s}} + z_{b}\frac{n_{b}}{N_{b}}
\end{equation}
Rebinning occurs as follow:
\begin{enumerate}
\item Begin with the highest valued bin in the original pair of distributions.  Call this the ``last'' bin and use it as $l$, and have $k$ be this bin as well.
\item Calculate $Z\left(I\left[k,l\right]\right)$
\item If $Z\le1$, set $k\to k-1$ and return to step 2.  If not, rebin bins $k$--$l$ into a single bin and name $k-1$ the new ``last'' bin $l$.
\item Continue until all bins have been iterated through; if $Z\le1$ for any remaining $n$ of the lowest-valued bins (as is often the case), simply rebin these as a single bin.
\end{enumerate}

Variable ranking is done iteratively (greedily) in each analysis region.  In each set, the validation significance of a BDT using an initial subset of variables is calculated (\texttt{dRBB} and \texttt{mBB} for the standard set; \texttt{j0\_j1} for the LI set; and \texttt{MH} for the RF set).  Each of the remaining unranked variables are then added separately, one at a time, to the BDT.  The variable yielding the highest validation significance is then added to the set list of ranked variables and removed from the list of unranked variables.  This process is repeated until no variables remain.  These rankings are shown in Figures \ref{fig:std-kf-Ranking}--\ref{fig:rf-sel-Ranking}.  Rankings tend to be fairly stable.

\begin{figure}[!htbp]\captionsetup{justification=centering}
  \centering
\begin{subfigure}[t]{0.49\textwidth}\centering\includegraphics[width=\textwidth]{figures/diag//training-std-kf-2tag2jet-75_150ptv_tD}}\caption{2 jet, low pTV}\end{subfigure}
\begin{subfigure}[t]{0.49\textwidth}\centering\includegraphics[width=\textwidth]{figures/diag//training-std-kf-2tag3jet-75_150ptv_tD}}\caption{$\ge3$ jet, low pTV}\end{subfigure}
\begin{subfigure}[t]{0.49\textwidth}\centering\includegraphics[width=\textwidth]{figures/diag//training-std-kf-2tag2jet-150ptv_tD}}\caption{2 jet, high pTV}\end{subfigure}
\begin{subfigure}[t]{0.49\textwidth}\centering\includegraphics[width=\textwidth]{figures/diag//training-std-kf-2tag3jet-150ptv_tD}}\caption{$\ge3$ jet, high pTV}\end{subfigure}
  \caption{Rankings for the standard variable set.}
  \label{fig:std-kf-Ranking}
\end{figure}

\begin{figure}[!htbp]\captionsetup{justification=centering}
  \centering
\begin{subfigure}[t]{0.49\textwidth}\centering\includegraphics[width=\textwidth]{figures/diag//training-li-met-2tag2jet-75_150ptv_tD}}\caption{2 jet, low pTV}\end{subfigure}
\begin{subfigure}[t]{0.49\textwidth}\centering\includegraphics[width=\textwidth]{figures/diag//training-li-met-2tag3jet-75_150ptv_tD}}\caption{$\ge3$ jet, low pTV}\end{subfigure}
\begin{subfigure}[t]{0.49\textwidth}\centering\includegraphics[width=\textwidth]{figures/diag//training-li-met-2tag2jet-150ptv_tD}}\caption{2 jet, high pTV}\end{subfigure}
\begin{subfigure}[t]{0.49\textwidth}\centering\includegraphics[width=\textwidth]{figures/diag//training-li-met-2tag3jet-150ptv_tD}}\caption{$\ge3$ jet, high pTV}\end{subfigure}
  \caption{Rankings for the LI variable set.}
  \label{fig:li-met-Ranking}
\end{figure}

\begin{figure}[!htbp]\captionsetup{justification=centering}
  \centering
\begin{subfigure}[t]{0.49\textwidth}\centering\includegraphics[width=\textwidth]{figures/diag//training-rf-sel-2tag2jet-75_150ptv_tD}}\caption{2 jet, low pTV}\end{subfigure}
\begin{subfigure}[t]{0.49\textwidth}\centering\includegraphics[width=\textwidth]{figures/diag//training-rf-sel-2tag3jet-75_150ptv_tD}}\caption{$\ge3$ jet, low pTV}\end{subfigure}
\begin{subfigure}[t]{0.49\textwidth}\centering\includegraphics[width=\textwidth]{figures/diag//training-rf-sel-2tag2jet-150ptv_tD}}\caption{2 jet, high pTV}\end{subfigure}
\begin{subfigure}[t]{0.49\textwidth}\centering\includegraphics[width=\textwidth]{figures/diag//training-rf-sel-2tag3jet-150ptv_tD}}\caption{$\ge3$ jet, high pTV}\end{subfigure}
  \caption{Rankings for the RF variable set.}
  \label{fig:rf-sel-Ranking}
\end{figure}

Once variables have been ranked, the BDT may be used both to evaluate performance in a simplified analysis scenario in the absence of systematic uncertainties (described below in Section \ref{subsec:statonlyperf}) and to create xml files for the production of fit inputs for an analysis including systematics.  Following the approach taken in the fiducial analysis, BDT discriminants using two ``k-folds'' are produced to prevent overtraining, since the samples used for training are the same as those used to produce inputs for the full profile likelihood fit.  In this scheme, a BDT trained on events with an even (odd) \texttt{EventNumber} are used to evaluate events with an odd (even) \texttt{EventNumber}.

\section{Statistics Only BDT Performance}
\label{subsec:statonlyperf}
As described above, cumulative significances can be extracted from pairs of signal and background BDT output distributions in a given region.  In order to evaluate performance of variable sets in the absence of systematic uncertainties, such pairs can be constructed by evaluating BDT score on the testing set of events using the optimal variable rankings in each region.  We show two versions of each testing distribution for each variable set in each signal region in Figures \ref{fig:std-kf-testing}--\ref{fig:rf-sel-testing}.  The training distribution is always shown as points.  The plots with block histograms with numbers of bins that match (do not match) the training distribution do not (do) have transformation D applied.  Transformation D histograms are included to show the distributions actually used for significance evaluation, while the untransformed histograms are included to illustrate that the level of overtraining is not too terrible\footnote{The raw distributions include a K-S test statistic for signal (background) distributions.}.  For better comparison of the distributions, all histograms have been scaled to have the same normalization.


\begin{figure}[!htbp]\captionsetup{justification=centering}
  \centering
  \begin{subfigure}[t]{0.49\textwidth}\centering\includegraphics[width=\textwidth]{figures/diag/bdt_dist-test-std-kf-2tag2jet-75_150ptv_tD-all}}\caption{2 jet, low pTV}\end{subfigure}
  \begin{subfigure}[t]{0.49\textwidth}\centering\includegraphics[width=\textwidth]{figures/diag/bdt_dist-test-std-kf-2tag2jet-75_150ptv_tD-all_no-tD}}\caption{2 jet, low pTV (no tD)}\end{subfigure}
  \begin{subfigure}[t]{0.49\textwidth}\centering\includegraphics[width=\textwidth]{figures/diag/bdt_dist-test-std-kf-2tag3jet-75_150ptv_tD-all}}\caption{$\ge3$ jet, low pTV}\end{subfigure}
  \begin{subfigure}[t]{0.49\textwidth}\centering\includegraphics[width=\textwidth]{figures/diag/bdt_dist-test-std-kf-2tag3jet-75_150ptv_tD-all_no-tD}}\caption{$\ge3$ jet, low pTV (no tD)}\end{subfigure}
  \begin{subfigure}[t]{0.49\textwidth}\centering\includegraphics[width=\textwidth]{figures/diag/bdt_dist-test-std-kf-2tag2jet-150ptv_tD-all}}\caption{2 jet, high pTV}\end{subfigure}
  \begin{subfigure}[t]{0.49\textwidth}\centering\includegraphics[width=\textwidth]{figures/diag/bdt_dist-test-std-kf-2tag2jet-150ptv_tD-all_no-tD}}\caption{2 jet, high pTV (no tD)}\end{subfigure}
  \begin{subfigure}[t]{0.49\textwidth}\centering\includegraphics[width=\textwidth]{figures/diag/bdt_dist-test-std-kf-2tag3jet-150ptv_tD-all}}\caption{$\ge3$ jet, high pTV}\end{subfigure}
  \begin{subfigure}[t]{0.49\textwidth}\centering\includegraphics[width=\textwidth]{figures/diag/bdt_dist-test-std-kf-2tag3jet-150ptv_tD-all_no-tD}}\caption{$\ge3$ jet, high pTV (no tD)}\end{subfigure}
  \caption{Training (points) and testing (block histogram) MVA distributions used for stat only testing for the standard variable set.}
  \label{fig:std-kf-testing}
\end{figure}

\begin{figure}[!htbp]\captionsetup{justification=centering}
  \centering
  \begin{subfigure}[t]{0.49\textwidth}\centering\includegraphics[width=\textwidth]{figures/diag/bdt_dist-test-li-met-2tag2jet-75_150ptv_tD-all}}\caption{2 jet, low pTV}\end{subfigure}
  \begin{subfigure}[t]{0.49\textwidth}\centering\includegraphics[width=\textwidth]{figures/diag/bdt_dist-test-li-met-2tag2jet-75_150ptv_tD-all_no-tD}}\caption{2 jet, low pTV (no tD)}\end{subfigure}
  \begin{subfigure}[t]{0.49\textwidth}\centering\includegraphics[width=\textwidth]{figures/diag/bdt_dist-test-li-met-2tag3jet-75_150ptv_tD-all}}\caption{$\ge3$ jet, low pTV}\end{subfigure}
  \begin{subfigure}[t]{0.49\textwidth}\centering\includegraphics[width=\textwidth]{figures/diag/bdt_dist-test-li-met-2tag3jet-75_150ptv_tD-all_no-tD}}\caption{$\ge3$ jet, low pTV (no tD)}\end{subfigure}
  \begin{subfigure}[t]{0.49\textwidth}\centering\includegraphics[width=\textwidth]{figures/diag/bdt_dist-test-li-met-2tag2jet-150ptv_tD-all}}\caption{2 jet, high pTV}\end{subfigure}
  \begin{subfigure}[t]{0.49\textwidth}\centering\includegraphics[width=\textwidth]{figures/diag/bdt_dist-test-li-met-2tag2jet-150ptv_tD-all_no-tD}}\caption{2 jet, high pTV (no tD)}\end{subfigure}
  \begin{subfigure}[t]{0.49\textwidth}\centering\includegraphics[width=\textwidth]{figures/diag/bdt_dist-test-li-met-2tag3jet-150ptv_tD-all}}\caption{$\ge3$ jet, high pTV}\end{subfigure}
  \begin{subfigure}[t]{0.49\textwidth}\centering\includegraphics[width=\textwidth]{figures/diag/bdt_dist-test-li-met-2tag3jet-150ptv_tD-all_no-tD}}\caption{$\ge3$ jet, high pTV (no tD)}\end{subfigure}
  \caption{Training (points) and testing (block histogram) MVA distributions used for stat only testing for the LI variable set.}
  \label{fig:li-met-testing}
\end{figure}

\begin{figure}[!htbp]\captionsetup{justification=centering}
  \centering
  \begin{subfigure}[t]{0.49\textwidth}\centering\includegraphics[width=\textwidth]{figures/diag/bdt_dist-test-rf-sel-2tag2jet-75_150ptv_tD-all}}\caption{2 jet, low pTV}\end{subfigure}
  \begin{subfigure}[t]{0.49\textwidth}\centering\includegraphics[width=\textwidth]{figures/diag/bdt_dist-test-rf-sel-2tag2jet-75_150ptv_tD-all_no-tD}}\caption{2 jet, low pTV (no tD)}\end{subfigure}
  \begin{subfigure}[t]{0.49\textwidth}\centering\includegraphics[width=\textwidth]{figures/diag/bdt_dist-test-rf-sel-2tag3jet-75_150ptv_tD-all}}\caption{$\ge3$ jet, low pTV}\end{subfigure}
  \begin{subfigure}[t]{0.49\textwidth}\centering\includegraphics[width=\textwidth]{figures/diag/bdt_dist-test-rf-sel-2tag3jet-75_150ptv_tD-all_no-tD}}\caption{$\ge3$ jet, low pTV (no tD)}\end{subfigure}
  \begin{subfigure}[t]{0.49\textwidth}\centering\includegraphics[width=\textwidth]{figures/diag/bdt_dist-test-rf-sel-2tag2jet-150ptv_tD-all}}\caption{2 jet, high pTV}\end{subfigure}
  \begin{subfigure}[t]{0.49\textwidth}\centering\includegraphics[width=\textwidth]{figures/diag/bdt_dist-test-rf-sel-2tag2jet-150ptv_tD-all_no-tD}}\caption{2 jet, high pTV (no tD)}\end{subfigure}
  \begin{subfigure}[t]{0.49\textwidth}\centering\includegraphics[width=\textwidth]{figures/diag/bdt_dist-test-rf-sel-2tag3jet-150ptv_tD-all}}\caption{$\ge3$ jet, high pTV}\end{subfigure}
  \begin{subfigure}[t]{0.49\textwidth}\centering\includegraphics[width=\textwidth]{figures/diag/bdt_dist-test-rf-sel-2tag3jet-150ptv_tD-all_no-tD}}\caption{$\ge3$ jet, high pTV (no tD)}\end{subfigure}
  \caption{Training (points) and testing (block histogram) MVA distributions used for stat only testing for the RF variable set.}
  \label{fig:rf-sel-testing}
\end{figure}

As can be seen in the summary of cumulative significances for each of these analysis regions and variable sets in Figure \ref{fig:statonlysob}, the performance of each of the variable sets is quite similar.  The standard set performs best, with the LI (RF) set having a cumulative significance that is 7.9\% (6.9\%) lower.  This suggests that the LI and RF variables, in the \ZH\,closed final state, have no more intrinsic descriptive power than the standard set.  That these figures are all relatively high ($\sim4.5$) is due largely to the absence of systematics and possibly in part due to the fact that many of the most significant bins occur at high values of the BDT output, which, as can be seen in any of the testing distributions, contain a small fraction of background events.  
\begin{figure}[!htbp]\captionsetup{justification=centering}
  \centering
  \includegraphics[width=0.66\linewidth]{figures/diag/rfli_sig_mtrx_tD}
  \caption{Results of testing significances sorted by analysis region and variable set.}
  \label{fig:statonlysob}
\end{figure}
An interesting feature to note in Figure \ref{fig:statonlysob} is that while the standard set does perform better in all regions, the gap is larger in the $\ge3$ jet regions, suggesting that further optimization in the $\ge3$ jet case could be useful.  Moreover, as discussed at the end of Chapter \ref{ch:object}, the choice of $\ge3$ jet and not exclusive 3 jet regions is a 2-lepton specific choice and may not be justified for the non-standard variable sets.

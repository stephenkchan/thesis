%!TEX root = ../dissertation.tex
\begin{savequote}[75mm]
Tod-Not-Brot
\qauthor{Old German Proverb}
\end{savequote}

\chapter{Telescoping Jets}
\label{ch:teljet}
\newthought{Another approach to improving} \ZH\, is the use of telescoping jets \cite{teljet}, which harnesses the power of multiple event interpretations.  The use of multiple event interpretations was originally developed with non-deterministic jet algorithms like the Q-jets (``quantum'' jets) algorithm \cite{qjet}.  When a traditional or ``classical'' algorithm, such as the Cambridge-Aachen\cite{ca} and anti-$k_t$\cite{akt} algorithms, is applied to an event, it  produces one set of jets for that event, i.e. a single interpretation of that event.  With multiple event interpretations, each event is instead given an ensemble of interpretations.  In the case of Q-jets, this ensemble is created through a non-deterministic clustering process for an anti-$k_t$ jet algorithm.  With telescoping jets, multiple jet cone radii (the characteristic size parameter, $R$) around a set of points in the pseudorapidity-azimuth ($\eta-\phi$) plane are used to generate a series of jet collections.  Instead of an event passing or not-passing a given set of cuts, a fraction (called the cut-weight, $z$) of interpretations will pass these cuts.  This cut-weight allows for enhanced background suppression and increased significance of observed quantities for a given data set, as detailed in Ref. \cite{multint}.  The telescoping jets algorithm provides the benefits of multiple event interpretations without the significant computational overhead of a non-deterministic  algorithm like the Q-jets algorithm, and its multiple cone sizes are particularly suited to studying processes like associated production, which suffers from a pronounced low tail in the dijet invariant mass distribution due to final state radiation (FSR) ``leaking'' outside the relatively narrow jets used for object reconstruction. 

\section{Monte Carlo Simulation}
\label{sec:mc}
The MC simulated samples used in this study are the same as in Ref. \cite{conf}.  The signal sample used is generated in \textsc{Pythia8} \cite{tjconf18} with the CTEQ6L1 parton distributions functions (PDFs) and AU2 tune\cite{tjconf19,tjconf20,tjconf21} for the $ZH$ process with $m_H=125$ \GeV\ (henceforth, $ZH125$).  The primary background processes examined in this study were $Z+$jets with massive $b$ and $c$ quarks.  These samples are generated with version 1.4.1 of the \textsc{Sherpa} generator \cite{tjconf28}.% at leading order (LO).  Additionally, $t\bar{t}$ production and di-boson ($ZZ$) production processes were studied in validation.  The $t\bar{t}$ samples are generated by the \textsc{Powheg} generator \cite{tjconf30,conf32}, using CT10 PDFs, interfaced with \textsc{Pythia6}, and the $ZZ$ samples are generated with the \textsc{Herwig} \cite{tjconf40} generator using the CTEQ6L1 PDFs and the AUET2 tune \cite{tjconf20,conf21}.

\section{Jet Reconstruction and Calibration}
\label{sec:cal}
In order to construct telescoping jets, jet axes must first be found around which to ``telescope.''  In the reconstructed-level analysis, the anti-$k_t$ algorithm with $R=0.4$ is used to reconstruct jets from topological clusters in the calorimeters.  The four vectors of these anti-$k_t$ algorithm with $R=0.4$ jets are calibrated to match truth information obtained from simulation and validated in data.  To take into account the effect of pile-up interactions, jet energies are corrected using a jet-area based technique \cite{tjconf54}, and each jet with $p_T<50$ \GeV\ and $\left|\eta\right|<$2.4 is subject to a requirement that at least 50\% of the scalar sum of the $p_T$ of tracks matched to this jet be composed of tracks also associated with the primary vertex.  Jet energies are also calibrated using $p_T$ and $\eta$-dependent correction factors \cite{tjconf55}.  Furthermore, at least two jets must have $\left|\eta\right|<2.5$ in order to be $b$-tagged.  The MV1 algorithm \cite{tjconf56,conf57,conf58,conf59,conf60} is used for $b$-tagging.  Once jets are reconstructed and $b$-tag weights have been calculated, the two hardest, $b$-tagged jets are used as the telescoping jet axes.  Additional details can be found in Ref. \cite{tjconf}.

After the telescoping jet axes have been established, telescoping jets are constructed using topological clusters in the calorimeters at a variety of jet cone sizes.  Including the original anti-$k_t$ jets used for the $R=0.4$ case, twelve total sets of jets of cone sizes ranging from $R=0.4$--1.5 are constructed, with each successive size having a radius 0.1 larger than the preceding set.  For each jet axis, telescoping jets consist of any topological cluster lying within $R$ of the axis.  In the event of overlap, clusters are assigned to the closer jet axis.  If a given cluster is equidistant from the two jet axes, the cluster is assigned to whichever jet axis is associated with the anti-$k_t$ jet with higher $p_T$.  Calibration for the telescoping jets is conducted using corrections for anti-$k_t$ calorimeter topological cluster jets; the $R=0.4$ corrections are used for telescoping $R=0.5$, and the $R=0.6$ corrections are used for telescoping $R\ge0.6$ (cf. Sec. \ref{sec:val}).  The telescoping cone jets ($R\ge0.5$) at reconstructed level are trimmed using Cambridge-Aachen jets with $R=0.3$ and $f_{cut}=0.05$ with respect to the untrimmed jet $p_T$ \cite{trimjet}.  Since these jets are trimmed, the active area correction is not applied.  In the event a $Z$ candidate electron falls within $R$ of the axis of a telescoping jet, its 4-momentum is subtracted from that of the jet vectorially.

A similar process is used to construct telescoping jets in the truth-level analysis below. Instead of the two hardest $b$-tagged anti-$k_t$ with $R=0.4$ jets reconstructed with calorimeter topological clusters, the two hardest truth $b$-jets in an event are used.  Instead of making a cut on $b$-tagging weight to $b$-tag, truth jets are examined to see whether a $b$-hadron with $p_T>5$ \GeV\ is contained within $\Delta R<0.4$ of the jet axis; the presence of a $b$-hadron is used to $b$-tag truth-level jets.  These two jets again provide the jets for the $R=0.4$ case and the axes around which telescoping takes place.  Stable truth particles, not including muons and neutrinos, are used in place of calorimeter topological clusters.  $Z$ candidate electron-telescoping jet overlap removal is perfored at truth level, too.  Missing $E_T$ is calculated using the vector sum of the four momenta of stable truth-level neutrinos.  Since there are no pileup particles stored at truth level, truth-level telescoping jets are not trimmed.

%For jet calibration validation, both reconstructed and truth jets were assembled as above but all jets in an event with $p_T\ge20$ \GeV\ were reconstructed and studied.

\section{Event Reconstruction and Selection}
\label{sec:evsel}
Events are selected on the basis of a combination of leptonic, jet, and missing $E_T$ requirements, which are outlined in Table \ref{tab:sumreq}.  Leptons are categorized by three sets of increasingly stringent quality requirements, which include lower limits on $E_T$, upper limits on $\left|\eta\right|$, impact-parameter requirements, and track-based isolation criteria.  The requirements differ for electrons \cite{tjconf50} and muons \cite{tjconf51}.  Events are selected with a combination of single lepton, dielectron, and dimuon requirements.  Each event must contain at least one lepton passing medium requirements and at least one other lepton passing loose requirements.  These leptons are used to create a dilepton invariant mass cut to ensure the presence of a $Z$ boson and suppress multijet backgrounds.

Event selection requirements are also imposed on the anti-$k_t$ with $R=0.4$ jets.  There must be at least two $b$-tagged jets in a given event.  The $p_T$ of the harder $b$-tagged jet must be at least 45 \GeV, and the second $b$-tagged jet must have $p_T$ of at least 20 \GeV .  There are further topological cuts on the separation of the two jets $\Delta R\left(b,\bar{b}\right)$, the distance between the two jets in the $\left(\eta,\phi\right)$ plane, according to the transverse momentum of the $Z$ boson, $p_T^Z$.  These are shown in Table \ref{tab:dRreq}.

The truth-level analysis has the same missing $E_T$, jet $p_T$, $m_{ll}$, and additional topological selection criteria, but the use of truth-level information simplifies the other requirements.  Instead of lepton quality requirements, $Z$ boson candidate leptons' statuses and MC record barcodes are checked to ensure the leptons are stable.

In the jet calibration validation, the reconstructed level analysis lepton and $m_{ll}$ requirements are imposed, but neither the missing $E_T$ nor the jet selection requirements are applied so as not to bias the validation.

\begin{table}[htbp]
\caption{A summary of basic event selection requirements.  Truth-level $b$-tagging is done with truth-level information.
\label{tab:sumreq}}
\begin{center}
\begin{tabular}{|l|l|l|l|}
\hline
Requirement & Reconstructed & Truth & Validation\\
\hline
Leptons & 1 medium + 1 loose lepton & 2 produced by $Z$ boson & 1 medium + 1 loose lepton \\
\hline
$b$-jet & 2 $b$-tags & 2 $b$-jets & ---\\
\hline
$p_T$ jet 1 (jet 2) & \multicolumn{2}{|c|}{$>45$ \GeV\ ($>20$) \GeV\ }  & ---\\
\hline
Missing $E_T$ & \multicolumn{2}{|c|}{$E_T^{\text{miss}}<60$ \GeV\ }& ---\\
\hline
$Z$ boson & \multicolumn{3}{|c|}{$83<m_{ll}<99$ \GeV\ }\\
\hline
\end{tabular}
\end{center}
\end{table}

\begin{table}[htbp]
\caption{Topological requirements of the event selection.
\label{tab:dRreq}}
\begin{center}
\begin{tabular}{|l|l|}
\hline
$p_T^Z$ [\GeV] & $\Delta R\left(b,\bar{b}\right)$\\
\hline
0--90 & 0.7-3.4\\
90--120 & 0.7-3.0\\
120--160 & 0.7--2.3\\
160--200 & 0.7-1.8 \\
$>200$ & $<1.4$ \\
\hline
\end{tabular}
\end{center}
\end{table}


\section{Validation of Jet Calibration}
\label{sec:val}
In order to validate the jet energy scale and resolution of jets constructed with this telescoping-jets algorithm, values of $p_T^{rec}/p_T^{tru}$ are studied for each value of $R$ for the $Z+$jets MC sample.  In a given event, all jets, not just the two hardest $b$-tagged jets, are telescoped.  These jets are constructed in the same way as in the reconstructed and truth-level analyses; reconstructed-level jets are made from calorimeter topological clusters within $R$ of the anti-$k_t$ with $R=0.4$ jet axes and then trimmed, and truth-level jets are made from stable truth particles within $R$ of the anti-$k_t$ with $R=0.4$ jet axes.  The reconstructed and truth-level telescoping jet ensembles are matched according to the separation in the $\left(\eta,\phi\right)$ plane of their corresponding anti-$k_t$ with $R=0.4$ jets used as seeds.  Only jets with $\left|\eta\right|<1.2$ are examined here, and the results of studies on the $ZH125$, $ZZ$, and $t\bar{t}$ samples, as well as over other $\left|\eta\right|$ ranges, are outlined in \cite{teljet}.  Any reconstructed jets not within $\Delta R=0.3$ of a truth jet are discarded.  In the event that multiple reconstructed jets are the same distance away from a given truth jet, the reconstructed jet with the highest $p_T$ gets matched.  Matching is retained for all $R$ values (i.e. telescoping seeds are matched, and telescoping jets are assumed to match if the anti-$k_t$ jets from which their seeds are derived match).

Once anti-$k_t$ with $R=0.4$ reconstructed and truth jets are matched, response functions are created by generating a series of distributions of $p_T^{rec}/p_T^{tru}$ in 20 \GeV\ bins of $p_T^{tru}$ from 20--200 \GeV, one bin for 200--300 \GeV, and one bin for 300--500 \GeV\ for each $R$, with bins chosen for purposes of statistics.  Ensembles with $p_T^{tru}<20$ \GeV\ are ignored since no calibration exists for jets with transverse momentum below this value.  The values of $\left<p_T^{rec}/p_T^{tru}\right>$ in each $p_T^{tru}$ bin are calculated by doing a two sigma gaussian fit on the distribution of $p_T^{rec}/p_T^{tru}$ in that bin and taking the mean of that fit, and the error on the mean is taken from the error of this parameter in the fit.  The resolutions are the values of the square root of the variance on this fit.  As the total response distributions in Figure \ref{fig:valzjets} show, performance is best for low $R$ values and high values of $p_T^{tru}$.  Figure \ref{fig:valzjets} shows the $R=0.4$ (anti$k_t$) case to show a baseline for performance, $R=0.6$ to show the deviations with ``correct'' calibrations, and $R=1.0,1.5$ to show how big those deviations get with larger $R$ jets.  The resolutions, $\sigma_{p_T^{rec}/p_T^{tru}}$, as a function of $p_T^{tru}$ are shown in Figure \ref{fig:valzjets}(b). For $p_T^{tru}>60$ \GeV, response is fairly consistent over various $R$ values.  Resolution, as might na\"ively be expected, is worse for increasingly larger values of $R$.  For $p_T^{tru}<60$ \GeV, resolution degrades, and response degrades in particular for increasing $R$; this is likely a result from residual pileup effects.

\begin{figure}[!htbp]\captionsetup{justification=centering}
  \begin{center}
\begin{subfigure}[t]{18pc}\centering\includegraphics[width=\textwidth]{figures/body/val_wztrim-r_ratio_eta0_12_zjets}\caption{}\end{subfigure}
\begin{subfigure}[t]{18pc}\centering\includegraphics[width=\textwidth]{figures/body/val_wztrim-r_resol_eta0_12_zjets}\caption{}\end{subfigure}
  \caption{\label{fig:valzjets}The mean and resolution of $p_T^{rec}/p_T^{tru}$  for the background $Z$+jets sample for $\left|\eta\right|<1.2$ and for $R=0.4$, 0.6, 1.0, and 1.5 in 20 \GeV\ bins of $p_T^{tru}$ for 20--200 \GeV, one bin for 200--300 \GeV, and one bin for 300-500 \GeV, with bins chosen for purposes of statistics.}
  \end{center}
\end{figure}
\clearpage

\section{Truth-Level Analysis}
\label{sec:tru}
To understand the limits and sources of any potential improvements, a truth-level analysis was conducted on MC samples with a $ZH$125 signal sample and a $Z+$jets background sample. Distributions for the dijet invariant mass, $m_{bb}$, were made for each telescoping radius.%\footnote{Distributions for $m_{bb}$ at truth and reconstructed level for all telescoping radii studied may be found in Appendix \ref{sec:mbball}}
  Both signal and background samples develop more pronounced tails in the high $m_{bb}$ region as $R$ increases, as shown in Figure \ref{fig:multradtru}.  $N_{events}$ is normalized to expected values in data.

\begin{figure}[!htbp]\captionsetup{justification=centering}
\begin{center}
\begin{subfigure}[t]{18pc}\centering\includegraphics[width=\textwidth]{figures/body/tru_ov_subel_wz_mbbmain_zh125}\caption{}\end{subfigure}
\begin{subfigure}[t]{18pc}\centering\includegraphics[width=\textwidth]{figures/body/tru_ov_subel_wz_mbbmain_zjets}\caption{}\end{subfigure}
\caption{\label{fig:multradtru}The $m_{bb}$ distribution for the telescoping jets with $R=0.5$, 1.0, and 1.5 truth-level jets is shown for the signal and background samples in (a) and (b), respectively.}
\end{center}
\end{figure}

One way to take advantage of this information is to make a cut on $m_{bb}$ for two different radii.  This is graphically depicted in Figure \ref{fig:2mwtru} for the optimized combination of $m_{bb,R=0.9}$ (telescoping cone jets constructed as outlined in Sec. \ref{sec:cal}) vs. $m_{bb,R=0.4}$ (anti-$k_t$ jets).  At truth-level, the majority of events in the signal $ZH$125 sample are concentrated in relatively narrow region of parameter space, where this is certainly not the case for the more diffuse $Z+$jets background sample.

\begin{figure}[!htbp]\captionsetup{justification=centering}
\begin{center}
\begin{subfigure}[t]{18pc}\centering\includegraphics[width=\textwidth]{figures/body/tru_ov_subel_wz_zh125_mbbRv4_ptall_r9}\caption{}\end{subfigure}
\begin{subfigure}[t]{18pc}\centering\includegraphics[width=\textwidth]{figures/body/tru_ov_subel_wz_zjets_mbbRv4_ptall_r9}\caption{}\end{subfigure}
\caption{\label{fig:2mwtru}The 2D distribution of $m_{bb,R=0.9}$ vs. $m_{bb,R=0.4}$ is shown for signal and background truth-level samples in (a) and (b), respectively.  The region chosen for the double $m_{bb}$ cut is outlined in orange.}
\end{center}
\end{figure}
%\clearpage

Another way to take advantage of multiple event interpretations is to make use of an event's cut-weight, denoted $z$ and defined as the fraction of interpretations in a given event that pass a certain set of cuts (in this note, a cut on $m_{bb}$).  The distribution of cut-weights for a sample of events is denoted $\rho\left(z\right)$.  To enhance the significance of a cut-based analysis, events can be weighted by the cut-weight or any function $t\left(z\right)$ of the cut-weight.  Weighting events by $t\left(z\right)$ modifies the usual $S/\delta B$ formula used to calculate significances.  In this note, $\delta B$ is based on Poissonian statistics and is taken as $0.5+\sqrt{0.25+N_B}$, where $N_B$ is the number of background events.


\section{Errors on Telescoping Significances}
\label{sec:telerr}
Significances of measurements are quoted in units of expected background fluctuations, schematically, $S/\delta B$.  For counting experiments with high numbers of events, we can use Gaussian statistics and express this as $S/\sqrt{B}$, which we here denote as $\mathscr{S}$.  However, with lower statistics, it becomes more appropriate to use Poissonian statistics, and
\begin{equation}
\mathscr{S}_{gaus}=\frac{S}{\sqrt{B}}\to\mathscr{S}_{pois}=\frac{S}{0.5+\sqrt{0.25+B}}
\end{equation}
where $0.5+\sqrt{0.25+B}$ is the characteristic upward fluctuation expected in a Poissonian data set using the Pearson chi-square test\cite{roofit}.

\section{Counting}
The significance is given as above, where $S=N_S$ and $B=N_B$.  That is, the signal and background are just the number of events in signal and background that pass some cuts.  The error for the Guassian case is the standard:
\begin{equation}
\Delta\mathscr{S}_{gaus}=\frac{1}{\sqrt{B}}\Delta S\oplus\frac{S}{2B^{3/2}}\Delta B
\end{equation}
The error for the Poissonian case is:
\begin{equation}
\Delta\mathscr{S}_{pois}=\frac{1}{0.5+\sqrt{0.25+B}}\Delta S\oplus\frac{S}{2\left(0.5+\sqrt{0.25+B}\right)^2\sqrt{0.25+B}}\Delta B
\end{equation}
where $\oplus$ denotes addition in quadrature, and $\Delta S (B)$ is the error on signal (background).

\section{Multiple Event Interpretations}
Using multiple event interpretations changes the formulae used in with simple counting.  That is, $S$ is not necessarily merely $N_S$, the number of events passing some signal cuts, and similarly for $B$ and $N_B$.  Using an event weighting by some function of the cut-weight, $z$, denoted $t\left(z\right)$, $S=N_S\left<t\right>_{\rho_S}$ and $B=N_B\left<t^2\right>_{\rho_B}$. So
\begin{equation}
\mathscr{S}_{t,gaus}=\frac{N_S\left<t\right>_{\rho_S}}{\sqrt{N_B\left<t^2\right>_{\rho_B}}}}\to\mathscr{S}_{t,pois}=\frac{N_S\left<t\right>_{\rho_S}}{0.5+\sqrt{0.25+N_B\left<t^2\right>_{\rho_B}}}=\frac{N_S\int_0^1 dz\,t\left(z\right)\rho_S\left(z\right)}{0.5+\sqrt{0.25+N_B\int_0^1dz\,t^2\left(z\right)\rho_B\left(z\right)}}
\end{equation}

For histograms, everything is done bin-wise.  The notation used below is as follows: $\rho_i$ is the value of $\rho\left(z\right)$ at bin $i$ (where the bins run from 0 to $n_{tel}$, where $n_{tel}$ is the total number of telescoping radii).  $t_i=t_i\left(\rho_{S,i},\rho_{B,i},i/n_{tel}\right)$ is the value of $t\left(z\right)$ at bin $i$, which can depend, in principle, on $\rho_{S,i}$, $\rho_{B,i}$, and $i/n_{tel}$ (the last of which is $z$ in bin $i$).  Explicitly,
\begin{equation*}
N=\sum_{i=0}^{n_{tel}} \rho_i,\;\int_0^1 dz\,t\left(z\right)\rho_S\left(z\right)&=&\sum_{i=0}^{n_{tel}}t_i\rho_{S,i},\;\int_0^1 dz\,t^2\left(z\right)\rho_B\left(z\right)&=&\sum_{i=0}^{n_{tel}}t^2_i\rho_{B,i}
\end{equation*}

For the calculations that follow, let $\xi=\sum_{i=0}^{n_{tel}}t_i\rho_{S,i}$, $\psi=0.5+\sqrt{0.25+N_B\sum_{i=0}^{n_{tel}}t^2_i\rho_{B,i}}$, $\partial_{S}=\frac{\partial}{\partial \rho_{S,i}}$ (and similarly for $B$), so $\mathscr{S}_t=N_S\xi/\psi$

Some partial derivatives:
\begin{eqnarray*}
\partial_SN_S=1,&\,&\partial_{B,i}N_B=1\\
\partial_S\xi=t_i+\left(\partial_S t_i\right)\rho_{S,i},&\,&\partial_B\xi=\left(\partial_B t_i\right)\rho_{B,i}\\
%\end{eqnarray*}
%\begin{eqnarray*}
\partial_S\psi=\frac{N_Bt_i\left(\partial_S t_i\right)\rho_{B,i}}{\sqrt{0.25+N_B\sum_{i=0}^{n_{tel}}t^2_i\rho_{B,i}}},&\,&\partial_B\psi=\frac{\sum_{i=0}^{n_{tel}}t^2_i\rho_{B,i}+N_B\left(t_i^2+2t_i\left(\partial_B t_i\right)\rho_{B,i}\right)}{2\sqrt{0.25+N_B\sum_{i=0}^{n_{tel}}t^2_i\rho_{B,i}}}\\
\partial_S\mathscr{S}_t=\frac{\xi}{\psi}+\frac{N_S}{\psi}\partial_S \xi-\frac{N_S\xi}{\psi^2}\partial_S \psi,&\,&\partial_B\mathscr{S}_t=N_S\left(\frac{1}{\psi}\partial_B \xi-\frac{\xi}{\psi^2}\partial_B \psi\right)
\end{eqnarray*}

Thus,
\begin{equation}
  \Delta\mathscr{S}_{t,i}=\left[\frac{\xi}{\psi}+\frac{N_S}{\psi}\partial_S \xi-\frac{N_S\xi}{\psi^2}\partial_S \psi\right]\Delta\rho_{S,i}\oplus N_S\left[\frac{1}{\psi}\partial_B \xi-\frac{\xi}{\psi^2}\partial_B \psi\right]\Delta\rho_{B,i}
\end{equation}
and the total error is given by the sum in quadrature over all bins $i$ of $\Delta\mathscr{S}_{t,i}$.
\section{$t\left(z\right)=z$}
With $t\left(z\right)=z$, $t_i=i/n_{tel}$, so $\partial_S t_i=\partial_B t_i=0$. So:
\begin{eqnarray*}
  \partial_S \psi&=&\partial_B \xi=0\\
  \partial_S\xi&=&\frac{i}{n_{tel}}\\
  \partial_B\psi&=&\frac{\sum_ii^2\rho_{B,i}+N_Bi^2}{n_{tel}\sqrt{n_{tel}^2+N_B\sum_ii^2\rho_{B,i}}}
\end{eqnarray*}
so  $\Delta\mathscr{S}_{z,i}$ reduces to
\begin{equation}
\Delta\mathscr{S}_{t,i}=\left[\frac{\xi+N_St_i}{\psi}\right]\Delta\rho_{S,i}\oplus\left[\frac{N_S\xi}{\psi^2}\partial_B \psi\right]\Delta\rho_{B,i}
\label{eqn:dsdb}
\end{equation}

\section{$t\left(z\right)=\rho_{S}\left(z\right)/\rho_{B}\left(z\right)$}
With the likelihood optimized\footnote{for the Gaussian statistics case} $t^*\left(z\right)=\rho_S\left(z\right)/\rho_B\left(z\right)$, $t_i=\rho_{S,i}/\rho_{B,i}$, so $\partial_St_i=1/\rho_{B,i}$ and $\partial_B t_i=-\rho_{S,i}/\rho_{B,i}^2$. So:
\begin{eqnarray*}
  \partial_S\xi&=&2\frac{\rho_{S,i}}{\rho_{B,i}}=2t_i\\
  \partial_B\xi&=&-\frac{\rho_{S,i}}{\rho_{B,i}}=-t_i\\
  \partial_S\psi&=&\frac{N_Bt_i}{\sqrt{0.25+N_B\sum_i\rho_{S,i}^2/\rho_{B,i}}}\\
  \partial_B\psi&=&\frac{\sum_i\rho_{S,i}^2/\rho_{B,i}-N_B\left(\rho_{S,i}/\rho_{B,i}\right)^2}{\sqrt{1+4N_B\sum_i\rho_{S,i}^2/\rho_{B,i}}}
\end{eqnarray*}
simplifying somewhat the terms in the per bin error in Equation \ref{eqn:dsdb}.

  The new significance figure using multiple event interpretations becomes, with $\rho_S$ and $\rho_B$ denoting the cut-weight distributions in signal and background, respectively
\begin{equation}
\frac{S}{\delta B}=\frac{N_S\left<t\right>_{\rho_S}}{0.5+\sqrt{0.25+N_B\left<t^2\right>_{\rho_B}}}
\label{eqn:timp}
\end{equation}
Of particular interest is the likelihood optimized $t\left(z\right)$,\footnote{Derived under the assumption of Gaussian statistics in Ref \cite{multint}} $t^*\left(z\right)=\rho_S\left(z\right)/\rho_B\left(z\right)$.  $m_{bb}$ windows are chosen separately for each scheme studied to maximize total significances and are summarized in Table \ref{tab:masswindow}. 
\begin{equation}
\left(\frac{S}{\delta B}\right)_z=\frac{N_S\epsilon_S}{0.5+\sqrt{0.25+N_B\left(\epsilon_B^2+\sigma_B^2}\right)}
\label{eqn:timpz}
\end{equation}
\begin{equation}
\left(\frac{S}{\delta B}\right)_{t^*\left(z\right)}=\frac{N_S\int_0^1 dz\,\frac{\rho_S^2\left(z\right)}{\rho_B\left(z\right)}}{0.5+\sqrt{0.25+N_B\int_0^1dz\,\frac{\rho_S^2\left(z\right)}{\rho_B\left(z\right)}}}
\label{eqn:timptstar}
\end{equation}
where $\epsilon_{S,B}$ are the means of $\rho_{S,B}\left(z\right)$ and $\sigma_B^2$ is the variance of $\rho_B\left(z\right)$.  Further details can be found in Refs. \cite{teljet,multint} and Appendix \ref{sec:telerr}.

\begin{table}[htbp]
\caption{$m_{bb}$ windows studied.  These windows were chosen to optimize significances over all $p_T^Z$.
\label{tab:masswindow}}
\begin{center}
\begin{tabular}{|l|l|l|}
\hline
Analysis Type & $S/\delta B$ Type                                           & Optimal $m_{bb}$ Window\\
\hline
Reconstructed & anti-$k_t$ $R=0.4$                                          &  90--140 \GeV\ \\
              & $t\left(z\right)=z$                                         & 110--155 \GeV\ \\
              & $t\left(z\right)=\rho_S\left(z\right)/\rho_B\left(z\right)$ & 110--155 \GeV\ \\
              & anti-$k_t$ $R=0.4$, telescoping $R=0.6$                     &  95--140 \GeV\ ($R=0.4$),  105--160 \GeV\ ($R=0.6$)\\
\hline
Truth         & anti-$k_t$ $R=0.4$                                          & 100--130 \GeV\ \\
              & $t\left(z\right)=z$                                         & 115--140 \GeV\ \\
              & $t\left(z\right)=\rho_S\left(z\right)/\rho_B\left(z\right)$ & 120--135 \GeV\ \\
              & anti-$k_t$ $R=0.4$, telescoping $R=0.9$                     & 100--130 \GeV\ ($R=0.4$), 100--155 \GeV\ ($R=0.9$)\\
\hline
\end{tabular}
\end{center}
\end{table}

\begin{figure}[!htbp]\captionsetup{justification=centering}
\begin{center}
\begin{subfigure}[t]{18pc}\centering\includegraphics[width=\textwidth]{figures/body/tru_ov_subel_wz_ntel11_zh125_rho_ptall_lo120_hi135}\caption{}\end{subfigure}
\begin{subfigure}[t]{18pc}\centering\includegraphics[width=\textwidth]{figures/body/tru_ov_subel_wz_ntel11_zjets_rho_ptall_lo120_hi135}\caption{}\end{subfigure}
\begin{subfigure}[t]{18pc}\centering\includegraphics[width=\textwidth]{figures/body/tru_ov_subel_wz_ntel11_zjets_rhoSB_ptall_lo120_hi135}\caption{}\end{subfigure}
\caption{\label{fig:rhotru}Truth-level $\rho\left(z\right)$ distributions for the $m_{bb}$ window optimizing $\left(S/\delta B\right)_{t^*\left(z\right)}$.  $\rho_S\left(z\right)$ for the signal $ZH125$ sample is shown in (a), and $\rho_B\left(z\right)$ for the background $Z+$jets sample is shown in (b).  The distribution of $\rho_S\left(z\right)/\rho_B\left(z\right)$ for these samples is shown in (c).}
\end{center}
\end{figure}

The truth-level distributions $\rho_S\left(z\right)$, $\rho_B\left(z\right)$, and $\rho_S\left(z\right)/\rho_B\left(z\right)$ are shown for the $m_{bb}$ window that optimizes $(S/\delta B)_{t^*\left(z\right)}$ in Figure \ref{fig:rhotru}, and significance improvements as a function of $p_T^Z$ are summarized in Figure \ref{fig:sigsumtru}.  Uncertainties in Figures \ref{fig:sigsumtru} and \ref{fig:sigsumrec} are statistical uncertainties.  JES systematics will need to be evaluated for different $R$'s, as modeling uncertainties is an outstanding issue, but these systematics will likely be strongly correlated for the different $R$'s and are not anticipated to be a very large contribution to total uncertainties.  While the two dimensional $m_{bb}$ cut and $t\left(z\right)=z$ schemes only showed marginal improvement at truth level at 2.87\%\footnote{The limited improvement is provably due to the simplified treatment of the 2D case; better performance with a more sophisticated treatment has been observed in Ref. \cite{newschwartz}.}  and 1.45\%, respectively, the likelihood optimized $t^*\left(z\right)$ showed a more substantial 40.7\% improvement overall, with a steady increase in improvement with increasing $p_T^Z$.  Figure \ref{fig:sigsumtru} (d) summarizes the improvements with respect to $p_T^Z$ for the $t^*\left(z\right)$ event weight for five, seven, and twelve telescoping radii (interpretations) per event.  Improvements increase with a greater number of interpretations and are more pronounced at higher $p_T^Z$ for this scheme.%\footnote{This is not the case for the $z$ event weight, which is more thoroughly treated in Section \ref{sec:multrad}.}
  The optimal $120<m_{bb}<135$ \GeV\ window for $t^*\left(z\right)$ case is among the smallest studied.  The benefits of this window's narrowness are suggested in Figure \ref{fig:rhotru}.  While the background cut-weight distribution, $\rho_B\left(z\right)$ in Figure \ref{fig:rhotru} (b) behaves as one might with a marked peak at $z=0$, the signal $\rho_S\left(z\right)$ distribution peaks at a relatively modest $z=0.3$, which indicates that much of the gain at truth level comes from background rejection.  This is possible at truth level since there is both truth-level information available and no smearing and since $\rho_S/\rho_B$ is the relevant quantity (as shown in Figure \ref{fig:rhotru} (c)).

\begin{figure}[!htbp]\captionsetup{justification=centering}
\begin{center}
\begin{subfigure}[t]{16pc}\centering\includegraphics[width=\textwidth]{figures/body/tru_ov_subel_wz_ntel11_pois_sdb_r4_lo100_hi130_r9_lo100_hi155_zjets}\caption{}\end{subfigure}
\begin{subfigure}[t]{16pc}\centering\includegraphics[width=\textwidth]{figures/body/tru_ov_subel_wz_ntel11_pois_sdb_z_lo115_hi140_zjets}\caption{}\end{subfigure}
\begin{subfigure}[t]{16pc}\centering\includegraphics[width=\textwidth]{figures/body/tru_ov_subel_wz_ntel11_pois_sdb_rho_lo120_hi135_zjets}\caption{}\end{subfigure}
\begin{subfigure}[t]{16pc}\centering\includegraphics[width=\textwidth]{figures/zht_comp/zht_comp_pois_sdb_tru_ov_subel_wz_imp_rho}\caption{}\end{subfigure}
\caption{\label{fig:sigsumtru}A summary of the improvements for different truth-level telescoping jet cuts and weights is shown in bins of $p_T^Z$.  %These bins are the same as for the topological cuts: 0--90 \GeV, 90--120 \GeV, 120--160 \GeV, 160--200 \GeV, and $>200$ \GeV.
  The final bin is the total improvement over all $p_T^Z$.  Shown are improvements for the 2D $m_{bb}$ cut (a), $t\left(z\right)=z$ (b), $t\left(z\right)=t^*\left(z\right)$ with 12 radii (c), and $t\left(z\right)=t^*\left(z\right)$ for various radii (d).}
\end{center}
\end{figure}

\clearpage

\section{Reconstructed-Level Analysis}
\label{sec:rec}
At reconstructed level, the same overall effect of introducing a high tail in $m_{bb}$ distributions with increasing $R$ is evident in comparing Figures \ref{fig:multradtru} and \ref{fig:multradrec}.  The optimal $m_{bb}$ windows, however, grow larger, due to the lack of truth-level information.

\begin{figure}[!htbp]\captionsetup{justification=centering}
\begin{center}
\begin{subfigure}[t]{18pc}\centering\includegraphics[width=\textwidth]{figures/body/rec_ov_subel_trim_mbbmain_zh125}\caption{}\end{subfigure}
\begin{subfigure}[t]{18pc}\centering\includegraphics[width=\textwidth]{figures/body/rec_ov_subel_trim_mbbmain_zjets}\caption{}\end{subfigure}
\caption{\label{fig:multradrec}The $m_{bb}$ distribution for the telescoping jets with $R=0.5$, $R=1.0$, and $R=1.5$ reconstructed-level jets is shown for the signal and background samples in (a) and (b), respectively.}
\end{center}
\end{figure}

Total significance gains at reconstructed level for the two dimensional $m_{bb}$ cut and the $t\left(z\right)=z$ case are similar, at 2.87\% and 1.45\%,respectively.  The optimal two-dimensional $m_{bb}$ cut at reconstructed level is $95<m_{bb,R=0.4}<140$ \GeV, $105<m_{bb,R=0.6}<160$ \GeV.  Just as at truth level, the $R=0.4$ $m_{bb}$ cut is comparable to the optimal single $R=0.4$ $m_{bb}$ cut, and the second $m_{bb}$ cut is at similar values (cf. Table \ref{tab:masswindow} and Figures \ref{fig:2mwtru} and \ref{fig:2mwrec}).  However, the optimal second telescoping radius is markedly smaller at $R=0.6$ versus the optimal truth-level second radius of $R=0.9$, which suggests that effects like pileup at reconstructed level obscure correlations between the $R=0.4$ interpretations and limit the usefulness of larger $R$ interpretations in this particular scheme. The $t\left(z\right)=z$ case has a wider optimal window and yields about half the improvement it does at truth level.%\footnote{A fuller treatment of this scheme is given in Section \ref{sec:multrad}.}

\begin{figure}[!htbp]\captionsetup{justification=centering}
\begin{center}
\begin{subfigure}[t]{18pc}\centering\includegraphics[width=\textwidth]{figures/body/rec_ov_subel_trim_zh125_mbbRv4_ptall_r6}\caption{}\end{subfigure}
\begin{subfigure}[t]{18pc}\centering\includegraphics[width=\textwidth]{figures/body/rec_ov_subel_trim_zjets_mbbRv4_ptall_r6}\caption{}\end{subfigure}
\caption{\label{fig:2mwrec}The 2D distribution of $m_{bb,R=0.8}$ vs. $m_{bb,R=0.4}$ is shown for signal and background reconstructed-level samples in (a) and (b), respectively.  The region chosen for the double $m_{bb}$ cut is outlined in orange.}
\end{center}
\end{figure}

\begin{figure}[!htbp]\captionsetup{justification=centering}
\begin{center}
\begin{subfigure}[t]{18pc}\centering\includegraphics[width=\textwidth]{figures/body/rec_ov_subel_trim_ntel11_zh125_rho_ptall_lo110_hi155}\caption{}\end{subfigure}
\begin{subfigure}[t]{18pc}\centering\includegraphics[width=\textwidth]{figures/body/rec_ov_subel_trim_ntel11_zjets_rho_ptall_lo110_hi155}\caption{}\end{subfigure}
\begin{subfigure}[t]{18pc}\centering\includegraphics[width=\textwidth]{figures/body/rec_ov_subel_trim_ntel11_zjets_rhoSB_ptall_lo110_hi155}\caption{}\end{subfigure}
\caption{\label{fig:rhorec}Reconstructed-level $\rho\left(z\right)$ distributions for the $m_{bb}$ window optimizing $\left(S/\delta B\right)_{t^*\left(z\right)}$.  $\rho_S\left(z\right)$ for the signal $ZH125$ sample is shown in (a), and $\rho_B\left(z\right)$ for the background $Z+$jets sample is shown in (b).  The distribution of $\rho_S\left(z\right)/\rho_B\left(z\right)$ for these samples is shown in (c).}
\end{center}
\end{figure}

The optimal $m_{bb}$ window for the $t^*\left(z\right)$ case is also markedly wider at reconstructed level, at $110<m_{bb}<155$ \GeV\ in comparison to the truth-level optimal $120<m_{bb}<135$ \GeV.  The $\rho\left(z\right)$ distributions for the signal $ZH125$ and background $Z+$jets as well as the $\rho_S\left(z\right)/\rho_B\left(z\right)$ in this window are shown in Figure \ref{fig:rhorec}.  Compared with the truth-level distributions in Figure \ref{fig:rhotru}, both the signal and background optimal $\rho\left(z\right)$ distributions have higher values at higher $z$.  The peak in $\rho_S\left(z\right)$ at $z=1$ suggests that at reconstructed level, maximizing the number of more ``signal-like'' events is the key to optimizing significances, as opposed to the optimal, background suppressing  $\rho\left(z\right)$ distributions at truth level.  The use of a greater number of interpretations per event (telescoping radii) does appear to result in overall greater improvement as at truth level, as twelve radii performed better than five, but this is less clear at reconstructed level, as shown in Figure \ref{fig:sigsumrec} (d).  The improvement at reconstructed level using an event weight of $t^*\left(z\right)$ is 20.5\%, just over half the improvement at truth level but still quite significant.  Summaries of improvements as a function of $p_T^Z$ for all three cases studied and for the $t^*\left(z\right)$ case for different numbers of telescoping radii are shown in Figure \ref{fig:sigsumrec}.

\begin{figure}[!htbp]\captionsetup{justification=centering}
\begin{center}
\begin{subfigure}[t]{16pc}\centering\includegraphics[width=\textwidth]{figures/body/rec_ov_subel_trim_ntel11_pois_sdb_r4_lo95_hi140_r6_lo105_hi160_zjets}\caption{}\end{subfigure}
\begin{subfigure}[t]{16pc}\centering\includegraphics[width=\textwidth]{figures/body/rec_ov_subel_trim_ntel11_pois_sdb_z_lo110_hi155_zjets}\caption{}\end{subfigure}
\begin{subfigure}[t]{16pc}\centering\includegraphics[width=\textwidth]{figures/body/rec_ov_subel_trim_ntel11_pois_sdb_rho_lo110_hi155_zjets}\caption{}\end{subfigure}
\begin{subfigure}[t]{16pc}\centering\includegraphics[width=\textwidth]{figures/zht_comp/zht_comp_pois_sdb_rec_ov_subel_trim_imp_rho}\caption{}\end{subfigure}
\caption{\label{fig:sigsumrec}A summary of the improvements for different reconstructed-level telescoping jet cuts and weights is shown in bins of $p_T^Z$.%  These bins are the same as for the topological cuts: 0--90 \GeV, 90--120 \GeV, 120--160 \GeV, 160--200 \GeV, and $>200$ \GeV.
  The final bin is the total improvement over all $p_T^Z$.  Shown are improvements for the 2D $m_{bb}$ cut (a), $t\left(z\right)=z$ (b), $t\left(z\right)=t^*\left(z\right)$ with 12 radii (c), and $t\left(z\right)=t^*\left(z\right)$ for various radii (d).}
\end{center}
\end{figure}

\begin{table}[htbp]
\caption{A summary of significances for different weighting schemes and cuts and for reconstructed and truth jets for a luminosity of $20.3\text{ fb}^{-1}.$
\label{tab:SdB}}
\begin{center}
\begin{tabular}{|l|c|c|c|c|c|c|}
\hline
Type                       & 0--90 \GeV\ & 90--120 \GeV\ & 120--160 \GeV\ & 160--200 \GeV\ & $>200$ \GeV\ & total\\
\hline
anti-$k_t$, $R=0.4_{rec}$  &    0.47492  &    0.28214    &    0.28339     &    0.25748     &    0.37337   &    0.76887\\
anti-$k_t$, $R=0.4_{tru}$  &    0.57414  &    0.30655    &    0.37309     &    0.35042     &    0.53569   &    0.98619\\
\hline
2 $m_{bb,rec}$             &    0.48903  &     0.2858    &    0.28812     &    0.25972     &    0.38297   &    0.78611\\
2 $m_{bb,tru}$             &     0.5724  &     0.3191    &    0.38364     &    0.36655     &    0.56414   &     1.0145\\
\hline
$z_{rec}$                  &    0.50698  &    0.27962    &    0.29937     &    0.25688     &    0.36846   &    0.79158\\
$z_{tru}$                  &    0.56894  &    0.31511    &    0.39065     &    0.36277     &    0.54206   &     1.0005\\
\hline
$t^*\left(z\right)_{rec}$  &    0.55085 &    0.29931     &    0.33367     &    0.30107     &    0.51321   &    0.92649\\
$t^*\left(z\right)_{tru}$  &    0.64425 &    0.38008     &    0.50904     &     0.5214     &    0.91337   &     1.3873\\
\hline
\end{tabular}
\end{center}
\end{table}

\clearpage
\section{Conclusions and Prospects}
\label{sec:con}
The use of telescoping jets to provide multiple event interpretations shows promise as an avenue to increase significances in the $H\to b\bar{b}$ search in ATLAS and make an observation in the systematics-limited environment of early Run 2.  A preliminary study using the telescoping jets algorithm with 12 telescoping radii to build 12 event interpretations on 2012 Monte Carlo based on the full, cut-based Run 1 analysis yielded a 20.5\% improvement in $S/\delta B$ over using anti-$k_t$ with $R=0.4$ alone at reconstructed level using a likelihood maximized event weighting to study the $ZH\to ll b\bar{b}$ process.  The jets used in this note at reconstructed level were trimmed in order to guarantee reasonable resolution in the large-$R$ interpretations.  The algorithm, in particular, showed discriminating power at high $p_T^Z$, so better performance can be expected in Run 2 with a higher $\sqrt{s}$ and higher numbers of events with large $p_T^Z$.  Additionally, the many simplifying assumptions regarding jet calibration and the relatively basic use of information\footnote{For examples of more sophistocated treatments compared to the treatment in this note, see Ref \cite{newschwartz}.} from multiple invariant masses in this note suggest that even further improvements than those quoted are possible.  While this note did not explore the correlations between multiple event interpretations and the variables used in the BDT of the latest multivariate version of the $H\to b\bar{b}$ analysis\cite{run1note}, new phenomenological studies suggest that such correlations are not strong\cite{newschwartz}.  The corresponding reconstructed-level study, using a BDT, is left for future work.  Also left for future work are better understanding the effects of jet trimming and which interpretations are the most useful.  

\begin{comment}
\begin{thebibliography}{12}

\bibitem{tjconf5}
ATLAS Collaboration, \emph{Observation of a new particle in the search of the Standard Model Higgs boson with the ATLAS deteector at the LHC}, Phys.Lett. \textbf{B716} (2012) 1--29, \href{http://arxiv.org/abs/1207.7214}{\texttt{arXiv:1207.7214 [hep-ex]}}.

\bibitem{tjconf6}
CMS Collaboration, \emph{Observation of a new boson at a mass of 125 \GeV\ with the CMS experiment at the LHC}, Phys.Lett. \textbf{B716} (2012) 30--61, \href{http://arxiv.org/abs/1207.7235}{\texttt{arXiv:1207.7235 [hep-ex]}}.

\bibitem{newresult}
ATLAS Collaboration, \emph{Search for the bb decay of the Standard Model Higgs boson inassociated ($W/Z$)$H$ production with the ATLAS detector}, \href{http://arxiv.org/abs/1409.6212}{\texttt{arXiv:1409.6212 [hep-ph]}} (2014).

\bibitem{tjconf}
ATLAS Collaboration, \emph{Search for the $b\bar{b}$ decay of the Standard Model Higgs boson in associated ($W/Z$)$H$ production with the ATLAS detector}, ATLAS-CONF-2013-079 (2013), \url{https://cdsweb.cern.ch/record/1563235}.

\bibitem{teljet}
Y. Chien, \emph{Telescoping Jets: Multiple Event Interpretations with Multiple $R$'s}, \href{http://arxiv.org/abs/1304.5240}{\texttt{arXiv:1304.5240 [hep-ph]}} (2013).

\bibitem{qjet}
S. D. Ellis, A. Hornig, T. S. Roy, D. Krohn and M. D. Schwartz, \emph{Qjets: A Non-Deterministic Approach to Tree-Based Jet Substructure}, Phys. Rev. Lett. \textbf{108}, 182003 (2012) \href{http://arxiv.org/abs/1201.1914}{\texttt{arXiv:1201.1914 [hep-ph]}}.

\bibitem{kt}
S. Catani, Y. L. Dokshitzer, M. H. Seymour, and B. R. Webber, \emph{Longitudinally invariant $K\left(t\right)$ clustering algorithms for hadron hadron colliders}, Nucl. Phys. B \textbf{406}, 187 (1993).

\bibitem{ca}
Y. L. Dokshitzer, G. D. Leder, S. Moretti and B. R. Webber, \emph{Better jet clustering algorithms}, JHEP \textbf{9708} (1997) 001, \href{http://arxiv.org/abs/hep-ph/9707323}{\texttt{arXiv:hep-ph/9707323 [hep-ph]}}.

\bibitem{akt}
M. Cacciari, G. P. Salam, and G. Soyez, \emph{The anti-$k_t$ jet clustering algorithm}, JHEP \textbf{0804} (2008) 063, \href{http://arxiv.org/abs/0802.1189}{\texttt{arXiv:0802.1189 [hep-ph]}}.

\bibitem{multint}
D. Kahawala, D. Krohn, and M. D. Schwartz, \emph{Jet Sampling: Improving Event Reconstruction through Multiple Interpretations}, \href{http://arxiv.org/abs/1304.2394}{\texttt{arXiv:1304.2394 [hep-ph]}} (2013).

\bibitem{trimjet}
D. Krohn, J. Thaler, L. Wang, \emph{Jet Trimming}, \href{http://arxiv.org/abs/0912.1342}{\texttt{arXiv:0912.1342 [hep-ph]}} (2009).

\bibitem{tjconf18}
T. Sj\"ostrand, S. Mrenna, and P. Z. Sands, \emph{A Brief Introduction to PYTHIA 8.1}, Comput. Phys. Commun. \textbf{178} (2008) 852--867, \href{http://arxiv.org/abs/0710.3820}{\texttt{arXiv:0710.3820 [hep-ph]}}.

\bibitem{conf19}
J. Pumplin, D. Stum, J. Huston, H. Lai, Pm M. Nadolsky, et al., \emph{New generation of parton distributions with unvertainties from global QCD analysis}, JEGP\textbf{0207} (2002) 012, \href{http://arxiv.org/abs/hep-ph/0201195}{\texttt{arXiv:hep-ph/0201195 [hep-ph]}}.

\bibitem{conf20}
ATLAS Collaboation, \emph{ATLAS tunes of PYTHIA6 and PYTHIA8 for MC11}, ATL-PHYS-PUB-2011-009 (2011), \href{http://cdsweb.cern.ch/record/1363300}.

\bibitem{conf21}
ATLAS Collaboration, \emph{New ATLAS event generator tunes to 2010 data}, ATL-PHYS-PUB-2011-008 (201), \href{https://cdsweb.cern.ch/record/1345343}.

\bibitem{conf28}
T. Gleisberg et al., \emph{Event generation with SHERPA 1.1}, JHEP \textbf{02} (2009) 007, \href{http://arxiv.org/abs/0811.4622}{\texttt{arXiv:0811.4622}}.

\bibitem{conf30}
P. Nason, \emph{A new method for combining NLO QCD with shower Monte Carlo algorithms}, JHEP \textbf{0411} (2004) 040, \newblock \href{http://arxiv.org/abs/hep-ph/0409146}{\texttt{arXiv:hep-ph/0409146}}.

\bibitem{conf31}
S. Frixione, P. Nason, and C. Oleari, \emph{Matching NLO QCD computations with Parton Shower simulations: the POWHEG method}, JHEP \textbf{0711} (2007) 070, \href{http://arxiv.org/abs/0709.2092}{\texttt{arXiv:0709.2092 [hep-ph]}}.

\bibitem{conf32}
S. Alioli, P. Nason, C. Oleari, and E. Re, \emph{A general framework for implementing NLO calculations in shower Monte Carlo programs: the POWHEG BOX}, JHEP \textbf{1006} (2010) 043, \href{http://arxiv.org/abs/1002.2581}{\texttt{arXiv:1002.2581 [hep-ph]}}.

\bibitem{conf40}
G. Corcella et al., \emph{HERWIG 6: an event generator for hadron emissionreactions with interfering gluons (includingsuper-symmetric processes)}, HEP \textbf{0101} (2001) 010.

\bibitem{conf50}
ATLAS Collaboration, \emph{Electron performance measurements with the ATLAS detector using the 2010 LHC proton-proton collision data}, Eur. Phys. J. \textbf{C72} (2012) 1909, \href{http://arxiv.org/abs/1110.3174v2}{\texttt{arXiv:1110.3714v2 [hep-ex]}}.

\bibitem{conf51}
ATLAS Collaboration, \emph{Measurement of the $W\to l\nu$ and $Z/\gamma^*\to ll$ production corss sections in proton-proton collisions at $\sqrt{s}=7$ \TeV\ with the ATLAS detector}, JHEP \textbf{1012} (2010) 060, \href{http://arxiv.org/abs/1010.2130}{\texttt{arXiv:1010.2130 [hep-ex]}}.

\bibitem{conf54}
M. Cacciari and G. P. Salam, \emph{Pileup subtraction using jet areas}, Phys. Lett. \textbf{B659} (2008) 119--126, \href{http://arxiv.org/abs/0707.1378}{\texttt{arXiv:0707.1378 [hep-ph]}}

\bibitem{conf55}
ATLAS Collaboration, \emph{Jet energy measurement with the ATLAS detector in proton-proton collisions at $\sqrt{s}=7$ \TeV\ }, Eur. Phys. J. \textbf{C73} (2013) 2304, \href{http://arxiv.org/abs/1112.6426}{\texttt{arXiv:1112.6426 [hep-ex]}}.

\bibitem{conf56}
Atlas Collaboration, \emph{b-jet tagging calibration on c-jets containing D$^{*+}$ mesons}, ATLAS-CONF-2012-039 (2012), \url{https://cdsweb.cern.ch/record/1435193}.

\bibitem{conf57}
Atlas Collaboration, \emph{Measurement of the Mistag Rate with 5 fb$^{-1}$ of Data Collected by the ATLAS Detector}, ATLAS-CONF-2012-040 (2012), \url{https://cdsweb.cern.ch/record/1435194}.

\bibitem{conf58}
Atlas Collaboration, \emph{Measurement of the b-tag Efficiency in a Sample of Jets Containing Muons with 5 fb$^{-1}$ of Data Collected by the ATLAS Detector}, ATLAS-CONF-2012-043 (2012), \url{https://cdsweb.cern.ch/record/1435197}.

\bibitem{conf59}
Atlas Collaboration, \emph{Commissioning of the ATLAS high-performance b-tagging algorithms in 7 \TeV\ collision data}, ATLAS-CONF-2011-102 (2011), \url{https://cdsweb.cern.ch/record/1369219}.

\bibitem{conf60}
Atlas Collaboration, \emph{Measuring the b-tag efficiency in a top-pair sample with 4.7 fb$^{-1}$ of Data Collected by the ATLAS Detector}, ATLAS-CONF-2012-097 (2012), \url{https://cdsweb.cern.ch/record/1460443}.

\bibitem{newschwartz}
Y. Chien et al., \emph{Quantifying the power of multiple event interpretations}, \href{http://arxiv.org/abs/1407.2892}{\texttt{arXiv:1407.2892} [hep-ph]}

\bibitem{roofit}
W. Verkerke, D. P. Kirkby, \emph{The RooFit toolkit for data modeling}, Proceedings for 
CHEP03 (2003) \href{http://arxiv.org/abs/physics/0306116}{\texttt{arXiv:physics/0306116}}

\end{thebibliography}
\clearpage
\section{Errors on Telescoping Significances}
\label{sec:telerr}
Significances of measurements are quoted in units of expected background fluctuations, schematically, $S/\delta B$.  For counting experiments with high numbers of events, we can use Gaussian statistics and express this as $S/\sqrt{B}$, which we here denote as $\mathscr{S}$.  However, with lower statistics, it becomes more appropriate to use Poissonian statistics, and
\begin{equation}
\mathscr{S}_{gaus}=\frac{S}{\sqrt{B}}\to\mathscr{S}_{pois}=\frac{S}{0.5+\sqrt{0.25+B}}
\end{equation}
where $0.5+\sqrt{0.25+B}$ is the characteristic upward fluctuation expected in a Poissonian data set using the Pearson chi-square test\cite{roofit}.

\section{Counting}
The significance is given as above, where $S=N_S$ and $B=N_B$.  That is, the signal and background are just the number of events in signal and background that pass some cuts.  The error for the Guassian case is the standard:
\begin{equation}
\Delta\mathscr{S}_{gaus}=\frac{1}{\sqrt{B}}\Delta S\oplus\frac{S}{2B^{3/2}}\Delta B
\end{equation}
The error for the Poissonian case is:
\begin{equation}
\Delta\mathscr{S}_{pois}=\frac{1}{0.5+\sqrt{0.25+B}}\Delta S\oplus\frac{S}{2\left(0.5+\sqrt{0.25+B}\right)^2\sqrt{0.25+B}}\Delta B
\end{equation}
where $\oplus$ denotes addition in quadrature, and $\Delta S (B)$ is the error on signal (background).

\section{Multiple Event Interpretations}
Using multiple event interpretations changes the formulae used in with simple counting.  That is, $S$ is not necessarily merely $N_S$, the number of events passing some signal cuts, and similarly for $B$ and $N_B$.  Using an event weighting by some function of the cut-weight, $z$, denoted $t\left(z\right)$, $S=N_S\left<t\right>_{\rho_S}$ and $B=N_B\left<t^2\right>_{\rho_B}$. So
\begin{equation}
\mathscr{S}_{t,gaus}=\frac{N_S\left<t\right>_{\rho_S}}{\sqrt{N_B\left<t^2\right>_{\rho_B}}}}\to\mathscr{S}_{t,pois}=\frac{N_S\left<t\right>_{\rho_S}}{0.5+\sqrt{0.25+N_B\left<t^2\right>_{\rho_B}}}=\frac{N_S\int_0^1 dz\,t\left(z\right)\rho_S\left(z\right)}{0.5+\sqrt{0.25+N_B\int_0^1dz\,t^2\left(z\right)\rho_B\left(z\right)}}
\end{equation}

For histograms, everything is done bin-wise.  The notation used below is as follows: $\rho_i$ is the value of $\rho\left(z\right)$ at bin $i$ (where the bins run from 0 to $n_{tel}$, where $n_{tel}$ is the total number of telescoping radii).  $t_i=t_i\left(\rho_{S,i},\rho_{B,i},i/n_{tel}\right)$ is the value of $t\left(z\right)$ at bin $i$, which can depend, in principle, on $\rho_{S,i}$, $\rho_{B,i}$, and $i/n_{tel}$ (the last of which is $z$ in bin $i$).  Explicitly,
\begin{equation*}
N=\sum_{i=0}^{n_{tel}} \rho_i,\;\int_0^1 dz\,t\left(z\right)\rho_S\left(z\right)&=&\sum_{i=0}^{n_{tel}}t_i\rho_{S,i},\;\int_0^1 dz\,t^2\left(z\right)\rho_B\left(z\right)&=&\sum_{i=0}^{n_{tel}}t^2_i\rho_{B,i}
\end{equation*}

For the calculations that follow, let $\xi=\sum_{i=0}^{n_{tel}}t_i\rho_{S,i}$, $\psi=0.5+\sqrt{0.25+N_B\sum_{i=0}^{n_{tel}}t^2_i\rho_{B,i}}$, $\partial_{S}=\frac{\partial}{\partial \rho_{S,i}}$ (and similarly for $B$), so $\mathscr{S}_t=N_S\xi/\psi$

Some partial derivatives:
\begin{eqnarray*}
\partial_SN_S=1,&\,&\partial_{B,i}N_B=1\\
\partial_S\xi=t_i+\left(\partial_S t_i\right)\rho_{S,i},&\,&\partial_B\xi=\left(\partial_B t_i\right)\rho_{B,i}\\
%\end{eqnarray*}
%\begin{eqnarray*}
\partial_S\psi=\frac{N_Bt_i\left(\partial_S t_i\right)\rho_{B,i}}{\sqrt{0.25+N_B\sum_{i=0}^{n_{tel}}t^2_i\rho_{B,i}}},&\,&\partial_B\psi=\frac{\sum_{i=0}^{n_{tel}}t^2_i\rho_{B,i}+N_B\left(t_i^2+2t_i\left(\partial_B t_i\right)\rho_{B,i}\right)}{2\sqrt{0.25+N_B\sum_{i=0}^{n_{tel}}t^2_i\rho_{B,i}}}\\
\partial_S\mathscr{S}_t=\frac{\xi}{\psi}+\frac{N_S}{\psi}\partial_S \xi-\frac{N_S\xi}{\psi^2}\partial_S \psi,&\,&\partial_B\mathscr{S}_t=N_S\left(\frac{1}{\psi}\partial_B \xi-\frac{\xi}{\psi^2}\partial_B \psi\right)
\end{eqnarray*}

Thus,
\begin{equation}
  \Delta\mathscr{S}_{t,i}=\left[\frac{\xi}{\psi}+\frac{N_S}{\psi}\partial_S \xi-\frac{N_S\xi}{\psi^2}\partial_S \psi\right]\Delta\rho_{S,i}\oplus N_S\left[\frac{1}{\psi}\partial_B \xi-\frac{\xi}{\psi^2}\partial_B \psi\right]\Delta\rho_{B,i}
\end{equation}
and the total error is given by the sum in quadrature over all bins $i$ of $\Delta\mathscr{S}_{t,i}$.
\section{$t\left(z\right)=z$}
With $t\left(z\right)=z$, $t_i=i/n_{tel}$, so $\partial_S t_i=\partial_B t_i=0$. So:
\begin{eqnarray*}
  \partial_S \psi&=&\partial_B \xi=0\\
  \partial_S\xi&=&\frac{i}{n_{tel}}\\
  \partial_B\psi&=&\frac{\sum_ii^2\rho_{B,i}+N_Bi^2}{n_{tel}\sqrt{n_{tel}^2+N_B\sum_ii^2\rho_{B,i}}}
\end{eqnarray*}
so  $\Delta\mathscr{S}_{z,i}$ reduces to
\begin{equation}
\Delta\mathscr{S}_{t,i}=\left[\frac{\xi+N_St_i}{\psi}\right]\Delta\rho_{S,i}\oplus\left[\frac{N_S\xi}{\psi^2}\partial_B \psi\right]\Delta\rho_{B,i}
\label{eqn:dsdb}
\end{equation}

\section{$t\left(z\right)=\rho_{S}\left(z\right)/\rho_{B}\left(z\right)$}
With the likelihood optimized\footnote{for the Gaussian statistics case} $t^*\left(z\right)=\rho_S\left(z\right)/\rho_B\left(z\right)$, $t_i=\rho_{S,i}/\rho_{B,i}$, so $\partial_St_i=1/\rho_{B,i}$ and $\partial_B t_i=-\rho_{S,i}/\rho_{B,i}^2$. So:
\begin{eqnarray*}
  \partial_S\xi&=&2\frac{\rho_{S,i}}{\rho_{B,i}}=2t_i\\
  \partial_B\xi&=&-\frac{\rho_{S,i}}{\rho_{B,i}}=-t_i\\
  \partial_S\psi&=&\frac{N_Bt_i}{\sqrt{0.25+N_B\sum_i\rho_{S,i}^2/\rho_{B,i}}}\\
  \partial_B\psi&=&\frac{\sum_i\rho_{S,i}^2/\rho_{B,i}-N_B\left(\rho_{S,i}/\rho_{B,i}\right)^2}{\sqrt{1+4N_B\sum_i\rho_{S,i}^2/\rho_{B,i}}}
\end{eqnarray*}
simplifying somewhat the terms in the per bin error in Equation \ref{eqn:dsdb}.

\clearpage

\section{Different Numbers of Telescoping Radii}
\label{sec:multrad}
Additional studies were performed to examine the behavior of $\rho\left(z\right)$ and significance improvements with different numbers of telescoping radii.  The five ($R=0.4-0.8$) and seven ($R=0.4-1.0$) radii cases are examined in addition to the 12 radii case used in the above analysis.  In all cases, the original anti-$k_t$ with $R=0.4$ jets serve as the interpretation for $R=0.4$, and interpretations are incremented by jet cone size steps of 0.1.  $\rho\left(z\right)$ distributions are shown in Figure \ref{fig:comprhozh125tru}-\ref{fig:comprhozjetsrec}, and significances are shown in Figure \ref{fig:compsdb}.

While $\rho\left(z\right)$ distributions do shift towards higher $z$ for higher $p_T^Z$, this effect is much more pronounced in the signal $ZH125$ distributions, in particular at reconstructed level.  The reconstructed-level signal $ZH125$ samples show a marked progression from a relatively flat distribution over $z$ at low $p_T^Z$ to the pronounced peak at $z=1$ one would expect from the total $\rho_S\left(z\right)$ distribution in Figures \ref{fig:rhorec} and \ref{fig:comprhozh125rec} (f).  This agrees with the observation that a large portion of improvements using the $t^*\left(z\right)$ event weight come at high $p_T^Z$, as the background $Z+$jets $\rho\left(z\right)$ distributions shift much less.

The relative shapes of the $\rho\left(z\right)$ distributions with different numbers of interpretations per event also suggest an explanation for the relatively poor performance of the $z$ event weighting.  At truth level, the $\rho_S\left(z\right)$ distribution has a peak at increasingly lower $z$ for increasing number of telescoping radii while the background shape is relatively stable.  Hence, while the likelihood optimized $t^*\left(z\right)$ appears effective for both signal recognition and background rejection, the $z$ event weight appears only suited to the former task; it also suggests that larger $R$ interpretations are also only particularly helpful for background rejection.  A summary of significances as a function of $p_T^Z$ for the standard $m_{bb}$ cut with anti-$k_t$ with $R=0.4$ jets and improvements for the $z$ and $t^*\left(z\right)$ event weights as a function of $p_T^Z$ for five, seven, and twelve telescoping radii (each starting at $R=0.4$ with additional interpretations with radius increasing in increments of 0.1) are shown in Figure \ref{fig:compsdb}.

\begin{figure}[!htbp]\captionsetup{justification=centering}
\begin{center}
\begin{subfigure}[t]{16pc}\centering\includegraphics[width=\textwidth]{figures/zht_comp/zht_comp_pois_rho_tru_ov_subel_wz_zh125_pt0}\caption{}\end{subfigure}
\begin{subfigure}[t]{16pc}\centering\includegraphics[width=\textwidth]{figures/zht_comp/zht_comp_pois_rho_tru_ov_subel_wz_zh125_pt1}\caption{}\end{subfigure}
\begin{subfigure}[t]{16pc}\centering\includegraphics[width=\textwidth]{figures/zht_comp/zht_comp_pois_rho_tru_ov_subel_wz_zh125_pt2}\caption{}\end{subfigure}
\begin{subfigure}[t]{16pc}\centering\includegraphics[width=\textwidth]{figures/zht_comp/zht_comp_pois_rho_tru_ov_subel_wz_zh125_pt3}\caption{}\end{subfigure}
\begin{subfigure}[t]{16pc}\centering\includegraphics[width=\textwidth]{figures/zht_comp/zht_comp_pois_rho_tru_ov_subel_wz_zh125_pt4}\caption{}\end{subfigure}
\begin{subfigure}[t]{16pc}\centering\includegraphics[width=\textwidth]{figures/zht_comp/zht_comp_pois_rho_tru_ov_subel_wz_zh125_pt-1}\caption{}\end{subfigure}
\caption{\label{fig:comprhozh125tru} $\rho\left(z\right)$ for the 5, 7, and 12 telescoping radii cases at truth level for the signal $ZH125$ sample over different values of $p_T^Z$: 0--90 \GeV\ (a), 90--120 \GeV\ (b), 120--160 \GeV\ (c), 160--200 \GeV\ (d), $>200$ \GeV\ (e), and all $p_T^Z$.}
\end{center}
\end{figure}
\clearpage
\begin{figure}[!htbp]\captionsetup{justification=centering}
\begin{center}
\begin{subfigure}[t]{16pc}\centering\includegraphics[width=\textwidth]{figures/zht_comp/zht_comp_pois_rho_tru_ov_subel_wz_zjets_pt0}\caption{}\end{subfigure}
\begin{subfigure}[t]{16pc}\centering\includegraphics[width=\textwidth]{figures/zht_comp/zht_comp_pois_rho_tru_ov_subel_wz_zjets_pt1}\caption{}\end{subfigure}
\begin{subfigure}[t]{16pc}\centering\includegraphics[width=\textwidth]{figures/zht_comp/zht_comp_pois_rho_tru_ov_subel_wz_zjets_pt2}\caption{}\end{subfigure}
\begin{subfigure}[t]{16pc}\centering\includegraphics[width=\textwidth]{figures/zht_comp/zht_comp_pois_rho_tru_ov_subel_wz_zjets_pt3}\caption{}\end{subfigure}
\begin{subfigure}[t]{16pc}\centering\includegraphics[width=\textwidth]{figures/zht_comp/zht_comp_pois_rho_tru_ov_subel_wz_zjets_pt4}\caption{}\end{subfigure}
\begin{subfigure}[t]{16pc}\centering\includegraphics[width=\textwidth]{figures/zht_comp/zht_comp_pois_rho_tru_ov_subel_wz_zjets_pt-1}\caption{}\end{subfigure}
\caption{\label{fig:comprhozjetstru} $\rho\left(z\right)$ for the 5, 7, and 12 telescoping radii cases at truth level for the signal $Z+$jets sample over different values of $p_T^Z$: 0--90 \GeV\ (a), 90--120 \GeV\ (b), 120--160 \GeV\ (c), 160--200 \GeV\ (d), $>200$ \GeV\ (e), and all $p_T^Z$.}
\end{center}
\end{figure}
\clearpage

\begin{figure}[!htbp]\captionsetup{justification=centering}
\begin{center}
\begin{subfigure}[t]{16pc}\centering\includegraphics[width=\textwidth]{figures/zht_comp/zht_comp_pois_rho_rec_ov_subel_trim_zh125_pt0}\caption{}\end{subfigure}
\begin{subfigure}[t]{16pc}\centering\includegraphics[width=\textwidth]{figures/zht_comp/zht_comp_pois_rho_rec_ov_subel_trim_zh125_pt1}\caption{}\end{subfigure}
\begin{subfigure}[t]{16pc}\centering\includegraphics[width=\textwidth]{figures/zht_comp/zht_comp_pois_rho_rec_ov_subel_trim_zh125_pt2}\caption{}\end{subfigure}
\begin{subfigure}[t]{16pc}\centering\includegraphics[width=\textwidth]{figures/zht_comp/zht_comp_pois_rho_rec_ov_subel_trim_zh125_pt3}\caption{}\end{subfigure}
\begin{subfigure}[t]{16pc}\centering\includegraphics[width=\textwidth]{figures/zht_comp/zht_comp_pois_rho_rec_ov_subel_trim_zh125_pt4}\caption{}\end{subfigure}
\begin{subfigure}[t]{16pc}\centering\includegraphics[width=\textwidth]{figures/zht_comp/zht_comp_pois_rho_rec_ov_subel_trim_zh125_pt-1}\caption{}\end{subfigure}
\caption{\label{fig:comprhozh125rec} $\rho\left(z\right)$ for the 5, 7, and 12 telescoping radii cases at reconstructed level distributions for the signal $ZH125$ sample over different values of $p_T^Z$: 0--90 \GeV\ (a), 90--120 \GeV\ (b), 120--160 \GeV\ (c), 160--200 \GeV\ (d), $>200$ \GeV\ (e), and all $p_T^Z$.}
\end{center}
\end{figure}
\clearpage

\begin{figure}[!htbp]\captionsetup{justification=centering}
\begin{center}
\begin{subfigure}[t]{16pc}\centering\includegraphics[width=\textwidth]{figures/zht_comp/zht_comp_pois_rho_rec_ov_subel_trim_zjets_pt0}\caption{}\end{subfigure}
\begin{subfigure}[t]{16pc}\centering\includegraphics[width=\textwidth]{figures/zht_comp/zht_comp_pois_rho_rec_ov_subel_trim_zjets_pt1}\caption{}\end{subfigure}
\begin{subfigure}[t]{16pc}\centering\includegraphics[width=\textwidth]{figures/zht_comp/zht_comp_pois_rho_rec_ov_subel_trim_zjets_pt2}\caption{}\end{subfigure}
\begin{subfigure}[t]{16pc}\centering\includegraphics[width=\textwidth]{figures/zht_comp/zht_comp_pois_rho_rec_ov_subel_trim_zjets_pt3}\caption{}\end{subfigure}
\begin{subfigure}[t]{16pc}\centering\includegraphics[width=\textwidth]{figures/zht_comp/zht_comp_pois_rho_rec_ov_subel_trim_zjets_pt4}\caption{}\end{subfigure}
\begin{subfigure}[t]{16pc}\centering\includegraphics[width=\textwidth]{figures/zht_comp/zht_comp_pois_rho_rec_ov_subel_trim_zjets_pt-1}\caption{}\end{subfigure}
\caption{\label{fig:comprhozjetsrec} $\rho\left(z\right)$ for the 5, 7, and 12 telescoping radii cases at reconstructed level for the background $Z+$jets sample over different values of $p_T^Z$: 0--90 \GeV\ (a), 90--120 \GeV\ (b), 120--160 \GeV\ (c), 160--200 \GeV\ (d), $>200$ \GeV\ (e), and all $p_T^Z$.}
\end{center}
\end{figure}
\clearpage

\begin{figure}[!htbp]\captionsetup{justification=centering}
\begin{center}
\begin{subfigure}[t]{16pc}\centering\includegraphics[width=\textwidth]{figures/body/tru_ov_subel_wz_ntel11_pois_sdb_r4_lo100_hi130_zjets}\caption{}\end{subfigure}
\begin{subfigure}[t]{16pc}\centering\includegraphics[width=\textwidth]{figures/body/rec_ov_subel_trim_ntel11_pois_sdb_r4_lo95_hi140_zjets}\caption{}\end{subfigure}
\begin{subfigure}[t]{16pc}\centering\includegraphics[width=\textwidth]{figures/zht_comp/zht_comp_pois_sdb_tru_ov_subel_wz_imp_z}\caption{}\end{subfigure}
\begin{subfigure}[t]{16pc}\centering\includegraphics[width=\textwidth]{figures/zht_comp/zht_comp_pois_sdb_rec_ov_subel_trim_imp_z}\caption{}\end{subfigure}
\caption{\label{fig:compsdb} $S/\delta B$ improvements for the 5, 7, and 12 telescoping radiii cases as a function of $p_T^Z$.  Absolute values for improvements as a function of $p_T^Z$ are shown for optimal anti-$k_t$ with $R=0.4$ jets with $m_{bb}$ cut at truth level in (a) and at reconstructed level in (b).  Improvements for $t\left(z\right)=z$ are shown at truth-level in (c) and at reconstructed level in (d).}
\end{center}
\end{figure}

\clearpage
\section{$m_{bb}$ Distributions for all Radii}
\label{sec:mbball}
$m_{bb}$ distributions at all telescoping radii are given for signal $ZH125$ and background $Z+$jets samples at truth and reconstructed-level in Figures \ref{fig:allradtru} and \ref{fig:allradrec}, respectively.
\begin{figure}[!htbp]\captionsetup{justification=centering}
\begin{center}
\begin{subfigure}[t]{16pc}\centering\includegraphics[width=\textwidth]{figures/body/tru_ov_subel_wz_mbbapp1_zh125}\caption{}\end{subfigure}
\begin{subfigure}[t]{16pc}\centering\includegraphics[width=\textwidth]{figures/body/tru_ov_subel_wz_mbbapp2_zh125}\caption{}\end{subfigure}
\begin{subfigure}[t]{16pc}\centering\includegraphics[width=\textwidth]{figures/body/tru_ov_subel_wz_mbbapp1_zjets}\caption{}\end{subfigure}
\begin{subfigure}[t]{16pc}\centering\includegraphics[width=\textwidth]{figures/body/tru_ov_subel_wz_mbbapp2_zjets}\caption{}\end{subfigure}
\caption{\label{fig:allradtru}The $m_{bb}$ distribution for the telescoping jets with $R=0.4$--0.9 and $R=1.0$--1.5 truth-level jets is shown for the $ZH125$ sample in (a) and (b), respectively, and for the $Z+$jets sample in (c) and (d).}
\end{center}
\end{figure}

\begin{figure}[!htbp]\captionsetup{justification=centering}
\begin{center}
\begin{subfigure}[t]{16pc}\centering\includegraphics[width=\textwidth]{figures/body/rec_ov_subel_trim_mbbapp1_zh125}\caption{}\end{subfigure}
\begin{subfigure}[t]{16pc}\centering\includegraphics[width=\textwidth]{figures/body/rec_ov_subel_trim_mbbapp2_zh125}\caption{}\end{subfigure}
\begin{subfigure}[t]{16pc}\centering\includegraphics[width=\textwidth]{figures/body/rec_ov_subel_trim_mbbapp1_zjets}\caption{}\end{subfigure}
\begin{subfigure}[t]{16pc}\centering\includegraphics[width=\textwidth]{figures/body/rec_ov_subel_trim_mbbapp2_zjets}\caption{}\end{subfigure}
\caption{\label{fig:allradrec}The $m_{bb}$ distribution for the telescoping jets with $R=0.4$--0.9 and $R=1.0$--1.5 reconstructed-level jets is shown for the $ZH125$ sample in (a) and (b), respectively, and for the $Z+$jets sample in (c) and (d).}
\end{center}
\end{figure}

\bibliographystyle{atlasBibStyleWithTitle}
%\bibliography{../common/bibliography}
\end{document}

\clearpage

\section{Different Truth-Level Options}
\begin{table}[htbp]
\caption{$m_{bb}$ windows studied.  These windows were chosen to optimize significances over all $p_T^Z$.
\label{tab:masswindow}}
\begin{center}
\begin{tabular}{|c|c|c|p{4cm}|p{4cm}|c|}
\hline
 Incl. $\mu,\nu$? & trim? &  \# $R$'s &          $S/\delta B$ Type                                  &           Optimal $m_{bb}$ Window                   &   Improvement\\
\hhline{|=|=|=|=|=|=|}
              yes &  no   &    ---    & anti-$k_t$ $R=0.4$                                          &  110--130 \GeV\                                     &   --- \\
\hhline{|~|~|-|-|-|-|}
                  &       &     5     & $t\left(z\right)=z$                                         &  120--130 \GeV\                                     & 22.5\%   \\
                  &       &           & $t\left(z\right)=\rho_S\left(z\right)/\rho_B\left(z\right)$ &  120--130 \GeV\                                     & 27.0\%   \\
\hhline{|~|~|-|-|-|-|}
                  &       &     7     & $t\left(z\right)=z$                                         &  120--130 \GeV\                                     & 19.7\%   \\
                  &       &           & $t\left(z\right)=\rho_S\left(z\right)/\rho_B\left(z\right)$ &  120--130 \GeV\                                     & 26.1\%   \\
\hhline{|~|~|-|-|-|-|}
                  &       &    12     & $t\left(z\right)=z$                                         &  120--130 \GeV\                                     & 8.73\%   \\
                  &       &           & $t\left(z\right)=\rho_S\left(z\right)/\rho_B\left(z\right)$ &  120--130 \GeV\                                     & 15.1\%   \\
                  &       &           & anti-$k_t$ $R=0.4$, telescoping $R=0.5$                     &  115--130 \GeV\ ($R=0.4$), 115--130 \GeV\ ($R=0.5$) & 5.54\%   \\
\hhline{|=|=|=|=|=|=|}
              yes &  yes  &    ---    & anti-$k_t$ $R=0.4$                                          &  110--130 \GeV\                                     &   --- \\
\hhline{|~|~|-|-|-|-|}
                  &       &     5     & $t\left(z\right)=z$                                         &  115--130 \GeV\                                     & 8.49\%   \\
                  &       &           & $t\left(z\right)=\rho_S\left(z\right)/\rho_B\left(z\right)$ &  115--130 \GeV\                                     & 11.8\%   \\
\hhline{|~|~|-|-|-|-|}
                  &       &     7     & $t\left(z\right)=z$                                         &  115--130 \GeV\                                     & 10.3\%   \\
                  &       &           & $t\left(z\right)=\rho_S\left(z\right)/\rho_B\left(z\right)$ &  115--130 \GeV\                                     & 14.2\%   \\
\hhline{|~|~|-|-|-|-|}
                  &       &    12     & $t\left(z\right)=z$                                         &  115--130 \GeV\                                     & 9.77\%   \\
                  &       &           & $t\left(z\right)=\rho_S\left(z\right)/\rho_B\left(z\right)$ &  115--130 \GeV\                                     & 15.1\%   \\
                  &       &           & anti-$k_t$ $R=0.4$, telescoping $R=0.7$                     &  110--130 \GeV\ ($R=0.4$), 115--130 \GeV\ ($R=0.7$) & 2.89\%   \\
\hhline{|=|=|=|=|=|=|}
               no &  no   &    ---    & anti-$k_t$ $R=0.4$                                          &  110--130 \GeV\                                     &   --- \\
\hhline{|~|~|-|-|-|-|}
                  &       &     5     & $t\left(z\right)=z$                                         &  120--130 \GeV\                                     & 15.3\%   \\
                  &       &           & $t\left(z\right)=\rho_S\left(z\right)/\rho_B\left(z\right)$ &  120--130 \GeV\                                     & 18.9\%   \\
\hhline{|~|~|-|-|-|-|}
                  &       &     7     & $t\left(z\right)=z$                                         &  120--130 \GeV\                                     & 17.4\%   \\
                  &       &           & $t\left(z\right)=\rho_S\left(z\right)/\rho_B\left(z\right)$ &  120--130 \GeV\                                     & 20.3\%   \\
                  &       &    12     & $t\left(z\right)=z$                                         &  120--130 \GeV\                                     & 16.3\%   \\
                  &       &           & $t\left(z\right)=\rho_S\left(z\right)/\rho_B\left(z\right)$ &  120--130 \GeV\                                     & 20.0\%   \\
\hhline{|~|~|-|-|-|-|}
                  &       &     5     & anti-$k_t$ $R=0.4$, telescoping $R=0.5$                     &  100--135 \GeV\ ($R=0.4$),  95--130 \GeV\ ($R=0.5$) & 3.66\%   \\
                  &       &     7     & anti-$k_t$ $R=0.4$, telescoping $R=0.9$                     &  100--130 \GeV\ ($R=0.4$), 105--155 \GeV\ ($R=0.9$) & 4.22\%   \\
                  &       &    12     & anti-$k_t$ $R=0.4$, telescoping $R=1.1$                     &  100--130 \GeV\ ($R=0.4$), 105--165 \GeV\ ($R=1.1$) & 4.45\%   \\
\hhline{|=|=|=|=|=|=|}
               no &  yes  &    ---    & anti-$k_t$ $R=0.4$                                          &  100--130 \GeV\                                     &   --- \\
                  &       &     5     & $t\left(z\right)=z$                                         &  105--130 \GeV\                                     & 7.00\%   \\
                  &       &           & $t\left(z\right)=\rho_S\left(z\right)/\rho_B\left(z\right)$ &  105--130 \GeV\                                     & 9.82\%   \\
\hhline{|~|~|-|-|-|-|}
                  &       &     7     & $t\left(z\right)=z$                                         &  105--130 \GeV\                                     & 9.43\%   \\
                  &       &           & $t\left(z\right)=\rho_S\left(z\right)/\rho_B\left(z\right)$ &  105--130 \GeV\                                     & 13.3\%   \\
\hhline{|~|~|-|-|-|-|}
                  &       &    12     & $t\left(z\right)=z$                                         &  105--130 \GeV\                                     & 12.2\%   \\
                  &       &           & $t\left(z\right)=\rho_S\left(z\right)/\rho_B\left(z\right)$ &  110--130 \GeV\                                     & 18.6\%   \\
\hhline{|~|~|-|-|-|-|}
                  &       &     5     & anti-$k_t$ $R=0.4$, telescoping $R=0.8$                     &  100--135 \GeV\ ($R=0.4$),  95--130 \GeV\ ($R=0.8$) & 3.84\%   \\
                  &       &    12     & anti-$k_t$ $R=0.4$, telescoping $R=0.9$                     &  100--130 \GeV\ ($R=0.4$), 105--155 \GeV\ ($R=0.9$) & 4.46\%   \\
\hhline{|=|=|=|=|=|=|}
\end{tabular}
\end{center}
\end{table}

\begin{figure}[!htbp]\captionsetup{justification=centering}
\begin{center}
%mbb app1 and app2
\begin{subfigure}[t]{\textwidth}\centering\includegraphics[width=\textwidth]{dth}\caption{}\end{subfigure}
\begin{subfigure}[t]{\textwidth}\centering\includegraphics[width=\textwidth]{dth}\caption{}\end{subfigure}
\begin{subfigure}[t]{\textwidth}\centering\includegraphics[width=\textwidth]{dth}\caption{}\end{subfigure}
\subfigure[]{\includegraphics[width=16pc]{../../plots/new_note/truth/tru_plain_mbbapp2_zjets}}%FIXME!
\caption{\label{fig:mbbplain}The $m_{bb}$ distributions for the no trim, include $\mu,\nu$ case.}
\end{center}
\end{figure}
\begin{figure}[!htbp]\captionsetup{justification=centering}
\begin{center}
%2mw rho S,B,SB
\begin{subfigure}[t]{16pc}\centering\includegraphics[width=\textwidth]{../../plots/new_note/truth/tru_plain_ntel11_zh125_rho_ptall_lo120_hi130}\caption{}\end{subfigure}
\begin{subfigure}[t]{16pc}\centering\includegraphics[width=\textwidth]{../../plots/new_note/truth/tru_plain_ntel11_zjets_rho_ptall_lo120_hi130}\caption{}\end{subfigure}
\begin{subfigure}[t]{18pc}\centering\includegraphics[width=\textwidth]{../../plots/new_note/truth/tru_plain_ntel11_zjets_rhoSB_ptall_lo120_hi130}\caption{}\end{subfigure}
\begin{subfigure}[t]{18pc}\centering\includegraphics[width=\textwidth]{../../plots/new_note/truth/tru_plain_zh125_mbbRv4_ptall_r5}\caption{}\end{subfigure}
\begin{subfigure}[t]{18pc}\centering\includegraphics[width=\textwidth]{../../plots/new_note/truth/tru_plain_zjets_mbbRv4_ptall_r5}\caption{}\end{subfigure}
\caption{\label{fig:mbbplain}The $rho$ and double $m_{bb}$ distributions for the no trim, include $\mu,\nu$ case.}
\end{center}
\end{figure}
\begin{figure}[!htbp]\captionsetup{justification=centering}
\begin{center}
%2mw rho S,B,SB
\begin{subfigure}[t]{18pc}\centering\includegraphics[width=\textwidth]{../../plots/new_note/truth/tru_plain_ntel11_sdb_rho_lo120_hi130_zjets}\caption{}\end{subfigure}
\begin{subfigure}[t]{18pc}\centering\includegraphics[width=\textwidth]{../../plots/new_note/truth/tru_plain_ntel11_sdb_z_lo120_hi130_zjets}\caption{}\end{subfigure}
\begin{subfigure}[t]{18pc}\centering\includegraphics[width=\textwidth]{../../plots/new_note/truth/tru_plain_ntel11_sdb_r4_lo115_hi130_r5_lo115_hi130_zjets}\caption{}\end{subfigure}
\caption{\label{fig:mbbplain}The $S/(\delta B)$ distributions for the no trim, include $\mu,\nu$ case.}
\end{center}
\end{figure}

\begin{figure}[!htbp]\captionsetup{justification=centering}
\begin{center}
%mbb app1 and app2
\begin{subfigure}[t]{18pc}\centering\includegraphics[width=\textwidth]{../../plots/new_note/truth/tru_trim_mbbapp1_zh125}\caption{}\end{subfigure}
\begin{subfigure}[t]{18pc}\centering\includegraphics[width=\textwidth]{../../plots/new_note/truth/tru_trim_mbbapp2_zh125}\caption{}\end{subfigure}
\begin{subfigure}[t]{18pc}\centering\includegraphics[width=\textwidth]{../../plots/new_note/truth/tru_trim_mbbapp1_zjets}\caption{}\end{subfigure}
\begin{subfigure}[t]{18pc}\centering\includegraphics[width=\textwidth]{../../plots/new_note/truth/tru_trim_mbbapp2_zjets}\caption{}\end{subfigure}
\caption{\label{fig:mbbtrim}The $m_{bb}$ distributions for the no trim, include $\mu,\nu$ case.}
\end{center}
\end{figure}
\begin{figure}[!htbp]\captionsetup{justification=centering}
\begin{center}
%2mw rho S,B,SB
\begin{subfigure}[t]{18pc}\centering\includegraphics[width=\textwidth]{../../plots/new_note/truth/tru_trim_ntel11_zh125_rho_ptall_lo115_hi130}\caption{}\end{subfigure}
\begin{subfigure}[t]{18pc}\centering\includegraphics[width=\textwidth]{../../plots/new_note/truth/tru_trim_ntel11_zjets_rho_ptall_lo115_hi130}\caption{}\end{subfigure}
\begin{subfigure}[t]{18pc}\centering\includegraphics[width=\textwidth]{../../plots/new_note/truth/tru_trim_ntel11_zjets_rhoSB_ptall_lo115_hi130}\caption{}\end{subfigure}
\begin{subfigure}[t]{18pc}\centering\includegraphics[width=\textwidth]{../../plots/new_note/truth/tru_trim_zh125_mbbRv4_ptall_r7}\caption{}\end{subfigure}
\begin{subfigure}[t]{18pc}\centering\includegraphics[width=\textwidth]{../../plots/new_note/truth/tru_trim_zjets_mbbRv4_ptall_r7}\caption{}\end{subfigure}
\caption{\label{fig:mbbtrim}The $rho$ and double $m_{bb}$ distributions for the no trimmed, include $\mu,\nu$ case.}
\end{center}
\end{figure}
\begin{figure}[!htbp]\captionsetup{justification=centering}
\begin{center}
%2mw rho S,B,SB
\begin{subfigure}[t]{18pc}\centering\includegraphics[width=\textwidth]{../../plots/new_note/truth/tru_trim_ntel11_sdb_rho_lo115_hi130_zjets}\caption{}\end{subfigure}
\begin{subfigure}[t]{18pc}\centering\includegraphics[width=\textwidth]{../../plots/new_note/truth/tru_trim_ntel11_sdb_z_lo115_hi130_zjets}\caption{}\end{subfigure}
\begin{subfigure}[t]{18pc}\centering\includegraphics[width=\textwidth]{../../plots/new_note/truth/tru_trim_ntel11_sdb_r4_lo110_hi130_r7_lo115_hi130_zjets}\caption{}\end{subfigure}
\caption{\label{fig:mbbtrim}The $S/(\delta B)$ distributions for the trimmed, include $\mu,\nu$ case.}
\end{center}
\end{figure}

\begin{figure}[!htbp]\captionsetup{justification=centering}
\begin{center}
%mbb app1 and app2
\begin{subfigure}[t]{18pc}\centering\includegraphics[width=\textwidth]{../../plots/new_note/truth/tru_wz_mbbapp1_zh125}\caption{}\end{subfigure}
\begin{subfigure}[t]{18pc}\centering\includegraphics[width=\textwidth]{../../plots/new_note/truth/tru_wz_mbbapp2_zh125}\caption{}\end{subfigure}
\begin{subfigure}[t]{18pc}\centering\includegraphics[width=\textwidth]{../../plots/new_note/truth/tru_wz_mbbapp1_zjets}\caption{}\end{subfigure}
\begin{subfigure}[t]{18pc}\centering\includegraphics[width=\textwidth]{../../plots/new_note/truth/tru_wz_mbbapp2_zjets}\caption{}\end{subfigure}
\caption{\label{fig:mbbwz}The $m_{bb}$ distributions for the no trim, don't include $\mu,\nu$ case.}
\end{center}
\end{figure}
\begin{figure}[!htbp]\captionsetup{justification=centering}
\begin{center}
%2mw rho S,B,SB
\begin{subfigure}[t]{18pc}\centering\includegraphics[width=\textwidth]{../../plots/new_note/truth/tru_wz_ntel11_zh125_rho_ptall_lo120_hi130}\caption{}\end{subfigure}
\begin{subfigure}[t]{18pc}\centering\includegraphics[width=\textwidth]{../../plots/new_note/truth/tru_wz_ntel11_zjets_rho_ptall_lo120_hi130}\caption{}\end{subfigure}
\begin{subfigure}[t]{18pc}\centering\includegraphics[width=\textwidth]{../../plots/new_note/truth/tru_wz_ntel11_zjets_rhoSB_ptall_lo120_hi130}\caption{}\end{subfigure}
\begin{subfigure}[t]{18pc}\centering\includegraphics[width=\textwidth]{../../plots/new_note/truth/tru_wz_zh125_mbbRv4_ptall_r11}\caption{}\end{subfigure}
\begin{subfigure}[t]{18pc}\centering\includegraphics[width=\textwidth]{../../plots/new_note/truth/tru_wz_zjets_mbbRv4_ptall_r11}\caption{}\end{subfigure}
\caption{\label{fig:mbbwz}The $rho$ and double $m_{bb}$ distributions for no trim, don't include $\mu,\nu$ case.}
\end{center}
\end{figure}
\begin{figure}[!htbp]\captionsetup{justification=centering}
\begin{center}
%2mw rho S,B,SB
\begin{subfigure}[t]{18pc}\centering\includegraphics[width=\textwidth]{../../plots/new_note/truth/tru_wz_ntel11_sdb_rho_lo120_hi130_zjets}\caption{}\end{subfigure}
\begin{subfigure}[t]{18pc}\centering\includegraphics[width=\textwidth]{../../plots/new_note/truth/tru_wz_ntel11_sdb_z_lo120_hi130_zjets}\caption{}\end{subfigure}
\begin{subfigure}[t]{18pc}\centering\includegraphics[width=\textwidth]{../../plots/new_note/truth/tru_wz_ntel11_sdb_r4_lo100_hi130_r11_lo105_hi165_zjets}\caption{}\end{subfigure}
\caption{\label{fig:mbbwz}The $S/(\delta B)$ distributions for the no trim, don't include $\mu,\nu$ case.}
\end{center}
\end{figure}


\begin{figure}[!htbp]\captionsetup{justification=centering}
\begin{center}
%mbb app1 and app2
\begin{subfigure}[t]{18pc}\centering\includegraphics[width=\textwidth]{../../plots/new_note/truth/tru_wztrim_mbbapp1_zh125}\caption{}\end{subfigure}
\begin{subfigure}[t]{18pc}\centering\includegraphics[width=\textwidth]{../../plots/new_note/truth/tru_wztrim_mbbapp2_zh125}\caption{}\end{subfigure}
\begin{subfigure}[t]{18pc}\centering\includegraphics[width=\textwidth]{../../plots/new_note/truth/tru_wztrim_mbbapp1_zjets}\caption{}\end{subfigure}
\begin{subfigure}[t]{18pc}\centering\includegraphics[width=\textwidth]{../../plots/new_note/truth/tru_wztrim_mbbapp2_zjets}\caption{}\end{subfigure}
\caption{\label{fig:mbbwztrim}The $m_{bb}$ distributions for the trimmed, don't include $\mu,\nu$ case.}
\end{center}
\end{figure}
\begin{figure}[!htbp]\captionsetup{justification=centering}
\begin{center}
%2mw rho S,B,SB
\begin{subfigure}[t]{18pc}\centering\includegraphics[width=\textwidth]{../../plots/new_note/truth/tru_wztrim_ntel11_zh125_rho_ptall_lo110_hi130}\caption{}\end{subfigure}
\begin{subfigure}[t]{18pc}\centering\includegraphics[width=\textwidth]{../../plots/new_note/truth/tru_wztrim_ntel11_zjets_rho_ptall_lo110_hi130}\caption{}\end{subfigure}
\begin{subfigure}[t]{18pc}\centering\includegraphics[width=\textwidth]{../../plots/new_note/truth/tru_wztrim_ntel11_zjets_rhoSB_ptall_lo110_hi130}\caption{}\end{subfigure}
\begin{subfigure}[t]{18pc}\centering\includegraphics[width=\textwidth]{../../plots/new_note/truth/tru_wztrim_zh125_mbbRv4_ptall_r9}\caption{}\end{subfigure}
\begin{subfigure}[t]{18pc}\centering\includegraphics[width=\textwidth]{../../plots/new_note/truth/tru_wztrim_zjets_mbbRv4_ptall_r9}\caption{}\end{subfigure}
\caption{\label{fig:mbbwztrim}The $rho$ and double $m_{bb}$ distributions for trimmed, don't include $\mu,\nu$ case.}
\end{center}
\end{figure}
\begin{figure}[!htbp]\captionsetup{justification=centering}
\begin{center}
%2mw rho S,B,SB
\begin{subfigure}[t]{18pc}\centering\includegraphics[width=\textwidth]{../../plots/new_note/truth/tru_wztrim_ntel11_sdb_rho_lo110_hi130_zjets}\caption{}\end{subfigure}
\begin{subfigure}[t]{18pc}\centering\includegraphics[width=\textwidth]{../../plots/new_note/truth/tru_wztrim_ntel11_sdb_z_lo105_hi130_zjets}\caption{}\end{subfigure}
\begin{subfigure}[t]{18pc}\centering\includegraphics[width=\textwidth]{../../plots/new_note/truth/tru_wztrim_ntel11_sdb_r4_lo100_hi130_r9_lo105_hi130_zjets}\caption{}\end{subfigure}
\caption{\label{fig:mbbwztrim}The $S/(\delta B)$ distributions for the trimmed, don't include $\mu,\nu$ case.}
\end{center}
\end{figure}
\clearpage
\section{Plots at Reconstructed-Level with Overlap Removal}
\section{Removal Over Electrons and Muons}
\begin{table}[htbp]
\caption{$m_{bb}$ windows studied.  These windows were chosen to optimize significances over all $p_T^Z$.
\label{tab:masswindow}}
\begin{center}
\begin{tabular}{|c|p{5cm}|p{5cm}|c|}
\hline
 \# $R$'s &              $S/\delta B$ Type                                  &           Optimal $m_{bb}$ Window     &   Improvement\\
\hline
 ---  &                                                  anti-$k_t$ $R=0.4$ &                         90--140 \GeV\ &     ---  \\
\hhline{|~|-|-|-|}
 5    &                                                 $t\left(z\right)=z$ &                        110--145 \GeV\ &    9.2\%  \\
      &         $t\left(z\right)=\rho_S\left(z\right)/\rho_B\left(z\right)$ &                        110--140 \GeV\ &    13.35\%  \\
\hhline{|~|-|-|-|}
 7    &                                                 $t\left(z\right)=z$ &                        110--150 \GeV\ &    7.5\%  \\
      &         $t\left(z\right)=\rho_S\left(z\right)/\rho_B\left(z\right)$ &                        110--145 \GeV\ &    12.27\%  \\
\hhline{|~|-|-|-|}
 12   &                                                 $t\left(z\right)=z$ &                        110--150 \GeV\ &    3.8\%  \\
      &         $t\left(z\right)=\rho_S\left(z\right)/\rho_B\left(z\right)$ &                        115--145 \GeV\ &   11.06\%  \\
      &                             anti-$k_t$ $R=0.4$, telescoping $R=0.6$ &                    (glitch...n/a)---  &    0.067\%  \\
\hline
\end{tabular}
\end{center}
\end{table}

\begin{figure}[!htbp]\captionsetup{justification=centering}
\begin{center}
%mbb app1 and app2
\begin{subfigure}[t]{18pc}\centering\includegraphics[width=\textwidth]{figures/recon/app_ovrm/rec_trim_mbbmain_zh125}\caption{}\end{subfigure}
\begin{subfigure}[t]{18pc}\centering\includegraphics[width=\textwidth]{figures/recon/app_ovrm/rec_trim_mbbmain_zjets}\caption{}\end{subfigure}
\caption{\label{fig:mbbwz}The $m_{bb}$ distributions for the reconstructed-level trimmed with $e,\mu$ overlap removal.}
\end{center}
\end{figure}
\begin{figure}[!htbp]\captionsetup{justification=centering}
\begin{center}
%2mw rho S,B,SB
\begin{subfigure}[t]{18pc}\centering\includegraphics[width=\textwidth]{figures/recon/app_ovrm/rec_trim_ntel11_zh125_rho_ptall_lo115_hi145}\caption{}\end{subfigure}
\begin{subfigure}[t]{18pc}\centering\includegraphics[width=\textwidth]{figures/recon/app_ovrm/rec_trim_ntel11_zjets_rho_ptall_lo115_hi145}\caption{}\end{subfigure}
\begin{subfigure}[t]{18pc}\centering\includegraphics[width=\textwidth]{figures/recon/app_ovrm/rec_trim_ntel11_zjets_rhoSB_ptall_lo115_hi145}\caption{}\end{subfigure}
\caption{\label{fig:mbbwz}The $\rho$ and double $m_{bb}$ distributions for the reconstructed-level trimmed with $e,\mu$ overlap removal.}
\end{center}
\end{figure}
\begin{figure}[!htbp]\captionsetup{justification=centering}
\begin{center}
%2mw rho S,B,SB
\begin{subfigure}[t]{18pc}\centering\includegraphics[width=\textwidth]{figures/recon/app_ovrm/rec_trim_ntel11_sdb_rho_lo115_hi145_zjets}\caption{}\end{subfigure}
\begin{subfigure}[t]{18pc}\centering\includegraphics[width=\textwidth]{figures/recon/app_ovrm/rec_trim_ntel11_sdb_z_lo110_hi150_zjets}\caption{}\end{subfigure}
\begin{subfigure}[t]{18pc}\centering\includegraphics[width=\textwidth]{figures/recon/app_ovrm/rec_trim_ntel11_sdb_r4_lo120_hi130_r6_lo120_hi130_zjets}\caption{}\end{subfigure}
\caption{\label{fig:mbbwz}The $S/(\delta B)$ distributions for the reconstructed-level trimmed with $e,\mu$ overlap removal.}
\end{center}
\end{figure}
\clearpage

\section{Removal Over Electrons Only}
\begin{table}[htbp]
\caption{$m_{bb}$ windows studied.  These windows were chosen to optimize significances over all $p_T^Z$.
\label{tab:masswindow}}
\begin{center}
\begin{tabular}{|c|p{5cm}|p{5cm}|c|}
\hline
 \# $R$'s &              $S/\delta B$ Type                                  &           Optimal $m_{bb}$ Window     &   Improvement\\
\hline
 ---  &                                                  anti-$k_t$ $R=0.4$ &                         90--140 \GeV\ &     ---  \\
\hline
   5  &                                                 $t\left(z\right)=z$ &                        110--145 \GeV\ &  12.24\%  \\
      &         $t\left(z\right)=\rho_S\left(z\right)/\rho_B\left(z\right)$ &                        105--145 \GeV\ &  16.39\%  \\
\hline
   7  &                                                 $t\left(z\right)=z$ &                        110--150 \GeV\ &  12.39\%  \\
      &         $t\left(z\right)=\rho_S\left(z\right)/\rho_B\left(z\right)$ &                        110--145 \GeV\ &  17.19\%  \\
\hline
  12  &                                                 $t\left(z\right)=z$ &                        115--155 \GeV\ &  12.12\%  \\
      &         $t\left(z\right)=\rho_S\left(z\right)/\rho_B\left(z\right)$ &                        110--155 \GeV\ &  19.64\%  \\
      &                             anti-$k_t$ $R=0.4$, telescoping $R=0.8$ &          95--140 \GeV, 95--160 \GeV\ &  3.407\%  \\
\hline
\end{tabular}
\end{center}
\end{table}

\begin{figure}[!htbp]\captionsetup{justification=centering}
\begin{center}
%mbb app1 and app2
\begin{subfigure}[t]{18pc}\centering\includegraphics[width=\textwidth]{figures/recon/ov_rmel/rec_trim_mbbmain_zh125}\caption{}\end{subfigure}
\begin{subfigure}[t]{18pc}\centering\includegraphics[width=\textwidth]{figures/recon/ov_rmel/rec_trim_mbbmain_zjets}\caption{}\end{subfigure}
\caption{\label{fig:mbbwz}The $m_{bb}$ distributions for the reconstructed-level trimmed with $e$ overlap removal.}
\end{center}
\end{figure}
\begin{figure}[!htbp]\captionsetup{justification=centering}
\begin{center}
%2mw rho S,B,SB
\begin{subfigure}[t]{18pc}\centering\includegraphics[width=\textwidth]{figures/recon/ov_rmel/rec_trim_ntel11_zh125_rho_ptall_lo110_hi155}\caption{}\end{subfigure}
\begin{subfigure}[t]{18pc}\centering\includegraphics[width=\textwidth]{figures/recon/ov_rmel/rec_trim_ntel11_zjets_rho_ptall_lo110_hi155}\caption{}\end{subfigure}
\begin{subfigure}[t]{18pc}\centering\includegraphics[width=\textwidth]{figures/recon/ov_rmel/rec_trim_ntel11_zjets_rhoSB_ptall_lo110_hi155}\caption{}\end{subfigure}
\caption{\label{fig:mbbwz}The $\rho$ and double $m_{bb}$ distributions for the reconstructed-level trimmed with $e$ overlap removal.}
\end{center}
\end{figure}
\begin{figure}[!htbp]\captionsetup{justification=centering}
\begin{center}
%2mw rho S,B,SB
\begin{subfigure}[t]{18pc}\centering\includegraphics[width=\textwidth]{figures/recon/ov_rmel/rec_trim_ntel11_sdb_rho_lo110_hi155_zjets}\caption{}\end{subfigure}
\begin{subfigure}[t]{18pc}\centering\includegraphics[width=\textwidth]{figures/recon/ov_rmel/rec_trim_ntel11_sdb_z_lo115_hi155_zjets}\caption{}\end{subfigure}
\begin{subfigure}[t]{18pc}\centering\includegraphics[width=\textwidth]{figures/recon/ov_rmel/rec_trim_ntel11_sdb_r4_lo95_hi140_r8_lo95_hi160_zjets}\caption{}\end{subfigure}
\caption{\label{fig:mbbwz}The $S/(\delta B)$ distributions for the reconstructed-level trimmed with $e$ overlap removal.}
\end{center}
\end{figure}
\clearpage

\section{Plots at Truth-Level with Overlap Removal}
\section{Removal Over Electrons and Muons}
\begin{table}[htbp]
\caption{$m_{bb}$ windows studied.  These windows were chosen to optimize significances over all $p_T^Z$.
\label{tab:masswindow}}
\begin{center}
\begin{tabular}{|c|c|p{5cm}|p{5cm}|c|}
\hline
\hhline{|=|=|=|=|=|}
Type                & \# $R$'s &              $S/\delta B$ Type                                     &           Optimal $m_{bb}$ Window     &   Improvement\\
\hhline{|=|=|=|=|=|}
    incl. $\mu,\nu$ & ---     &                                                  anti-$k_t$ $R=0.4$ &                        110--130 \GeV\ &     ---  \\
\hhline{|~|~|-|-|-|}
                    &   5     &                                                 $t\left(z\right)=z$ &                        120--130 \GeV\ &  23.55\%  \\
                    &         &         $t\left(z\right)=\rho_S\left(z\right)/\rho_B\left(z\right)$ &                        120--130 \GeV\ &  27.66\%  \\
\hhline{|~|~|-|-|-|}
                    &   7     &                                                 $t\left(z\right)=z$ &                        120--130 \GeV\ &   21.5\%  \\
                    &         &         $t\left(z\right)=\rho_S\left(z\right)/\rho_B\left(z\right)$ &                        120--130 \GeV\ &  27.36\%  \\
\hhline{|~|~|-|-|-|}
                    &  12     &                                                 $t\left(z\right)=z$ &                        120--130 \GeV\ &  11.59\%  \\
                    &         &         $t\left(z\right)=\rho_S\left(z\right)/\rho_B\left(z\right)$ &                        120--130 \GeV\ &  18.12\%  \\
                    &         &                             anti-$k_t$ $R=0.4$, telescoping $R=0.5$ &        115--130 \GeV, 115--130 \GeV\ &  5.715\%  \\
\hhline{|=|=|=|=|=|}
incl. $\mu,\nu$, trim& ---    &                                                  anti-$k_t$ $R=0.4$ &                        110--130 \GeV\ &     ---  \\
\hhline{|~|~|-|-|-|}
                    &   5     &                                                 $t\left(z\right)=z$ &                        115--130 \GeV\ &  9.351\%  \\
                    &         &         $t\left(z\right)=\rho_S\left(z\right)/\rho_B\left(z\right)$ &                        115--130 \GeV\ &  12.25\%  \\
\hhline{|~|~|-|-|-|}
                    &   7     &                                                 $t\left(z\right)=z$ &                        115--130 \GeV\ &  11.68\%  \\
                    &         &         $t\left(z\right)=\rho_S\left(z\right)/\rho_B\left(z\right)$ &                        115--130 \GeV\ &  14.86\%  \\
\hhline{|~|~|-|-|-|}
                    &  12     &                                                 $t\left(z\right)=z$ &                        115--130 \GeV\ &  12.77\%  \\
                    &         &         $t\left(z\right)=\rho_S\left(z\right)/\rho_B\left(z\right)$ &                        115--130 \GeV\ &  17.32\%  \\
                    &         &                             anti-$k_t$ $R=0.4$, telescoping $R=0.7$ &        110--130 \GeV, 115--130 \GeV\ &   3.08\%  \\
\hhline{|=|=|=|=|=|}
       no $\mu,\nu$ & ---     &                                                  anti-$k_t$ $R=0.4$ &                        100--130 \GeV\ &     ---  \\
\hhline{|~|~|-|-|-|}
                    &   5     &                                                 $t\left(z\right)=z$ &                        115--130 \GeV\ &  13.61\%  \\
                    &         &         $t\left(z\right)=\rho_S\left(z\right)/\rho_B\left(z\right)$ &                        115--130 \GeV\ &  16.44\%  \\
\hhline{|~|~|-|-|-|}
                    &   7     &                                                 $t\left(z\right)=z$ &                        120--130 \GeV\ &  14.96\%  \\
                    &         &         $t\left(z\right)=\rho_S\left(z\right)/\rho_B\left(z\right)$ &                        120--130 \GeV\ &  17.73\%  \\
\hhline{|~|~|-|-|-|}
                    &  12     &                                                 $t\left(z\right)=z$ &                        120--130 \GeV\ &  13.47\%  \\
                    &         &         $t\left(z\right)=\rho_S\left(z\right)/\rho_B\left(z\right)$ &                        120--130 \GeV\ &  16.56\%  \\
                    &         &                             anti-$k_t$ $R=0.4$, telescoping $R=0.5$ &         100--135 \GeV, 95--130 \GeV\ &  2.796\%  \\
\hhline{|=|=|=|=|=|}
 no $\mu,\nu$, trim & ---     &                                                  anti-$k_t$ $R=0.4$ &                        100--130 \GeV\ &     ---  \\
\hhline{|~|~|-|-|-|}
                    &   5     &                                                 $t\left(z\right)=z$ &                        105--130 \GeV\ &  5.217\%  \\
                    &         &         $t\left(z\right)=\rho_S\left(z\right)/\rho_B\left(z\right)$ &                        105--130 \GeV\ &  7.035\%  \\
\hhline{|~|~|-|-|-|}
                    &   7     &                                                 $t\left(z\right)=z$ &                        105--130 \GeV\ &  6.651\%  \\
                    &         &         $t\left(z\right)=\rho_S\left(z\right)/\rho_B\left(z\right)$ &                        105--130 \GeV\ &  8.796\%  \\
\hhline{|~|~|-|-|-|}
                    &  12     &                                                 $t\left(z\right)=z$ &                        105--130 \GeV\ &   6.67\%  \\
                    &         &         $t\left(z\right)=\rho_S\left(z\right)/\rho_B\left(z\right)$ &                        110--130 \GeV\ &  9.516\%  \\
                    &         &                             anti-$k_t$ $R=0.4$, telescoping $R=0.5$ &         100--130 \GeV, 95--130 \GeV\ & 0.9839\%  \\
\hhline{|=|=|=|=|=|}
\end{tabular}
\end{center}
\end{table}

\begin{figure}[!htbp]\captionsetup{justification=centering}
\begin{center}
%mbb app1 and app2
\begin{subfigure}[t]{18pc}\centering\includegraphics[width=\textwidth]{figures/truth/ovrm/tru_ovrm_plain_mbbmain_zh125}\caption{}\end{subfigure}
\begin{subfigure}[t]{18pc}\centering\includegraphics[width=\textwidth]{figures/truth/ovrm/tru_ovrm_plain_mbbmain_zjets}\caption{}\end{subfigure}
\caption{\label{fig:mbbplain}The truth-level $m_{bb}$ distributions for the no trim, include $\mu,\nu$ case, $e,\mu$ removal.}
\end{center}
\end{figure}
\begin{figure}[!htbp]\captionsetup{justification=centering}
\begin{center}
%2mw rho S,B,SB
\begin{subfigure}[t]{18pc}\centering\includegraphics[width=\textwidth]{figures/truth/ovrm/tru_ovrm_plain_ntel11_zh125_rho_ptall_lo120_hi130}\caption{}\end{subfigure}
\begin{subfigure}[t]{18pc}\centering\includegraphics[width=\textwidth]{figures/truth/ovrm/tru_ovrm_plain_ntel11_zjets_rho_ptall_lo120_hi130}\caption{}\end{subfigure}
\begin{subfigure}[t]{18pc}\centering\includegraphics[width=\textwidth]{figures/truth/ovrm/tru_ovrm_plain_ntel11_zjets_rhoSB_ptall_lo120_hi130}\caption{}\end{subfigure}
\caption{\label{fig:mbbplain}The truth-level $\rho$ and double $m_{bb}$ distributions for the no trim, include $\mu,\nu$ case, $e,\mu$ removal.}
\end{center}
\end{figure}
\begin{figure}[!htbp]\captionsetup{justification=centering}
\begin{center}
%2mw rho S,B,SB
\begin{subfigure}[t]{18pc}\centering\includegraphics[width=\textwidth]{figures/truth/ovrm/tru_ovrm_plain_ntel11_sdb_rho_lo120_hi130_zjets}\caption{}\end{subfigure}
\begin{subfigure}[t]{18pc}\centering\includegraphics[width=\textwidth]{figures/truth/ovrm/tru_ovrm_plain_ntel11_sdb_z_lo120_hi130_zjets}\caption{}\end{subfigure}
\begin{subfigure}[t]{18pc}\centering\includegraphics[width=\textwidth]{figures/truth/ovrm/tru_ovrm_plain_ntel11_sdb_r4_lo115_hi130_r5_lo115_hi130_zjets}\caption{}\end{subfigure}
\caption{\label{fig:mbbplain}The truth-level $S/(\delta B)$ distributions for the no trim, include $\mu,\nu$ case, $e,\mu$ removal.}
\end{center}
\end{figure}

\begin{figure}[!htbp]\captionsetup{justification=centering}
\begin{center}
%mbb app1 and app2
\begin{subfigure}[t]{18pc}\centering\includegraphics[width=\textwidth]{figures/truth/ovrm/tru_ovrm_trim_mbbmain_zh125}\caption{}\end{subfigure}
\begin{subfigure}[t]{18pc}\centering\includegraphics[width=\textwidth]{figures/truth/ovrm/tru_ovrm_trim_mbbmain_zjets}\caption{}\end{subfigure}
\caption{\label{fig:mbbtrim}The truth-level $m_{bb}$ distributions for the no trim, include $\mu,\nu$ case, $e,\mu$ removal.}
\end{center}
\end{figure}
\begin{figure}[!htbp]\captionsetup{justification=centering}
\begin{center}
%2mw rho S,B,SB
\begin{subfigure}[t]{18pc}\centering\includegraphics[width=\textwidth]{figures/truth/ovrm/tru_ovrm_trim_ntel11_zh125_rho_ptall_lo115_hi130}\caption{}\end{subfigure}
\begin{subfigure}[t]{18pc}\centering\includegraphics[width=\textwidth]{figures/truth/ovrm/tru_ovrm_trim_ntel11_zjets_rho_ptall_lo115_hi130}\caption{}\end{subfigure}
\begin{subfigure}[t]{18pc}\centering\includegraphics[width=\textwidth]{figures/truth/ovrm/tru_ovrm_trim_ntel11_zjets_rhoSB_ptall_lo115_hi130}\caption{}\end{subfigure}
\caption{\label{fig:mbbtrim}The truth-level $\rho$ and double $m_{bb}$ distributions for the no trimmed, include $\mu,\nu$ case, $e,\mu$ removal.}
\end{center}
\end{figure}
\begin{figure}[!htbp]\captionsetup{justification=centering}
\begin{center}
%2mw rho S,B,SB
\begin{subfigure}[t]{18pc}\centering\includegraphics[width=\textwidth]{figures/truth/ovrm/tru_ovrm_trim_ntel11_sdb_rho_lo115_hi130_zjets}\caption{}\end{subfigure}
\begin{subfigure}[t]{18pc}\centering\includegraphics[width=\textwidth]{figures/truth/ovrm/tru_ovrm_trim_ntel11_sdb_z_lo115_hi130_zjets}\caption{}\end{subfigure}
\begin{subfigure}[t]{18pc}\centering\includegraphics[width=\textwidth]{figures/truth/ovrm/tru_ovrm_trim_ntel11_sdb_r4_lo110_hi130_r7_lo115_hi130_zjets}\caption{}\end{subfigure}
\caption{\label{fig:mbbtrim}The truth-level $S/(\delta B)$ distributions for the trimmed, include $\mu,\nu$ case, $e,\mu$ removal.}
\end{center}
\end{figure}

\begin{figure}[!htbp]\captionsetup{justification=centering}
\begin{center}
%mbb app1 and app2
\begin{subfigure}[t]{18pc}\centering\includegraphics[width=\textwidth]{figures/truth/ovrm/tru_ovrm_wz_mbbmain_zh125}\caption{}\end{subfigure}
\begin{subfigure}[t]{18pc}\centering\includegraphics[width=\textwidth]{figures/truth/ovrm/tru_ovrm_wz_mbbmain_zjets}\caption{}\end{subfigure}
\caption{\label{fig:mbbwzovrm}The truth-level $m_{bb}$ distributions for the no trim, don't include $\mu,\nu$ case.}
\end{center}
\end{figure}
\begin{figure}[!htbp]\captionsetup{justification=centering}
\begin{center}
%2mw rho S,B,SB
\begin{subfigure}[t]{18pc}\centering\includegraphics[width=\textwidth]{figures/truth/ovrm/tru_ovrm_wz_ntel11_zh125_rho_ptall_lo120_hi130}\caption{}\end{subfigure}
\begin{subfigure}[t]{18pc}\centering\includegraphics[width=\textwidth]{figures/truth/ovrm/tru_ovrm_wz_ntel11_zjets_rho_ptall_lo120_hi130}\caption{}\end{subfigure}
\begin{subfigure}[t]{18pc}\centering\includegraphics[width=\textwidth]{figures/truth/ovrm/tru_ovrm_wz_ntel11_zjets_rhoSB_ptall_lo120_hi130}\caption{}\end{subfigure}
\caption{\label{fig:rhowzovrm}The truth-level $\rho$ and double $m_{bb}$ distributions for no trim, don't include $\mu,\nu$ case, $e,\mu$ removal.}
\end{center}
\end{figure}
\begin{figure}[!htbp]\captionsetup{justification=centering}
\begin{center}
%2mw rho S,B,SB
\begin{subfigure}[t]{18pc}\centering\includegraphics[width=\textwidth]{figures/truth/ovrm/tru_ovrm_wz_ntel11_sdb_rho_lo120_hi130_zjets}\caption{}\end{subfigure}
\begin{subfigure}[t]{18pc}\centering\includegraphics[width=\textwidth]{figures/truth/ovrm/tru_ovrm_wz_ntel11_sdb_z_lo120_hi130_zjets}\caption{}\end{subfigure}
\begin{subfigure}[t]{18pc}\centering\includegraphics[width=\textwidth]{figures/truth/ovrm/tru_ovrm_wz_ntel11_sdb_r4_lo100_hi135_r5_lo95_hi130_zjets}\caption{}\end{subfigure}
\caption{\label{fig:sdbwzovrm}The truth-level $S/(\delta B)$ distributions for the no trim, don't include $\mu,\nu$ case, $e,\mu$ removal.}
\end{center}
\end{figure}


\begin{figure}[!htbp]\captionsetup{justification=centering}
\begin{center}
%mbb app1 and app2
\begin{subfigure}[t]{18pc}\centering\includegraphics[width=\textwidth]{figures/truth/ovrm/tru_ovrm_wztrim_mbbmain_zh125}\caption{}\end{subfigure}
\begin{subfigure}[t]{18pc}\centering\includegraphics[width=\textwidth]{figures/truth/ovrm/tru_ovrm_wztrim_mbbmain_zjets}\caption{}\end{subfigure}
\caption{\label{fig:mbbwztrimovrm}The truth-level $m_{bb}$ distributions for the trimmed, don't include $\mu,\nu$ case, $e,\mu$ removal.}
\end{center}
\end{figure}
\begin{figure}[!htbp]\captionsetup{justification=centering}
\begin{center}
%2mw rho S,B,SB
\begin{subfigure}[t]{18pc}\centering\includegraphics[width=\textwidth]{figures/truth/ovrm/tru_ovrm_wztrim_ntel11_zh125_rho_ptall_lo110_hi130}\caption{}\end{subfigure}
\begin{subfigure}[t]{18pc}\centering\includegraphics[width=\textwidth]{figures/truth/ovrm/tru_ovrm_wztrim_ntel11_zjets_rho_ptall_lo110_hi130}\caption{}\end{subfigure}
\begin{subfigure}[t]{18pc}\centering\includegraphics[width=\textwidth]{figures/truth/ovrm/tru_ovrm_wztrim_ntel11_zjets_rhoSB_ptall_lo110_hi130}\caption{}\end{subfigure}
\caption{\label{fig:mbbwztrim}The truth-level $\rho$ and double $m_{bb}$ distributions for trimmed, don't include $\mu,\nu$ case, $e,\mu$ removal.}
\end{center}
\end{figure}
\begin{figure}[!htbp]\captionsetup{justification=centering}
\begin{center}
%2mw rho S,B,SB
\begin{subfigure}[t]{18pc}\centering\includegraphics[width=\textwidth]{figures/truth/ovrm/tru_ovrm_wztrim_ntel11_sdb_rho_lo110_hi130_zjets}\caption{}\end{subfigure}
\begin{subfigure}[t]{18pc}\centering\includegraphics[width=\textwidth]{figures/truth/ovrm/tru_ovrm_wztrim_ntel11_sdb_z_lo105_hi130_zjets}\caption{}\end{subfigure}
\begin{subfigure}[t]{18pc}\centering\includegraphics[width=\textwidth]{figures/truth/ovrm/tru_ovrm_wztrim_ntel11_sdb_r4_lo100_hi130_r5_lo95_hi130_zjets}\caption{}\end{subfigure}
\caption{\label{fig:mbbwztrim}The truth-level $S/(\delta B)$ distributions for the trimmed, don't include $\mu,\nu$ case, $e,\mu$ removal.}
\end{center}
\end{figure}
\clearpage

\section{Removal Over Electrons Only}
\begin{table}[htbp]
\caption{$m_{bb}$ windows studied.  These windows were chosen to optimize significances over all $p_T^Z$.
\label{tab:masswindow}}
\begin{center}
\begin{tabular}{|c|c|p{5cm}|p{5cm}|c|}
\hhline{|=|=|=|=|=|}
Type& \# $R$'s &              $S/\delta B$ Type                                  &           Optimal $m_{bb}$ Window     &   Improvement\\
\hhline{|=|=|=|=|=|}
     incl. $\mu,\nu$& ---  &                                                  anti-$k_t$ $R=0.4$ &                        110--130 \GeV\ &     ---  \\
\hhline{|~|~|-|-|-|}
                    &   5  &                                                 $t\left(z\right)=z$ &                        120--130 \GeV\ &  23.48\%  \\
                    &      &         $t\left(z\right)=\rho_S\left(z\right)/\rho_B\left(z\right)$ &                        120--130 \GeV\ &  27.62\%  \\
\hhline{|~|~|-|-|-|}
                    &   7  &                                                 $t\left(z\right)=z$ &                        120--130 \GeV\ &   21.3\%  \\
                    &      &         $t\left(z\right)=\rho_S\left(z\right)/\rho_B\left(z\right)$ &                        120--130 \GeV\ &   27.2\%  \\
\hhline{|~|~|-|-|-|}
                    &  12  &                                                 $t\left(z\right)=z$ &                        120--130 \GeV\ &  11.21\%  \\
                    &      &         $t\left(z\right)=\rho_S\left(z\right)/\rho_B\left(z\right)$ &                        120--130 \GeV\ &  17.78\%  \\
                    &      &                             anti-$k_t$ $R=0.4$, telescoping $R=0.5$ &        115--130 \GeV, 115--130 \GeV\ &   5.69\%  \\
\hhline{|=|=|=|=|=|}
incl. $\mu,\nu$, trim& ---  &                                                  anti-$k_t$ $R=0.4$ &                        110--130 \GeV\ &     ---  \\
\hhline{|~|~|-|-|-|}
                    &   5  &                                                 $t\left(z\right)=z$ &                        115--130 \GeV\ &  9.314\%  \\
                    &      &         $t\left(z\right)=\rho_S\left(z\right)/\rho_B\left(z\right)$ &                        115--130 \GeV\ &  12.24\%  \\
\hhline{|~|~|-|-|-|}
                    &   7  &                                                 $t\left(z\right)=z$ &                        115--130 \GeV\ &  11.58\%  \\
                    &      &         $t\left(z\right)=\rho_S\left(z\right)/\rho_B\left(z\right)$ &                        115--130 \GeV\ &  14.82\%  \\
\hhline{|~|~|-|-|-|}
                    &  12  &                                                 $t\left(z\right)=z$ &                        115--130 \GeV\ &  12.36\%  \\
                    &      &         $t\left(z\right)=\rho_S\left(z\right)/\rho_B\left(z\right)$ &                        115--130 \GeV\ &  17.07\%  \\
                    &      &                             anti-$k_t$ $R=0.4$, telescoping $R=0.7$ &        110--130 \GeV, 115--130 \GeV\ &  3.143\%  \\
\hhline{|=|=|=|=|=|}
        no $\mu,\nu$& ---  &                                                  anti-$k_t$ $R=0.4$ &                        100--130 \GeV\ &     ---  \\
\hhline{|~|~|-|-|-|}
                    &   5  &                                                 $t\left(z\right)=z$ &                        115--130 \GeV\ &  15.69\%  \\
                    &      &         $t\left(z\right)=\rho_S\left(z\right)/\rho_B\left(z\right)$ &                        115--130 \GeV\ &  18.94\%  \\
                    &      &                             anti-$k_t$ $R=0.4$, telescoping $R=0.5$ &         100--135 \GeV, 95--130 \GeV\ &  3.685\%  \\
\hhline{|~|~|-|-|-|}
                    &   7  &                                                 $t\left(z\right)=z$ &                        120--130 \GeV\ &   18.4\%  \\
                    &      &         $t\left(z\right)=\rho_S\left(z\right)/\rho_B\left(z\right)$ &                        120--130 \GeV\ &  21.46\%  \\
\hhline{|~|~|-|-|-|}
                    &  12  &                                                 $t\left(z\right)=z$ &                        120--130 \GeV\ &  18.33\%  \\
                    &      &         $t\left(z\right)=\rho_S\left(z\right)/\rho_B\left(z\right)$ &                        120--130 \GeV\ &  22.31\%  \\
                    &      &                             anti-$k_t$ $R=0.4$, telescoping $R=0.9$ &        100--130 \GeV, 105--150 \GeV\ &  4.087\%  \\
\hhline{|=|=|=|=|=|}
  no $\mu,\nu$, trim& ---  &                                                  anti-$k_t$ $R=0.4$ &                        100--130 \GeV\ &     ---  \\
\hhline{|~|~|-|-|-|}
                    &   5  &                                                 $t\left(z\right)=z$ &                        105--130 \GeV\ &  7.272\%  \\
                    &      &         $t\left(z\right)=\rho_S\left(z\right)/\rho_B\left(z\right)$ &                        105--130 \GeV\ &  9.952\%  \\
                    &      &                             anti-$k_t$ $R=0.4$, telescoping $R=0.8$ &        100--130 \GeV, 105--130 \GeV\ &  3.913\%  \\
\hhline{|~|~|-|-|-|}
                    &   7  &                                                 $t\left(z\right)=z$ &                        105--130 \GeV\ &  9.982\%  \\
                    &      &         $t\left(z\right)=\rho_S\left(z\right)/\rho_B\left(z\right)$ &                        105--130 \GeV\ &  13.49\%  \\
\hhline{|~|~|-|-|-|}
                    &  12  &                                                 $t\left(z\right)=z$ &                        105--130 \GeV\ &  13.52\%  \\
                    &      &         $t\left(z\right)=\rho_S\left(z\right)/\rho_B\left(z\right)$ &                        110--130 \GeV\ &  19.26\%  \\
                    &      &                             anti-$k_t$ $R=0.4$, telescoping $R=0.9$ &        100--130 \GeV, 105--130 \GeV\ &   4.56\%  \\
\hhline{|=|=|=|=|=|}
\end{tabular}
\end{center}
\end{table}


\begin{figure}[!htbp]\captionsetup{justification=centering}
\begin{center}
%mbb app1 and app2
\begin{subfigure}[t]{18pc}\centering\includegraphics[width=\textwidth]{figures/truth/ov_rmel/tru_ov_rmel_plain_mbbmain_zh125}\caption{}\end{subfigure}
\begin{subfigure}[t]{18pc}\centering\includegraphics[width=\textwidth]{figures/truth/ov_rmel/tru_ov_rmel_plain_mbbmain_zjets}\caption{}\end{subfigure}
\caption{\label{fig:mbbplainovrmel1}The truth-level $m_{bb}$ distributions for the no trim, include $\mu,\nu$ casecase, $e$ removal..}
\end{center}
\end{figure}
\begin{figure}[!htbp]\captionsetup{justification=centering}
\begin{center}
%2mw rho S,B,SB
\begin{subfigure}[t]{18pc}\centering\includegraphics[width=\textwidth]{figures/truth/ov_rmel/tru_ov_rmel_plain_ntel11_zh125_rho_ptall_lo120_hi130}\caption{}\end{subfigure}
\begin{subfigure}[t]{18pc}\centering\includegraphics[width=\textwidth]{figures/truth/ov_rmel/tru_ov_rmel_plain_ntel11_zjets_rho_ptall_lo120_hi130}\caption{}\end{subfigure}
\begin{subfigure}[t]{18pc}\centering\includegraphics[width=\textwidth]{figures/truth/ov_rmel/tru_ov_rmel_plain_ntel11_zjets_rhoSB_ptall_lo120_hi130}\caption{}\end{subfigure}
\caption{\label{fig:rhoplainovrm1}The truth-level $rho$ and double $m_{bb}$ distributions for the no trim, include $\mu,\nu$ case, $e$ removal.}
\end{center}
\end{figure}
\begin{figure}[!htbp]\captionsetup{justification=centering}
\begin{center}
%2mw rho S,B,SB
\begin{subfigure}[t]{18pc}\centering\includegraphics[width=\textwidth]{figures/truth/ov_rmel/tru_ov_rmel_plain_ntel11_sdb_rho_lo120_hi130_zjets}\caption{}\end{subfigure}
\begin{subfigure}[t]{18pc}\centering\includegraphics[width=\textwidth]{figures/truth/ov_rmel/tru_ov_rmel_plain_ntel11_sdb_z_lo120_hi130_zjets}\caption{}\end{subfigure}
\begin{subfigure}[t]{18pc}\centering\includegraphics[width=\textwidth]{figures/truth/ov_rmel/tru_ov_rmel_plain_ntel11_sdb_r4_lo115_hi130_r5_lo115_hi130_zjets}\caption{}\end{subfigure}
\caption{\label{fig:mbbplain}The truth-level $S/(\delta B)$ distributions for the no trim, include $\mu,\nu$ case, $e$ removal.}
\end{center}
\end{figure}

\begin{figure}[!htbp]\captionsetup{justification=centering}
\begin{center}
%mbb app1 and app2
\begin{subfigure}[t]{18pc}\centering\includegraphics[width=\textwidth]{figures/truth/ov_rmel/tru_ov_rmel_trim_mbbmain_zh125}\caption{}\end{subfigure}
\begin{subfigure}[t]{18pc}\centering\includegraphics[width=\textwidth]{figures/truth/ov_rmel/tru_ov_rmel_trim_mbbmain_zjets}\caption{}\end{subfigure}
\caption{\label{fig:mbbtrim}The truth-level $m_{bb}$ distributions for the no trim, include $\mu,\nu$ case, $e$ removal.}
\end{center}
\end{figure}
\begin{figure}[!htbp]\captionsetup{justification=centering}
\begin{center}
%2mw rho S,B,SB
\begin{subfigure}[t]{18pc}\centering\includegraphics[width=\textwidth]{figures/truth/ov_rmel/tru_ov_rmel_trim_ntel11_zh125_rho_ptall_lo115_hi130}\caption{}\end{subfigure}
\begin{subfigure}[t]{18pc}\centering\includegraphics[width=\textwidth]{figures/truth/ov_rmel/tru_ov_rmel_trim_ntel11_zjets_rho_ptall_lo115_hi130}\caption{}\end{subfigure}
\begin{subfigure}[t]{18pc}\centering\includegraphics[width=\textwidth]{figures/truth/ov_rmel/tru_ov_rmel_trim_ntel11_zjets_rhoSB_ptall_lo115_hi130}\caption{}\end{subfigure}
\caption{\label{fig:mbbtrim}The truth-level $rho$ and double $m_{bb}$ distributions for the no trimmed, include $\mu,\nu$ case, $e$ removal.}
\end{center}
\end{figure}
\begin{figure}[!htbp]\captionsetup{justification=centering}
\begin{center}
%2mw rho S,B,SB
\begin{subfigure}[t]{18pc}\centering\includegraphics[width=\textwidth]{figures/truth/ov_rmel/tru_ov_rmel_trim_ntel11_sdb_rho_lo115_hi130_zjets}\caption{}\end{subfigure}
\begin{subfigure}[t]{18pc}\centering\includegraphics[width=\textwidth]{figures/truth/ov_rmel/tru_ov_rmel_trim_ntel11_sdb_z_lo115_hi130_zjets}\caption{}\end{subfigure}
\begin{subfigure}[t]{18pc}\centering\includegraphics[width=\textwidth]{figures/truth/ov_rmel/tru_ov_rmel_trim_ntel11_sdb_r4_lo110_hi130_r7_lo115_hi130_zjets}\caption{}\end{subfigure}
\caption{\label{fig:mbbtrim}The truth-level $S/(\delta B)$ distributions for the trimmed, include $\mu,\nu$ case, $e$ removal.}
\end{center}
\end{figure}

\begin{figure}[!htbp]\captionsetup{justification=centering}
\begin{center}
%mbb app1 and app2
\begin{subfigure}[t]{18pc}\centering\includegraphics[width=\textwidth]{figures/truth/ov_rmel/tru_ov_rmel_wz_mbbmain_zh125}\caption{}\end{subfigure}
\begin{subfigure}[t]{18pc}\centering\includegraphics[width=\textwidth]{figures/truth/ov_rmel/tru_ov_rmel_wz_mbbmain_zjets}\caption{}\end{subfigure}
\caption{\label{fig:mbbwz}The truth-level $m_{bb}$ distributions for the no trim, don't include $\mu,\nu$ case, $e$ removal.}
\end{center}
\end{figure}
\begin{figure}[!htbp]\captionsetup{justification=centering}
\begin{center}
%2mw rho S,B,SB
\begin{subfigure}[t]{18pc}\centering\includegraphics[width=\textwidth]{figures/truth/ov_rmel/tru_ov_rmel_wz_ntel11_zh125_rho_ptall_lo120_hi130}\caption{}\end{subfigure}
\begin{subfigure}[t]{18pc}\centering\includegraphics[width=\textwidth]{figures/truth/ov_rmel/tru_ov_rmel_wz_ntel11_zjets_rho_ptall_lo120_hi130}\caption{}\end{subfigure}
\begin{subfigure}[t]{18pc}\centering\includegraphics[width=\textwidth]{figures/truth/ov_rmel/tru_ov_rmel_wz_ntel11_zjets_rhoSB_ptall_lo120_hi130}\caption{}\end{subfigure}
\caption{\label{fig:mbbwz}The truth-level $rho$ and double $m_{bb}$ distributions for no trim, don't include $\mu,\nu$ case, $e$ removal.}
\end{center}
\end{figure}
\begin{figure}[!htbp]\captionsetup{justification=centering}
\begin{center}
%2mw rho S,B,SB
\begin{subfigure}[t]{18pc}\centering\includegraphics[width=\textwidth]{figures/truth/ov_rmel/tru_ov_rmel_wz_ntel11_sdb_rho_lo120_hi130_zjets}\caption{}\end{subfigure}
\begin{subfigure}[t]{18pc}\centering\includegraphics[width=\textwidth]{figures/truth/ov_rmel/tru_ov_rmel_wz_ntel11_sdb_z_lo120_hi130_zjets}\caption{}\end{subfigure}
\begin{subfigure}[t]{18pc}\centering\includegraphics[width=\textwidth]{figures/truth/ov_rmel/tru_ov_rmel_wz_ntel11_sdb_r4_lo100_hi130_r9_lo105_hi150_zjets}\caption{}\end{subfigure}
\caption{\label{fig:mbbwz}The truth-level $S/(\delta B)$ distributions for the no trim, don't include $\mu,\nu$ case, $e$ removal.}
\end{center}
\end{figure}


\begin{figure}[!htbp]\captionsetup{justification=centering}
\begin{center}
%mbb app1 and app2
\begin{subfigure}[t]{18pc}\centering\includegraphics[width=\textwidth]{figures/truth/ov_rmel/tru_ov_rmel_wztrim_mbbmain_zh125}\caption{}\end{subfigure}
\begin{subfigure}[t]{18pc}\centering\includegraphics[width=\textwidth]{figures/truth/ov_rmel/tru_ov_rmel_wztrim_mbbmain_zjets}\caption{}\end{subfigure}
\caption{\label{fig:mbbwztrim}The truth-level $m_{bb}$ distributions for the trimmed, don't include $\mu,\nu$ case, $e$ removal.}
\end{center}
\end{figure}
\begin{figure}[!htbp]\captionsetup{justification=centering}
\begin{center}
%2mw rho S,B,SB
\begin{subfigure}[t]{18pc}\centering\includegraphics[width=\textwidth]{figures/truth/ov_rmel/tru_ov_rmel_wztrim_ntel11_zh125_rho_ptall_lo110_hi130}\caption{}\end{subfigure}
\begin{subfigure}[t]{18pc}\centering\includegraphics[width=\textwidth]{figures/truth/ov_rmel/tru_ov_rmel_wztrim_ntel11_zjets_rho_ptall_lo110_hi130}\caption{}\end{subfigure}
\begin{subfigure}[t]{18pc}\centering\includegraphics[width=\textwidth]{figures/truth/ov_rmel/tru_ov_rmel_wztrim_ntel11_zjets_rhoSB_ptall_lo110_hi130}\caption{}\end{subfigure}
\begin{subfigure}[t]{18pc}\centering\includegraphics[width=\textwidth]{figures/truth/ov_rmel/tru_ov_rmel_wztrim_zh125_mbbRv4_ptall_r9}\caption{}\end{subfigure}
\begin{subfigure}[t]{18pc}\centering\includegraphics[width=\textwidth]{figures/truth/ov_rmel/tru_ov_rmel_wztrim_zjets_mbbRv4_ptall_r9}\caption{}\end{subfigure}
\caption{\label{fig:mbbwztrim}The truth-level $rho$ and double $m_{bb}$ distributions for trimmed, don't include $\mu,\nu$ case, $e$ removal.}
\end{center}
\end{figure}
\begin{figure}[!htbp]\captionsetup{justification=centering}
\begin{center}
%2mw rho S,B,SB
\begin{subfigure}[t]{18pc}\centering\includegraphics[width=\textwidth]{figures/truth/ov_rmel/tru_ov_rmel_wztrim_ntel11_sdb_rho_lo110_hi130_zjets}\caption{}\end{subfigure}
\begin{subfigure}[t]{18pc}\centering\includegraphics[width=\textwidth]{figures/truth/ov_rmel/tru_ov_rmel_wztrim_ntel11_sdb_z_lo105_hi130_zjets}\caption{}\end{subfigure}
\begin{subfigure}[t]{18pc}\centering\includegraphics[width=\textwidth]{figures/truth/ov_rmel/tru_ov_rmel_wztrim_ntel11_sdb_r4_lo100_hi130_r9_lo105_hi130_zjets}\caption{}\end{subfigure}
\caption{\label{fig:mbbwztrim}The truth-level $S/(\delta B)$ distributions for the trimmed, don't include $\mu,\nu$ case, $e$ removal.}
\end{center}
\end{figure}


\clearpage
\section{Truth-Level Plots for Pure Telescoping Jets}
I did a quick study where all of the truth particles included in the truth-level telescoping jets must come from a $b$-quark (or $\bar{b}$).  The results are as impressive as one might expect them to be. Of course this situation is far better than anything one could ever expect, but, for me at least, it has been a good illustration of what is possible and heuristic lesson.  The best improvement in all the trim/no-trim, include/don't include $\mu,\nu$ is in the double $m_{bb}$ cut using the largest telescoping radius, which makes sense because perfect knowledge of whether or not truth-particles come from $b$'s should negate most of the downside of a bigger jet radius in signal.  I only present the case that uses $\mu$'s and $\nu$'s.  Trimming, naturally, makes these numbers far less astronomical, and not including the $\mu$'s and $\nu$'s also has a detrimental effect, though not so large.  Moreover, the improvements showed marked improvements with larger numbers of telescoping radii.  These results suggest that the degradation with increased $R$ in the previous section are a result of the $m_{bb}$ distributions looking too alike at high $R$ and not some bug.  Perhaps it would be worthwhile to make a cut on telescoping jet ``$b$-purity'' (it wouldn't be too much trouble for me to do this and test out values; the largest time sink would just be to remake n-tuples---I should be able to code that in a day or so).

\begin{figure}[!htbp]\captionsetup{justification=centering}
\begin{center}
%2mw rho S,B,SB
\begin{subfigure}[t]{18pc}\centering\includegraphics[width=\textwidth]{figures/truth/bcheck/tru_bchk_plain_mbbapp1_zh125}\caption{}\end{subfigure}
\begin{subfigure}[t]{18pc}\centering\includegraphics[width=\textwidth]{figures/truth/bcheck/tru_bchk_plain_mbbapp2_zh125}\caption{}\end{subfigure}
\begin{subfigure}[t]{18pc}\centering\includegraphics[width=\textwidth]{figures/truth/bcheck/tru_bchk_plain_mbbapp1_zjets}\caption{}\end{subfigure}
\begin{subfigure}[t]{18pc}\centering\includegraphics[width=\textwidth]{figures/truth/bcheck/tru_bchk_plain_mbbapp2_zjets}\caption{}\end{subfigure}
\caption{\label{fig:sdbbchkplain}The $m_{bb}$ distributions for the untrimmed, don't include $\mu,\nu$ case.}
\end{center}
\end{figure}

\begin{figure}[!htbp]\captionsetup{justification=centering}
\begin{center}
%2mw rho S,B,SB
\begin{subfigure}[t]{18pc}\centering\includegraphics[width=\textwidth]{figures/truth/bcheck/tru_bchk_plain_ntel11_zh125_rho_ptall_lo120_hi130}\caption{}\end{subfigure}
\begin{subfigure}[t]{18pc}\centering\includegraphics[width=\textwidth]{figures/truth/bcheck/tru_bchk_plain_ntel11_zjets_rho_ptall_lo120_hi130}\caption{}\end{subfigure}
\begin{subfigure}[t]{18pc}\centering\includegraphics[width=\textwidth]{figures/truth/bcheck/tru_bchk_plain_ntel11_zjets_rhoSB_ptall_lo120_hi130}\caption{}\end{subfigure}
\begin{subfigure}[t]{18pc}\centering\includegraphics[width=\textwidth]{figures/truth/bcheck/tru_bchk_plain_zh125_mbbRv4_ptall_r15}\caption{}\end{subfigure}
\begin{subfigure}[t]{18pc}\centering\includegraphics[width=\textwidth]{figures/truth/bcheck/tru_bchk_plain_zjets_mbbRv4_ptall_r15}\caption{}\end{subfigure}
\caption{\label{fig:sdbbchkplain}The $\rho\left(z\right)$ and double $m_{bb}$ distributions for the untrimmed, don't include $\mu,\nu$ case.}
\end{center}
\end{figure}

\begin{figure}[!htbp]\captionsetup{justification=centering}
\begin{center}
%2mw rho S,B,SB
\begin{subfigure}[t]{18pc}\centering\includegraphics[width=\textwidth]{figures/truth/bcheck/tru_bchk_plain_ntel11_sdb_z_lo120_hi130_zjets}\caption{}\end{subfigure}
\begin{subfigure}[t]{18pc}\centering\includegraphics[width=\textwidth]{figures/truth/bcheck/tru_bchk_plain_ntel11_sdb_rho_lo120_hi130_zjets}\caption{}\end{subfigure}
\begin{subfigure}[t]{18pc}\centering\includegraphics[width=\textwidth]{figures/truth/bcheck/tru_bchk_plain_ntel11_sdb_r4_lo120_hi130_r15_lo120_hi130_zjets}\caption{}\end{subfigure}
\caption{\label{fig:sdbbchkplain}The $S/(\delta B)$ distributions for the untrimmed, don't include $\mu,\nu$ case.  Note for (b), the total improvement is 1221\% (there's some bug, but I thought it was more important to get these plots to you than hunt it down).}
\end{center}
\end{figure}


%\section{$m_{bb}$ Plots}
%The plots by $p_T$ bin.

%\section{Monte Carlo Dataset Names}


W.~Verkerke and D.~Kirkby, {\em {The RooFit toolkit for data modeling}}, in
  {\em 2003 Computing in High Energy and Nuclear Physics, CHEP03}.
\newblock 2003.
\newblock \href{http://arxiv.org/abs/0306116}{{\ttfamily
  arXiv:physics:0306116}}.


\appendix
\clearpage
\section{Comparisons between fits to gaussian, Novosibirsk and Bukin functions}\label{sec:fitsApp}

\end{comment}
